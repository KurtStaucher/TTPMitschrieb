\chapter{Groups and Symmetries}
\section{Representation of Groups}
A group is an object of the form $G = \{ g_i | i = 1,\ldots\}$ where $g_i$ are the elements of the group. In the group a multiplication exists so, that $g_1 g_2 = g_3 \in G$. This also has the following traits:
\begin{itemize}
\item it has to be associative: $g_1(g_2g_3) = (g_1g_2)g_3$
\item there is an identity element $e \equiv 1$ so that $g\cc e = e \cc g = g ~~\forall g \in G$
\item $\forall g \in G$ there is an inverse $g^{-1} \in G$: $gg^{-1} = g^{-1}g = e$
\end{itemize}
The representation of $G$ is a mapping $r: G \df \mathbbm{C}^{(n,n)}$ where $r(g_i) = M_i$ is a $n\times n$-matrix.\\
With this representation the multiplication rules are being preserved: $r(g_1g_2) = r(g_1)r(g_2)$ or $M_3 = M_1M_2$. Also $r(e) = \ehm_n$ and $r(g^{-1}g) = MM^{-1} = \ehm$.\\
A reducable representation is a representation such that a single unitary $n\times n$ matrix $U$ exists such that
\[ U M_i U^{-1} = \Matz{M_i'}{0}{0}{M_i''}\]
So instead of working with $M_i$ we could have worked with the block diagonal matrices $M_i'$ and $M_i''$.\\
An irreducable representation then is a representation where no unitary matrix exists that splits the matrix $M_i$ into block diagonal matrices.\\
For finite groups $G = \{ g_i | i = 1, \ldots, n \}$ the dimensions $d_r$ of all the irreducible representations are bounded by 
\[n = \sum_r d_r^2\]
Therefore an infinite group has infinte number of different finite dimensional irreducible representations.
\section{Lie Groups and Lie Algebra}
Lie groups are a special case of groups. They are parametrizised by $G = \{ U(\theta)| \theta = ( \theta_1, \ldots, \theta_n) \in \mathbbm{R}^n\}$ with $U(0) = e$. \\
$U(\theta)$ is analytic in all its components; it is infinitely differentiable.\\
The simplest example of a Lie group is the three dimensional rotation group $SO(3)$ with the rotation matrices $R(\phi,\psi,\theta)$. The three rotations are\\
the rotation around the $z$-axis:
\[ R_z(\theta) = \Mat{\cos\theta}{\sin\theta}{0}{- \sin\theta}{\cos\theta}{0}{0}{0}{1}\]
the rotation around the $y$-axis:
\[ R_y(\psi) = \Mat{\cos\psi}{0}{-\sin\psi}{0}{1}{0}{\sin\psi}{0}{\cos\psi}\]
the rotation around the $x$-axis:
\[ R_x(\phi) = \Mat{1}{0}{0}{0}{\cos\phi}{\sin\psi}{0}{-\sin\phi}{\cos\phi}\]


\sub{Lie algebra}
As the elements of the Lie group are infinitely differentiable we can apply the definition of the Taylor expansion onto an element of the group. This leads to the generators of the group
\[L_a \equiv \frac{1}{i} \left.\frac{\p U}{\p \phi_a} \right|_{\theta = 0}\]
These generators completely describe the groups properties.\\
The Taylor expansion also leads to infinitesimal transformations:
\[ U(\theta) = 1 + i \sum_a \theta_a L_a + \ldots\]
In summation convention this also can be written as $\sum_a \theta_a L_a = \theta_a L_a = \theta\cc L$.\\
As one general group property is the formation of the $1$-element: $U(\theta)U^{-1}(\theta) = 1$ this also has to be applicable for inifitiesimal transformations. This leads to
\[ G \ni U(\theta)U(\psi)U^{-1}(\theta)U^{-1}(\psi) \neq 1\]
which is not neccesarily the $1$-element for non commuting groups. The Taylor expanison of this expression leads to
\[ 1 + i^2 \theta_a\psi_b ( L_aL_b + L_aL_b - L_aL_b - L_bL_a) + \ldots = 1 - \theta_a \psi_b [L_a, L_b]+\ldots = 1 + i \sum_c \theta_a \psi_b ( - L_c f^{abc})\]
In the first step the only terms of first order are either linear in $\psi$ or $\theta$ so they cancel. In the second order all terms quadratic in $\theta$ or $\psi$ also cancel and the only things left are mixed terms. In the second step the definition of the commutator was applied and in the last step we used, that the resulting element had to be in the group $G$ again, so it must be able to be written as a taylor expansion again. This therefore leads to the identity
\[ [ L_a, L_b] = -i L_c f^{abc}\]
Where $f^{abc}$ are group specific structure constants.\\
For the $SO(3)$ group this is for example the known $\eps$-tensor. The generators are the components of angular momentum:
\[L_x = \Mat{}{}{}{}{}{-i}{}{i}{}, ~~ L_y = \Mat{}{}{i}{}{}{}{-i}{}{}, ~~L_z = \Mat{}{-i}{}{i}{}{}{}{}{}\]
The identity here is $[L_x,L_y] = iL_z$.\\
These generators form the basis of the Lie algebra.\\
\\
Another trait of the group can be shown for a finite group with $\theta = (\theta_1, \ldots, \theta_n)$. If we use $\eps = \frac{1}{N}\theta$ then for high $N$, $\eps$ becomes small. So a Taylor expansion can be applied:
\[ U(\eps) = 1 + i \eps_a L_a = 1 + i \frac{\theta_a L_a}{N}\]
So the original $\theta$ can be written as the following
\[ U(\theta) = U(\eps)^N \df \lim_{N \df \infty} U(\eps) ^N = \lim_{N\df\infty} \left( 1 + i \frac{\theta_aL_a}{N}\right)^N = e^{i\theta_aL_a} = \sum_{k = 0}^{\infty}\frac{1}{k!}(i\theta_aL_a)^k\]
The $i$ was introduced in the definition of the generators, so the generators would be hermitian operators. If we look at $U(\theta)$, which is unitary it follows
\[ U(\theta) = e^{i\theta L} ~~\df~~ U^{-1}(\theta) = e^{-i\theta L}\]
and thus
\[ e^{-i\theta_aL_a^\dagger} = U^\dagger(\theta) = U^{-1}(\theta) = e^{-i\theta_aL_a}\]
From this we can see, that $L_a$ and $L_a^\dagger$ must be the same.


\sub{Special unitary groups}
Another group of Lie groups are the special unitary groups $SU(N) = \{ U \in\mathbbm{C}^{(N\times N)} | U^{-1} = U^\dagger, \det(U) = 1\}$. Its group elements are unitary and have a determinante of one. The number of generators $L_a$ can be expressed formally for every $N$:\\
$L_a$ has to be a hermitian $N\times N$ matrix with trace $\tr(L_a) = 0$. This follows from
\[ \det(U) = \det \left( e^{i \theta_a L_a}\right) = e^{i \tr(\theta_a L_a)} \soll 1 ~~\df ~~ \tr(L_a) = 0\]
Now because there are $N^2$ matrices but one of them neccesarily is $\ehm$ which is not traceless, there are always $N^2-1$ generators in $SU(N)$.\\
For $N = 2$ the generators are $L_a = \frac{\sigma_a}{2}$ where $\sigma_a$ are the Pauli matrices.\\
For $N = 3$ the generators are the eight Gell-Mann matrices. They have the following forms
\begin{align*} \lambda_1 &= \Mat{}{1}{}{1}{}{}{}{}{},~~~\lambda_2 = \Mat{}{-i}{}{i}{}{}{}{}{},~~~\lambda_3 = \Mat{1}{}{}{}{-1}{}{}{}{}\\
\lambda_4 &= \Mat{}{}{1}{}{}{}{1}{}{}, ~~~\lambda_5 = \Mat{}{}{-i}{}{}{}{i}{}{}\\
\lambda_6 &= \Mat{}{}{}{}{}{1}{}{1}{}, ~~~\lambda_7 = \Mat{}{}{}{}{}{-i}{}{i}{}, ~~~\lambda_8 = \frac{1}{\sqrt{3}} \Mat{1}{}{}{}{1}{}{}{}{-2}
\end{align*}
These matrices have normalization conditions which also apply for abitrary $N$:
\[ \tr(\lambda^a \lambda^b) = 2\delta^{ab}, ~~~ L_a = \frac{\lambda_a}{2}\]


~\newline\newline\sub{Rank of the Lie algebra}
The Lie algebra also has a rank. The rank is defined as the maximum number of commuting generators in the algebra. $SU(N)$ has $N-1$ diagonal generators, which naturally all commute. So the rank of an arbitraty $SU(N)$ algebra is always $N-1$.\\
These $N-1$ eigenvalues of these $L_a$ then specify the basis states in a $SU(N)$ multiplet.\\
For example for $N = 2$ the rank is one, so it has only one invariant, which is $\vv{J}^2$ so that $[\vv{J}^2, J_i] = 0$.\\
Generally, if we have any polynomial such that $C = \eta_{ab} L_a L_b + \eta_{abc} L_aL_bL_c + \ldots$ that commutes with all generators, $[C,L_a] = 0 ~\forall a$ then it is called a Casimir invariant.\\
For $SU(2)$ the only Casimir invariant is $\vv{J}^2$, for $SU(3)$ there are two different Casimir invariants.\\
For any $SU(N)$ there is a Casimir invariant $C_2 = L_aL_a$ which is also referred to as the quadratic Casimir of the $SU(N)$ group.\\
Furhtermore, the eigenvalues of all independant Casimir invariants do specify an irreducible representation.


\sub{Adjoint irreducible representation}
If we take a look at the Jacobi identity
\[ [A,[B,C]] + [B,[C,A]] + [C,[A,B]] = 0\]
and take $A = L_a$, $B= L_b$, $C=L_j$ we get
\[ [A,B] = i f_{abm} L_m ~~~\df~~~ [C,[A,B]] = i f_{abm}[L_j,L_m] = if_{abm}if_{jmc} L_c\]
from this follows the relationship
\[ (-if_{aim})(-if_{bmj}) - (-if_{bim})(-if_{amj}) = if_{abc}(-if_{aij})\]
If we now write $if_{abc} = (F^a)_{bc}$ as a matrix element of a matrix $F^a$ we get
\[ (F^a)_{im} (F^b)_{mj} - (F^b)_{im}(F^a)_{mj} = i f_{abc} (F^c)_{ij}\]
due to the summation convention this is equivalent to
\[ (F^aF^b)_{ij} - (F^bF^a)_{ij} = if_{abc}(F^c)_{ij}\]
So this in it self satisfies the commutation relation of the Lie algebra:
\[ [F^a, F^b] = i f^{abc} F^c\]
So these objects for an irreducible representation of the Lie algebra called the adjoint irreducible representation.\\
\\
In summary, for any $SU(N)$ group there are three irreducible representations:
\begin{itemize}
\item the trivial representation: $U(\theta) = 1$
\item the fundamental representation: $U(\theta) = \exp\left( i \frac{\lambda^a}{2} \theta^a\right)$
\item the adjoint representation: $U(\theta) = \exp \left( i F^a \theta ^a\right)$
\end{itemize}


\section{The Lorentz Group and Relativistic Invariance}
\sub{Lorentz transformations}
Consider two inertial frames with common origin at $t = 0$ and moving with relative velocity $v$ along the $x$-axis. Lets assume that in the first frame an event is taking place at $x^\mu = (t, \vx)^T$ and is seen at $x'^\mu = (t', \vx')^T$ in the primed frame. The transformation is as follows
\[ x'^\mu = \vektorz{t'}{\vx'} = \begin{pmatrix} \gamma t - \gamma v t \\  \gamma v t + \gamma x\\ y \\z\end{pmatrix} = \begin{pmatrix} \gamma & - \gamma v & 0 & 0 \\ -\gamma v & \gamma & 0 & 0 \\ 0 & 0 & 1 & 0 \\ 0 & 0 & 0 & 1\end{pmatrix}\cc x^\mu\]
The transformation matrix is called $\Lambda$ so that $x'^\mu = \Lambda^\mu_{~\alpha} x^\alpha \equiv L x$.\\
\\A Lorentz transformation now is any linear transformation $\Lambda$ which keeps the relative length invariant:
\[ s^2 = x'^\mu x'^\nu g_{\mu\nu} = \Lambda^\mu_{~\alpha} \Lambda^\nu_{~\beta}x^\alpha x^\beta g_{\mu\nu}\soll x^\alpha x^\beta g_{\alpha\beta}\]
This must hold for all possible $x^\mu$.\\
From this follows
\[ \Lambda^\mu_{~\alpha} \Lambda^\nu_{~\beta} g_{\mu\nu} = g_{\alpha\beta}\]
If we now regard $x^\mu$ as a column vector $x$ and $\Lambda^\mu_{~\nu}$ as the elements of some $4\times 4$ matrix $L$ then follows
\[x' = L \cc x, ~~~~ s^2 = x^Tg x\]
Then the condition on the $\Lambda$'s reads
\[g^\alpha_{~\beta} = \Lambda_\mu^{~\alpha} g^\mu_{~\nu} \Lambda^\nu_{~\beta}~~\df~~ g_{\alpha\beta} = \Lambda^\mu_{~\alpha} g_{\mu\nu}\Lambda^\nu_{~\beta} = (L^TgL)_{\alpha\beta}\]
Now, because $g$ is symmetric ($g_{\alpha\beta} = g_{\beta\alpha}$) also $L^TgL$ must be symmetric. Therefore there are only ten conditions on $L$ (ten independent elements).\\
If we now take the determinante of that expression we find
\[\det(g) = \det(L^T)\det(g) \det(L)~~~\df~~~ \det(L) = \pm 1\]
and with the Jacobi determinant follows
\[ \int \dd ^4 x' = \int \dd ^4 x| \det(L) | = \int \dd ^4 x\]
We call Lorentz transformations with $\det(L) = 1$ proper Lorentz transformations and Lorentz transformations with $\det(L) = -1$ improper.\\
Also, if we take $\alpha = \beta = 0$ in the equation, we find 
\[g_{00} = 1 = \Lambda^{\mu}_{~0} g_{\mu\nu} \Lambda^\nu_{~0} = (\Lambda^0_{~0})^2 - (\Lambda^i_{~0})^2\]
So necessarily $|\Lambda^0_{~0}| \geq 1$.\\
We then call Lorentz transformations with $\Lambda^0_{~0} \geq 1$ orthochronous and Lorentz transformations with $\Lambda^0_{~0} \leq 1$ non orthochronous.

\sub{Lorentz group}
All Lorentz transformations form a group, the Lorentz group. It has the following attributes
\begin{itemize}
\item the product of two Lorentz transformations is again a Lorentz transformation
\item the inverse exists. In the example of the beginning it would replace $v \df -v$.
\item there is a unitary element ($L = \ehm_4$)
\end{itemize}
Because a product of Lorentz transformations is again a Lorentz transformation we can reduce the types of different transformations to four:
\begin{enumerate}
\item time inversions: $x'^0 = - x^0, x'^i = x^i$ (non orthochronous, improper)
\item parity transformation: $x'^0 = x^0, x'^i = - x^i$ (orthochronous, improper)
\item rotations: $x'^0 = x^0, x'^i = a^{ij} x^j$ with a rotation matrix $a^{ij}$, $L = \Matz{1}{0}{0}{a}$ where $a$ is the $3\times 3$ rotation matrix with $\det(a) = 1$. (orthochronous, proper)
\item boosts (here in one direction): $x'^0 = x^0 \cosh \eta + x^1 \sinh \eta$, $x'^1 = x^0 \sinh\eta x^1 \cosh\eta$, $x'^{(2,3)} = x^{(2,3)}$ where $\eta$ is the rapidity. (orthochronous, proper)
\end{enumerate}
The number of possible rotations is three (for example the Euler angles), the number of boosts also is three (for example the three boost directions or two angles plus $v$). Therefore, any proper, orthochronous Lorentz transformation can be described by six real parameters. The time inversion and parity transformation do not have infinitesimal representations, because they are discrete transformations.

\sub{Inifitesimal generators of proper and orthochronous Lorentz transformations}
For any adequate description of the Lorentz group we need to study infinitesimal generators. Therefore we consider the infinitesimal Lorentz transformation:
\[ \Lambda^\mu_{~\nu} = \delta^\mu_{~\nu} + \eps^\mu_{~\nu} \equiv g^\mu_{~\nu} + \eps^\mu_{~\nu}\]
Now, the condition $g = L^T g L$ can be rewritten as 
\[ \delta^\alpha_{~\beta} = g^\alpha_{~\beta} = \Lambda_\mu^{~\alpha} g^\mu_{~\nu}\Lambda^\nu_{~\beta} = \Lambda_\mu^{~\alpha}\Lambda^\mu_{~\beta} = ( g_\mu^{~\alpha} + \eps_\mu^{~\alpha})(g^\mu_{~\beta} + \eps^\mu_{~\beta}) = g_\beta^{~\alpha} \eps_\beta^{~\alpha} + \eps^\alpha_{~\beta} + \OO(\eps^2)\]
From there follows $\eps_\beta^{~\alpha} + \eps^\alpha_{~\beta} = 0$, for example $\eps_{\alpha\beta} = - \eps_{\beta\alpha}$. So this is antisymmetric with six independant real elements.\\
\\
We now introduce $L_{\mu\nu} = i ( x_\mu \p_\mu - x_\nu\p_\mu)$ with $\p_\mu = \left( \frac{\p}{\p t}, \vn\right)^T$ as a generalization of the angular momentum operator $J^i =i \eps^{ijk} x_j \p_k$. It comes as as surprise, that these $L_{\mu\nu}$ exactly are the generators of the infinitesimal Lorentz transformation:
\[J^i = i \eps^{ijk} x_j \p _k = \frac{1}{2}\eps^{ijk} L_{jk}\]
With $\delta x^\mu = \Lambda^\mu_{~\nu} x^\nu - x^\mu = \eps^\mu_{~\nu} x^\nu$ it follows
\[ \frac{i}{2} e^{\rho\sigma} L_{\rho\sigma} x^\mu = \frac{i}{2} i \eps^{\rho\sigma} ( x_\rho g_\sigma^{~\mu}- x_\sigma g_\rho^{~\mu}) = - \frac{1}{2} (e^{\rho\mu} x_\rho - \eps^{\mu\sigma}x_\sigma )= \eps^{\mu\sigma}x_\sigma = \delta x^\mu\]
where we used $\eps^{\rho\mu} = - \eps^{\mu\rho}$. Therefore
\[ \frac{i}{2}\eps^{\rho\sigma} L_{\rho\sigma} x^\mu = \eps^\mu_{~\nu} x^\nu\]
So the $L_{\rho\sigma}$ are indeed the generators of rotations in Minkovski space, explicitly, the $SO(3,1)$. The Lie algebra is
\[ [L_{\mu\nu}, L_{\rho\sigma}] = i ( g_{\nu\rho} L_{\mu\sigma} - g_{\mu\rho}L_{\nu\sigma} - g_{\nu\sigma}L_{\mu\rho} + g_{\mu\sigma}L_{\nu\rho})\]
\newline
As a generalization, $L_{\mu\nu}$ is analogous to orbital angular momentum. We can add a spin term, which of course commutes with $L$ and forms some algebra similar to the $L$'s among themselves.\\
The most general representation of $SO(3,1)$ generators is by $M_{\mu\nu} \equiv i(x_\mu \p_\nu - x_\nu\p_\mu) + S_{\mu\nu}$. The $M_{\mu\nu}$ space components, or more familiarly, the $J^i = \frac{1}{2} \eps^{ijk} M_{jk}$ components are the generators of inifinetismal rotations with $[J_i, J_j] = i \eps^{ijk} J^k$ (the commutation relation from $SO(2)$).\\
The $M^{0i} \equiv K^i$ are space time components and generate the boosts.\\
The commutation relation of these $K^i$ and the known $J^i$ are
\[ [K^i, K^j] = - \eps^{ijk} J^k, ~~~~ [J^i, K^j] = i \eps^{ijk} K^k\]
These are much simpler than the commutation relation of the $L_{\mu\nu}$.\\
Also, there usually is a trick to separate two $SO(2)$'s from each other by taking the linear combinations
\[N^i \equiv \frac{1}{2}(J^i + i K^i), ~~~~~ N^{i\dagger} \equiv \frac{1}{2} ( J^i - i K^i)\]
For these we find the commutation relations
\[ [N^i, N^{i\dagger}] = 0, ~~~ [N^i, N^j] = i \eps^{ijk} N^k, ~~~ [N^{i\dagger}, N^{j\dagger}] = i \eps^{ijk} N^{k\dagger}\]
Because the first one is zero, $N^\dagger$ and $N$ are decoupled.\\
Finally, we find, that the finite dimensional representations of the restriced Lorentz group is characterized by $(n,m)$ where $n,m = 0,  \frac{1}{2}, 1, \ldots$ and are given by the eigenvalues of the Casimir operators:
\[N^iN^i|n,n_3\rk = n(n+1)|n,n_3\rk, ~~~~ N^{i\dagger}N^{i\dagger}|m,m_3\rk = m(m+1)|m,m_3\rk\]
These $m$ and $n$ are related by parity: $J_i \overset{P}{\df} J_i$, $K_i \overset{P}{\df} - K_i$. This follows, because $J_i$ has two spacial directions and $K_i$ only has one. Now, the parity transformation changes the sign of the spacial components, so for $J_i$ this cancels. The $N$ transform as
\[ N^i \overset{P}{\df} N^{i\dagger}, ~~~ \df ~~~ (n,m) \overset{P}{\df} (m,n)\]
Hence $n$ and $m$ are related. Additionaly we can identify the spin of the representation as $n+m$, this follows, because $J^i = N^i + N^{i\dagger}$ (?).\\
Some special cases are for example:
\begin{itemize}
\item The scalar representation with spin 0 is $(0,0)$.
\item $\left(\frac{1}{2}, 0\right)$ and $\left( 0, \frac{1}{2}\right)$ are representations of spin $\frac{1}{2}$ where the first one represents right handed spinors and the second one left handed spinors. Both are Weyl spinors.
\item Combining $\left(0, \frac{1}{2}\right)\oplus \left(\frac{1}{2}, 0 \right)$ as a direct sum this leads to Dirac spinors. This is for example needed, when a parity invariant theory is wanted.
\end{itemize}
With $J$ and $K$ we can describe finite Lorentz transformations by exponenciating the infinitesimal ones in a non covariant form:
\[ e^{-i(\vv{\omega}\vv{J} + \vv{\nu}\vv{K})}\]
Here $\vv{\omega}$ is the direction of the rotation axis and $|\vv{\omega}|$ is the rotation angle. The direction of the boost is given by $\vv{\nu}$ and $|\vv{\nu}|$ is the rapidity of the boost.\newline
\sub{Poincaré group}
The Poincaré group is obtaioned from the Lorentz group by adding translations $x^\mu \df x'^\mu = x^\mu + a^\mu$ where $a^\mu$ is a constant 4 vector. The most general translation is
\[ x^\mu \df x'^\mu = \Lambda^\mu_{~\nu}x^\nu + a^\mu\]
This has ten real parameters, so we need four additional generators. These will be the momentum operators.\\
Infinitesimally the difference between the translated and untranslated vector can be expressed as $\delta x^\mu = \eps^\mu = - \eps^\rho P_\rho x^\mu$ with $P_\rho = i \p_\rho$.\\
The commutation relations of the momentum operator are
\[ [P_\mu, P_\nu] = 0, ~~~[M_{\mu\nu}, P_\rho] = -i g_{\mu\nu}P_\nu + ig_{\nu\rho}P_\mu\]


\section{Transformation of Fields under Lorentz Transformations}
Lets take a look at a vector field $A_\mu(x)$ (for example the vector potential of the electromagnetic field). An observer in the primed frame describes the same physical situation by another field $A'_\mu(x)$. The two observed fields are of course related, for $x'_\mu = \Lambda_\mu^{~\nu}x_\nu$ it follows
\[A'_\mu = \Lambda_\mu^{~\nu}A_\nu(x)\]
This relation is, what defines $A_\mu$ as a vectorfield. We can generalize this:\\
Let $x_\mu$ and $x'_\mu = \Lambda_\mu^{~\nu} x_\nu$. This should describe some point $P$ in two inertial frames. Let $\Lambda_A^{~B}$ ($A,B = 1, \ldots, d$) be the corresponding $d$-dimensional representation matrixes of an irreducible representation of $SO(3,1)$. A $d$-component field $f_A(x)$ and $f'_A(x')$ in the two transformations is said to transform under this irreducible representation if $f'_A(x') = \Lambda_A^{~B} f_B(x)$. In the infinitesimal version this is $\Lambda_A^{~B} = \delta_A^{~B} - \frac{i}{2} \eps^{\alpha \beta}(S_{\alpha\beta})_A^{~B}$.\\
Here are some examples
\begin{table}[H]
\begin{tabular}{lll}
irr. rep. of $SO(3,1)$ & field & transf. property\\
\midrule
$\left(0, 0\right)$, scalar & $\phi(x)$ & $\phi'(x') = \phi(x)$\\
$\left(\frac{1}{2}, \frac{1}{2}\right)$, vector & $V^\mu(x)$ & $V'^\mu(x') = \Lambda^\mu_{~\nu}V^\nu(x)$\\
$\left(0,\frac{1}{2}\right)$, left handed Weyl spinor & $\psi_L(x)$ & $\psi'_L(x') = \Lambda_L \psi_L(x)$\\
$\left(\frac{1}{2}, 0 \right)$, right handed Weyl spinor & $\psi_R(x)$ & $\psi'_R(x') = \Lambda_R \psi_R(x)$\\
\end{tabular}
\end{table}
The transformation matrices $\Lambda_R$ and $\Lambda_L$ are given by:
\[\Lambda_{L,R} = e^{-i(\vv{\omega}\vv{J} + \vv{\nu}\vv{K})}\]
with $\vv{J} = \vv{N} + \vv{N}^\dagger$ and $\vv{K} = i (\vv{N}^\dagger - \vv{N})$. For $\Lambda_L$ the $N$ are $\vv{N}^\dagger = \frac{\vv{\sigma}}{2}$ and $\vv{N} = 0$. For $\Lambda_R$ they are $\vv{N}^\dagger = 0$ and $\vv{N} = \frac{\vv{\sigma}}{2}$. This follows, because $\vv{N}$ are $SU(2)$ groups and we need a representation of $SU(2)$ which describes spin $\frac{1}{2}$ particles.\\
With this parametrization the $\Lambda$ are given by
\[ \Lambda_L = \exp\left( -i \frac{\vv{\sigma}}{2} \left(\vv{\omega} + i \vv{\nu}\right)\right)\]
\[ \Lambda_R = \exp\left( i \frac{\vv{\sigma}}{2} \left( \vv{\omega} - i \vv{\nu}\right)\right)\]
Some properties of $\Lambda_{R,L}$ are
\begin{itemize}
\item $\Lambda_L^{-1} = \Lambda_R^\dagger$
\item $\sigma^2 \Lambda_L \sigma^2 = \Lambda_R^*$
\item $\sigma^2 \Lambda_L^{-1} \sigma^2 = \Lambda_L^T$
\end{itemize}
Now consider a Lorentz transformation of the field $\sigma^2 \psi_R^*$:
\[ \sigma^2 \psi_R^* ~~~\df~~~ \sigma^2 \Lambda_R^*\psi_R^* = \underbrace{\sigma^2 \Lambda_R^*\sigma^2}_{\Lambda_L} \sigma^2 \psi_R^* = \Lambda_L \sigma^2 \psi_R^*\]
Thus $\sigma^2 \psi_R^*$ is a left handed spinor because it transforms the same way. It is called the charge conjugate of $\psi_R$. Therefore:
\begin{itemize}
\item $\psi_L$ and $\sigma^2\psi_R^*$ transform as $\left(0, \frac{1}{2}\right)$
\item $\psi_R$ and $\sigma^2\psi_L^*$ transform as $\left(\frac{1}{2}, 0\right)$
\end{itemize}
$\psi_L$ and $\psi_R$ are called Weyl spinors.
\sub{Construction of scalars and vectors}
The Weyl spinors can be combined to construct scalars and vectors. First, consider products of two left handed spinors $\chi_L$ and $\psi_L$. Since $\left( 0, \frac{1}{2}\right) \otimes \left( 0, \frac{1}{2}\right) = (0,0) \oplus (0,1)$ we can construct a scalar from them:
\[ \chi_L^T \sigma^2 \psi_L ~~\overset{LT}{\df} \chi_L^T \Lambda_L^T \sigma^2 \Lambda_L \psi_L = \chi_L^T \sigma^2 \psi_L\]
In the second step we used, that $\Lambda_L^T \sigma^2 \Lambda_L$ is equal to $\sigma^2 \Lambda_L^{-1}\sigma^2\sigma^2 \Lambda_L$ due to the properties of $\Lambda_{L}$ and that $(\sigma^2)^2= 1$ and $\Lambda_L^{-1}\Lambda_L = 1$. \\
We therefore know, that $\chi^T_L\sigma^2\psi_L$ is a scalar. With
\[ \chi_L = \vektorz{\chi_1}{\chi_2}, ~~~ \psi_L = \vektorz{\psi_1}{\psi_2}\]
follows
\[ \chi_L^T \sigma^2 \psi_L = \begin{pmatrix} \chi_1 & \chi_2\end{pmatrix} \Matz{0}{-i}{i}{0} \vektorz{\psi_1}{\psi_2} = - i (\chi_1 \psi_2 - \chi_2 \psi_1)\]
This shows, that this scalar is antisymmetrical in the two fields. It is important to realize, that these are spin $\frac{1}{2}$ fields, so they must be represented by anitcommuting Grassmann numbers to conserve the properties demanded by spin $\frac{1}{2}$.\\
Hence even for $\chi_L = \psi_L$ we get
\[ \psi_L^T \sigma^2 \psi_L = - i( \psi_1 \psi_2 - \psi_2 \psi_1) = 2i \psi_2 \psi_1 \neq 0\]
This is non zero because of the anticommuting Grassmann numbers. This would not be the case, if these scalars would commute like normal scalars.\\
This combination is used in Majorana mass terms.\\
We can also take $\chi_L = \sigma^2\psi_R^*$ because it also is a lefthanded spinor. Then we get
\[\chi_L^T \sigma^2 \psi_L = \psi^\dagger_R(-\sigma^2)^2 \psi_L = - \psi_R^\dagger \psi_L\]
Which is a particle of the Dirac mass term.\\
\\
Vectors $\left(\frac{1}{2},\frac{1}{2}\right)$ can be obtained from the product of $\psi_L$ and $\sigma^2\psi_R^*$, $\left(\frac{1}{2}, 0 \right) \otimes \left(0,\frac{1}{2}\right) = \left(\frac{1}{2},\frac{1}{2}\right)$. \\
We start from $\psi_L^\dagger \psi_L$ which is invariant under rotations. Under boosts this transforms as 
\[ \psi_L^\dagger \psi_L ~~\df~~ \psi_L^\dagger \Lambda_L^\dagger \Lambda_L \psi_L = \psi_L^\dagger e^{\vv{\sigma}\vv{\nu}} \psi_L = \psi_L^\dagger \psi_L + \vv{\nu} \psi_L^\dagger \vv{\sigma} \psi_L + \ldots\]
Similarly for $\psi_L^\dagger \sigma^i \psi_L$:
\[ \psi_L^\dagger \sigma^i \psi_L ~~\df~~ \psi_L^\dagger e^{\vv{\sigma}\vv{\nu}/2} \sigma^i e^{\vv{\sigma}\vv{\nu}/2} \psi_L = \psi_L^\dagger \sigma^i \psi_L - \nu^i \psi_L^\dagger \psi_L\]
In the second step we used $e^{\vv{\sigma}\vv{\nu}/2} \sigma^i e^{\vv{\sigma}\vv{\nu}/2} = \sigma^i - \frac{1}{2} \nu^j \{\sigma^j,\sigma^i\} = \sigma^i - \nu^i \delta^{ji}$.\\
Summarizing the difference between transformed and untransformed vectors $\delta$ is given by
\[ \delta(\psi_L^\dagger \psi_L) = - \nu^i \psi_L^\dagger (-\sigma^i) \psi_L\]
and
\[ \delta( \psi_L(-i\sigma^i) \psi_L) = -\nu^i \psi^\dagger_L \psi_L\]
If we compare this with the general transformation of 4 vectors under boosts which is given by
\[ \frac{1}{2} \eps^{\mu\nu} M_{\mu\nu} = \eps^{0i} M_{0i} = - \eps^{0i}K^i = \vv{\nu}\vv{K} \]
where of course only boosts were considered. The difference here is given by
\[ \delta V^\mu = - \eps^\mu_{~\nu} V^\nu\]
For the time component this leads to
\[ \delta V^0 = \eps^{0i} V^i = - \nu^i V^i\]
and for the spacial components
\[ \delta V^i = - \eps^{i0} V^0 = \eps^{0i} V^0 =  -\nu^i V^0\]
Therefore $( \psi_L^\dagger \psi_L, - \psi_L^\dagger \vv{\sigma} \psi_L)$ transforms as a 4 vector. This is often times written as $\psi_L ^\dagger \sigma_-^\mu \psi_L$ with $\sigma_-^\mu \equiv (\ehm, - \vv{\sigma})$.\\
Similarly for the right handed spinors $\psi_R^\dagger \psi_R$ with $\Lambda_R = e^{\vv{\sigma}\vv{\nu}/2}$ we get the vector
\[ \psi_R^\dagger \sigma_+^\mu \psi_R = ( \psi_R^\dagger \psi_R, \psi_R^\dagger \vv{\sigma} \psi_R), ~~~ \sigma_+^\mu \equiv (\ehm, \vv{\sigma})\]
Lorentz scalars now can be formed by contracting with Lorentz vectors in particular with $\p_\mu$: $( \p_\mu \psi_L^\dagger) \sigma_-^\mu \psi_L,~~ \psi_L^\dagger \sigma_-^\mu ( \p_\mu \psi_L) $, etc are thus candidates for kinetic energy terms in Lagrangians. Becaue no point in spacetime is special $x^\mu$ does not appear in the physical theories, only terms with $\p^\mu$. Now, because $\p^\mu$ transforms as $\left(\frac{1}{2}, \frac{1}{2}\right)$ and thus as a vector field, $\psi^\dagger_L \sigma_-^\mu ( \p_\mu\psi_L)$ and $\psi_R^\dagger \sigma_+^\mu \p_\mu \psi_R$ transform as $(0,0)$ and are thus scalars. These expressions are just contractions of two four vectors.\\
\\
If we take a look at $\psi_R^\dagger\psi_L$ and $\psi^\dagger_L\psi_R$ it is obvious, that we need a $\psi_L$ if we have a $\psi_R^\dagger$ to form a scalar. This is the only way to get a scalar. Equally we can only get a vector if $\psi_R^\dagger$ is multiplied with another $\psi_R$, or the same with $\psi_L$.\\
That means, that in $\psi^\dagger_R \sigma_+^\mu \p_\mu \psi_R$ the $\sigma_+^\mu\p_\mu \psi_R$ must transform as a left handed Weyl spinor with $\left( 0,\frac{1}{2}\right)$. The same holds for $\sigma_-^\mu\p_\mu\psi_L$ with must transform as a right handed Weyl spinor with $\left( \frac{1}{2},0\right)$.\\
These terms are just spinors with a derivative multiplied with dirac matrices. So they are candidates for terms in equations of motion.


\sub{Field Equations for Weyl Spinors}
If we take a look at the newly found terms and try to construct an equation of motion for them, we can use $i \sigma_-^\mu\p_\mu \psi_L$ as a starting point. This is a right handed Weyl spinorfield, and thus it must be also equal to another right handed spinorfield.\\
There are three possibilities, to which every other solution can be reduced:
\begin{itemize}
\item $i \sigma_-^\mu \p_\mu \psi_L = 0$. This is the correct equation for massless, left handed particles, for example the massless neutrinos.
\item $i \sigma_-^\mu \p_\mu \psi_L = m \sigma^2 \psi_L^*$. This is the equation for Majorana particels. It would describe Majorana neutrionos. The $m$ was introduced, because the left side has units of energy, coming from $\p^\mu$, thus also the right side has to have units of energy, which led to an introduction of the mass $m$.
\item $i \sigma_-^\mu \p_\mu \psi_L = m \psi_R$. This is the Dirac equation. Now, because there is a $\psi_R$ in the equation for $\psi_L$ we also need an equation of motion for $\psi_R$, which must have at least one derivative of $\psi_R$ in it. Therefore we introduce another Dirac equation $i \sigma_+^\mu \p_\mu \psi_R = m \psi_L$.
\end{itemize}
The last two equations are equivalent with the known Dirac equation. This follows easily if we write the two equations as a combination for the Dirac spinor $\psi = \vektorz{\psi_L}{\psi_R}$:
\[ m \psi = m \vektorz{\psi_L}{\psi_R} = i \Matz{0}{\sigma_+^\mu}{\sigma_-^\mu}{0} \p_\mu\vektorz{\psi_R}{\psi_L}\]
The resulting matrix is called $\gamma^\mu$ and should have the known anticommutation relation for gamma matrices: $\{\gamma^\mu, \gamma^\nu\} = 2g^{\mu\nu}$:
\[\{\gamma^\mu, \gamma^\nu\} = \Matz{\sigma_+^\mu\sigma_-^\nu + \sigma_+^\nu\sigma_-^\mu}{0}{0}{\sigma_-^\mu\sigma_-^\nu+ \sigma_-^\nu\sigma_+^\mu}\]
If both indices are zero, this reduces to $2\ehm_4 = 2 g^{00}$ which satisfies the anticommutation relation. If only one index is zero and the other is $i = 1,2,3$ the matrix is equal to zero, because $\sigma_-^i + \sigma_+^i = \sigma^i - \sigma^i = 0 = 2 g^{0i}$ due to the signs in the definition of the $\sigma_\pm$ matrices. \\
If both indices are $i,j = 1,2,3$ the matrix can be written as 
\[ \{\gamma^i,\gamma^j\} = \Matz{-\{\sigma^i,\sigma^j\}}{0}{0}{-\{\sigma^i,\sigma^j\}}\]
The anticommutator of the Pauli matrices is known to be $\{\sigma^i, \sigma^j\} = 2\delta_{ij}$, thus the anticommutation relation is $\{\gamma^i,\gamma^j\} = - 2\delta_{ij} = 2 g^{ij}$ due to Minkovski metrics. All in all, the wanted anticommutation relation is fullfilled and therefore this is in fact the Dirac equation in another form.

\section{Parity}
Lets take a look at the parity transformation $x^\mu = (t, \vx) ~\overset{P}{\df}~ (t, - \vx) = \tilde{x}^\mu$.\\
In the irreducible representation of $SO(3,1)$ this transformation is $(j_1, j_2)~\overset{P}{\df}~ (j_2, j_1)$. This follows from the first component being $\vv{N} = \vv{J} + i \vv{K}$ and the second one being $\vv{N}^\dagger = \vv{J} - i \vv{K}$. Under parity transformation $\vv{J}$ stays the same and $i\vv{K}$ gets a minus sign, thus $\vv{N}$ transforms to $\vv{N}^\dagger$ and vice versa.\\
This means, that a left handed field becomes a right handed one and vice versa:
\[\psi_L(x) ~\overset{P}{\df}~ \psi_R(\tilde{x})\]
To construct a parity invariant theory, the right handed and left handed fields have to have the same masses and signs, etc. to remain conserved under parity transformation.\\
For the Dirac spinor the parity transformations yields
\[ \psi(x) = \vektorz{\psi_L}{\psi_R} ~\overset{P}{\df}~ \vektorz{\psi_R(\tilde{x})}{\psi_L(\tilde{x})} = \Matz{0}{\ehm_2}{\ehm_2}{0} \vektorz{\psi_L}{\psi_R} = \gamma^0 \psi(\tilde{x})\]
A projection onto $\psi_L$ or $\psi_R$ can be done with the projection operators:
\[ P_L \equiv P_- = \frac{1}{2} ( 1 - \gamma^5) = \Matz{\ehm_2}{0}{0}{0}\]
\[ P_R \equiv P_+ = \frac{1}{2} ( 1 + \gamma^5) = \Matz{0}{0}{0}{\ehm_2}\]
These operators project $\psi_R$ or $\psi_L$ out of $\psi$. Here the $\gamma^5$ matrix is $\gamma^5 = i \gamma^0 \gamma^1\gamma^2\gamma^3 = \Matz{-\ehm}{0}{0}{\ehm}$. The projection can be seen by applying $P_\pm$:
\[P_L\psi = \vektorz{\psi_L}{0} \equiv \psi_L, ~~~ P_R \psi = \vektorz{0}{\psi_R} \equiv \psi_R\]
This four component vector is sometimes written as $\psi_L$ or $\psi_R$ respectively, which can be easily confused with the two component Weyl spinor $\psi_L$ or $\psi_R$ which it consists of. Only from context it is visible, which of the two is meant.\\
This treatment of parity transformation is needed, because it will show, that quantum electro dynamics (QED) - the electromagnetic interaction, and quantum chromo dynamics (QCD) - the strong interaction, are in fact parity invariant, but the weak interaction is not. In fact it violates the parity conservation maximally, only left handed particles interact with the $W^\pm$ bosons.

\section{Plane Wave Solutions of the Dirac Equation}

Lets take a look at $\psi(x) = e^{-ikx}\psi(k)$ and find its plane wave solutions with the Dirac equation $i \p^\mu \gamma_\mu \psi = m \psi$. In $k$ space the Dirac equation yields $i (-i k^\mu) \gamma_\mu \psi = \slashed{k} \psi$ thus the equation to be solved is $(\slashed{k} - m) \psi(k) = 0$. Multiplying this equation with another $\slashed{k}$ leads to
\[ (k^2 - m \slashed{k}) \psi(k) = (k^2 - m ^2) \psi(k)= 0 \]
because $\slashed{k}^2 = k^2$ and $m\slashed{k}\psi(k) = m^2 \psi(k)$. (?)\\
Thus, $k^2 = m^2$ or $k^0 = \pm \sqrt{ \vv{k}^2 + m ^2}$ are solutions for $k$. The $\pm$ solutions are positive and negative energy solutions. For positive energy this is
\[\psi^+(x) = e^{-ipx}u(p)\]
and for negative energies
\[ \psi^-(x) = e^{+ipx}v(p)\]
Where $p^0 = + \sqrt{\vv{p} + m^2} = \pm k^0$ was used to express the exponents the same way.\\
All in all we have four solutions here, two $v$ solutions which correspond to anti partices and two $u$ solutions which correspond to partices, both with spin $\frac{1}{2}$. For both, $\left( 0, \frac{1}{2}\right)$ and $\left(\frac{1}{2}, 0\right)$ holds $\vv{J} \equiv \vv{S} = \frac{\vv{\sigma}}{2}$.\\
Thus for the Dirac spinor, the spin operator is given by
\[ \vv{S} = \Matz{\frac{\vv{\sigma}}{2}}{0}{0}{\frac{\vv{\sigma}}{2}}\]
The problem here arises from $(\slashed{k} - m)\psi(k) = 0$. If we look at $\slashed{p}$ we get
\[ \slashed{p} = \Matz{0}{E - \vv{\sigma}\vv{p}}{E + \vv{\sigma}\vv{p}}{0}\]
where $\vv{\sigma}\vp$ is the projection of the Pauli matrices onto the direction of the momentum, so $[\slashed{p}, S_i]\neq 0$ for directions not in the directions of the spin. So no simultaneous eigenvalues can be found for both.\\
The exceptions are twofold, the first is in the rest frame of the particle with $\vp = 0$. There we can choose a spin direction and boost to an arbitrary frame. This does not work for massless particles in ultrarelativistics.\\
The second, more general approach is, to choose the direction of $\vp$ as the quantization axis, then the helicity eigenvalues are the simultaneous eigenvalues for momentum and spin of the particle.\\
The components of $\vv{S}$ in the direction of $\vp$ are called the helicity:
\[\frac{\vv{S}\vp}{|\vp|} u(\vp, \lambda) = \frac{1}{2|\vp|}\Matz{\vv{\sigma}\vp}{0}{0}{\vv{\sigma}\vp}u(p,\lambda) = \frac{\lambda}{2}u(p,\lambda)\]
This matrix now commutes with $\slashed{p}$, and in any specific direction the eigenvalues are $\pm \frac{1}{2} = \lambda$. This means, that $\lambda = + \frac{1}{2}$ corresponds with positive heilcity and this spin in the direction of the momentum $\vp$.\\
The same holds for anti particles
\[ \frac{ \vv{S}\vp }{ |\vp| } v(p,\lambda) = - \frac{\lambda}{2}v(\vp, \lambda)\]
The minus sign is due to the interpretation of the anti particles as holes in the Dirac sea.\\
Expicitly, $u$ and $v$ have the following forms
\[ u(p, \lambda = \pm 1) = \vektorz{ \sqrt{ E\mp |\vp|} \chi_\pm(\vp)}{\sqrt{E\pm|\vp|} \chi_\pm(\vp)}\]
\[ v(p, \lambda = \pm 1) = \pm\vektorz{-\sqrt{E\pm|\vp|} \chi_\mp(\vp)}{\sqrt{E\mp |\vp|}\chi_\mp(\vp)}\]
These normalize to
\[ \bar{u}(p, \lambda) u(p, \lambda') = 2m \delta_{\lambda\lambda'}, ~~\bar{v}(p, \lambda)v(p, \lambda') = - 2m \delta_{\lambda\lambda'}, ~~ \bar{u}(p, \lambda)v(p,\lambda') = 0\]
and form the completeness relations
\[ \sum_\lambda u(p,\lambda) \bar{u}(p,\lambda) = \slashed{p} + m\]
\[ \sum_\lambda v(p, \lambda) \bar{v}(p,\lambda) = \slashed{p} - m\]
The important thing to notice is, that if we have a highly relativistic particle, the energy becomes $E = \sqrt{\vp^2 + m^2} \approx \vp$, so whenever there is a minus sign in the square roots of the expressions for $u$ and $v$, these are zero. So the two component spinors are reduced:
\[ u(p, \lambda = +1) ~~\df~~ \vektorz{0}{\sqrt{2E} \chi_+} = \vektorz{0}{\psi_R}\]
which has positive helicity and positive chirality,
\[ u(p, \lambda = -1) ~~\df~~ \vektorz{\sqrt{2E} \chi_-}{0} = \vektorz{\psi_L}{0}\]
which has negative helicity and negative chirality. And for the anti particles
\[v(p, \lambda = +1) ~~\df~~\vektorz{-\sqrt{2E}\chi_-}{0} = \vektorz{\psi_L}{0}\]
which has positive helicity but negative chirality and
\[v(p, \lambda = -1) ~~\df~~\vektorz{0}{- \sqrt{2E} \chi_+} = \vektorz{0}{\psi_R}\]
which has negative helicity but positive chirality.\\
Thus a particle with positive helicity in the ultra relativistic limit always consists of only the right handed component, whereas an anti particle with positive helicity is always reduced to a left handed component.

\section{Charge Conjugation}
Lets take a look at the Dirac equation $(i \slashed{\p} - q \slashed{A} -m ) \psi = 0$. This describes particles and anti particles at the same time. However, what is defined to be a particle and what an anti particles is random. The equation also describes particles with charge $-q$ instead of $q$ in the same electro magnetic field. The equation of a charge conjugated particle thus yields
\[ ( i \slashed{\p} + q \slashed{A} - m) \psi^C = 0\]
where $\psi^C(x) = C \bar{\psi}^T = - i \gamma^0 \gamma^2 \bar{\psi}^T$.\\
$\psi^C$ therefore is the chage conjugate field which makes anti particles a basic degree of freedom.\\
Explicitly we can take a look at a Dirac field
\[ \psi = \vektorz{\psi_L}{\psi_R} = P_L \psi + P_R\psi = \vektorz{\psi_L}{0} + \vektorz{0}{\psi_R} \equiv \psi_L + \psi_R\]
the charge conjugate of this field is
\[ \psi^C = i \gamma^2 \gamma^0 ( \psi^\dagger \gamma^0 )^T = i \gamma^2 (\gamma^0)^2 \psi^* = i \gamma^2 \psi^* = i \Matz{0}{\sigma_2}{-\sigma_2}{0} \vektorz{\psi_L^*}{\psi_R^*} = \vektorz{i \sigma_2 \psi_R^*}{-i\sigma_2\psi_L}\]
As we showed earlier, $i \sigma_2 \psi_R^*$ transforms as a left handed field and $i \sigma_2 \psi_L^*$ transforms as a right handed field.\\
One special case of these Dirac fields are the Majorana fermions. For them $\psi = \psi^C$ holds. Therfore, the charge conjugated field is
\[ \psi^C = \vektorz{i \sigma_2 \psi_R}{-i \sigma_2 \psi_L} = \vektorz{\psi_L}{\psi_R}\]
which follows because $\psi_L = i \sigma_2 \psi_R^* ~\df~ -i \sigma_2 \psi_L^* = -i \sigma_2( -i(-\sigma_2)) \psi_R = \psi_R$. Thus we can write a Majorana field as
\[ \psi_M = \vektorz{\psi_L}{-i \sigma_2 \psi_L^*}\]
This defines a fermion spinor which is automatically its own charge conjugate. Furthermore it can be shown, that every particles for which this relation holds, must have no charge.\\
\\
If we now take a look at the charge conjugation of $\psi_L$ we get
\[ (\psi_L)^C = \vektorz{\psi_L}{0}^C = \vektorz{ 0}{-i \sigma_2 \psi_L^*} = (\psi^C)_R\]
which is a right handed field. On the other hand, if we take the charge conjugate of the Dirac field and take the left handed component of it we get
\[ (\psi^C)_L = \vektorz{ i \sigma_2 \psi_R^*}{-i \sigma_2 \psi_L^*} = \frac{1}{2} ( 1 - \gamma^5) \vektorz{ i \sigma_2 \psi_R^* }{-i \sigma_2 \psi_L^*} = \vektorz{ i \sigma_2 \psi_R^*}{0} \]
which is a left handed field. Therfore the charge conjugation of a left handed field and the left handed component of a charge conjugated Dirac field are not equal.


 
