\documentclass{include/thesisclass}
	
%% -------------------------
%% |    Thesis Settings    |
%% -------------------------
\usepackage{bbm}
\usepackage{float}
\SelectLanguage{english}
\usepackage{slashed}
% details on this thesis
\newcommand{\thesisauthor}{Jan van der Linden}
\newcommand{\thesistopic}{Lecture notes\\Theoretische Teilchenphysik I}
\newcommand{\thesisreviewerone}{Prof. Dr. D. Zeppenfeld}
\newcommand{\thesistimestart}{SS 17} 
\newcommand{\thesispagehead}{theoretical particle physics} 

\usepackage{esvect}
\hyphenation
{
    über-nom-me-nen an-ge-ge-be-nen
    %Pro-to-koll-in-stan-zen
    %Ma-na-ge-ment  Netz-werk-ele-men-ten
    %Netz-werk Netz-werk-re-ser-vie-rung
    %Netz-werk-adap-ter Fein-ju-stier-ung
    %Da-ten-strom-spe-zi-fi-ka-tion Pa-ket-rumpf
    %Kon-troll-in-stanz
}
\newcommand{\LL}{\mathcal{L}}
\newcommand{\SSS}{\mathcal{S}}
\newcommand{\DD}{\mathcal{D}}
\newcommand{\Dd}{{\rm D}}
\newcommand{\val}{\vv{\alpha}}
\newcommand{\cc}{\cdot}
\newcommand{\rk}{\rangle}
\newcommand{\lk}{\langle}
\newcommand{\vx}{\vv{x}}
\newcommand{\vp}{\vv{p}}
\newcommand{\vr}{\vv{r}}
\newcommand{\xd}{\hat{x}}
\newcommand{\ad}{\hat{a}}
\newcommand{\pd}{\hat{p}}
\newcommand{\df}{\rightarrow}
\newcommand{\la}{\lambda}
\newcommand{\dd}{{\rm d}}
\newcommand{\hilb}{\mathcal{H}}
\newcommand{\ehm}{\mathbbm{1}}
\newcommand{\vB}{\vv{E}}
\newcommand{\vE}{\vv{B}}
\newcommand{\trans}{\mathcal{T}}
\newcommand{\p}{\partial}
\newcommand{\vK}{\vv{K}}
\newcommand{\OO}{\mathcal{O}}
\newcommand{\soll}{\overset{!}{=}}
\newcommand{\D}{\Delta}
\newcommand{\vV}{\vv{V}}
\newcommand{\vJ}{\vv{J}}
\newcommand{\eps}{\epsilon}
\newcommand{\vn}{\vv{\nabla}}
\newcommand{\vw}{\vv{\omega}}
\newcommand{\vektor}[3]{\begin{pmatrix} #1 \\ #2 \\ #3 \end{pmatrix}}
\newcommand{\vektorz}[2]{\begin{pmatrix} #1 \\ #2 \end{pmatrix}}
\newcommand{\Mat}[9]{\begin{pmatrix}#1&#2&#3\\#4&#5&#6\\#7&#8&#9\end{pmatrix}}
\newcommand{\Matz}[4]{\begin{pmatrix}#1&#2\\#3&#4\end{pmatrix}}
\newcommand{\Ld}[1]{\Lambda_{~#1}}
\newcommand{\Lu}[1]{\Lambda^{~#1}}
\newcommand{\dslash}{\slashed{\partial}}
\newcommand{\sub}[1]{~\newline\newline\textbf{#1}\\}
\newcommand{\tr}{{\rm tr}}

%% -----------------------
%% |    Main Document    |
%% -----------------------
\begin{document}
    % Titlepage and ToC
    \FrontMatter

    % coordinates for background border
\newcommand{\diameter}{20}
\newcommand{\xone}{-15}
\newcommand{\xtwo}{160}
\newcommand{\yone}{15}
\newcommand{\ytwo}{-253}




\begin{titlepage}
    % background border
    \begin{tikzpicture}[overlay]
    \draw[color=gray]
            (\xone mm, \yone mm)
      -- (\xtwo mm, \yone mm)
    arc (90:0:\diameter pt)
      -- (\xtwo mm + \diameter pt , \ytwo mm)
        -- (\xone mm + \diameter pt , \ytwo mm)
    arc (270:180:\diameter pt)
        -- (\xone mm, \yone mm);
    \end{tikzpicture}



    % KIT image and sign for faculty of physics
    \begin{textblock}{10}[0,0](4.5,2.5)
        \includegraphics[width=.25\textwidth]{include/kitlogo.pdf}
    \end{textblock}
    \changefont{phv}{m}{n}    % helvetica




    % horizontal line
    \begin{textblock}{10}[0,0](4.2,3.1)
        \begin{tikzpicture}[overlay]
        \draw[color=gray]
                (\xone mm + 5 mm, -12 mm)
          -- (\xtwo mm + \diameter pt - 5 mm, -12 mm);
        \end{tikzpicture}
    \end{textblock}



    % begin of text part
    \changefont{phv}{m}{n}    % helvetica
    \centering



    % thesis topic (en and ge)
    \vspace*{3cm}
    \Huge\thesistopic\\




    % author name and institute
    \vspace*{2cm}
    \Large von\\
    \vspace*{1cm}
    \huge\thesisauthor\\
    \vspace*{1cm}
    \Large Vorlesung gehalten von



    % examiners (Referenten)
    \vspace*{1.5cm}
    \Large
    \thesisreviewerone\\
    



    % working time
    \vspace{1cm}
    \begin{center}
        \large{\thesistimestart}
    \end{center}



    % lowest text blocks concerning the KIT
    \begin{textblock}{10}[0,0](4,16.8)
        \tiny{KIT -- Universität des Landes Baden-Württemberg und nationales %
              Forschungszentrum in der Helmholtz-Gemeinschaft}
    \end{textblock}
    \begin{textblock}{10}[0,0](14,16.75)
        \large{\textbf{www.kit.edu}}
    \end{textblock}
\end{titlepage}

    
    \begingroup \let\clearpage\relax    % in order to avoid listoffigures and
    \tableofcontents                    % listoftables on new pages
    \endgroup



    % Contents
    \MainMatter
\chapter{Introduction}
\section{Quarks and Leptons}
Particles of matter:
\begin{itemize}
	\item electrons ($e^-$) and other leptons are elementary particles. 
	\item protons and neutrons ($|p\rk = |uud\rk$, $|n\rk = |udd\rk$) are combinations of elementary quarks and gluons. The binding energy of the quarks is very large in comparison to the absolute energy of the proton and neutron ($m_pc^2 = 938~\si{MeV}$) if you compare this to the binding energies of Atoms ($\sim 1~\si{Ry}$) and their absolute energies ($\sim 10^9~\si{Ry}$).\\
		Because the proton and the neutron are similar/symmetric in the strong interaction (not in the electroweak interaction though) we can combine them into a isospin dublett $\vektorz{p}{n}$.
\end{itemize}
There are many more particles/boundstates of quarks and gluons for different combination of quarks. Another example are the $\Delta$ baryons. These are spin $\frac{3}{2}$ particles and have masses of $m_\Delta c^2 \approx 1230 ~\si{MeV}$:
\begin{itemize}
	\item $\Delta^-:~|ddd\rk$
	\item $\Delta^0:~|ddu\rk$
	\item $\Delta^+:~|duu\rk$
	\item $\Delta^{++}:~|uuu\rk$
\end{itemize}
Because the $\Delta$ baryons are spin $\frac{3}{2}$ particles all of the quarks spins must be aligned, so the spin wavefunction is symmetric. Also the orbital wavefunction is symmetric for the $\Delta^{++}$-baryon because it consists of thrice the same quark. However, the total wavefunction of the baryons must be antisymmetric because it is a fermion.\\
This is the reason a color charge was introduced to interpret the Dirac statistics correctly and characterize the strong interaction with a new quantum number.\\
Alltogether one can describe the four $\Delta$ baryons in an isospin quartet with $I = \frac{3}{2}$.\\
\\
Another group of particles are the mesons, they consist of one quark and an anti quark. The lightest examples are the pions:
\begin{itemize}
\item $\pi^+:~|u\bar{d}\rk$
\item $\pi^0:~\frac{1}{\sqrt{2}}\left( |u\bar{u}\rk - |d \bar{d}\rk\right)$
\item $\pi^-:~|d\bar{u}\rk$
\end{itemize}
They have masses of $m_\pi c^2 \approx 140~\si{MeV}$ and are spin $0$ particles. Together they form the isospin triplet $I = 1$.\\
\\
Another group of mesons with spin $0$ are the kaons. These have another type of quark, the strange quark. For this new quark a new quantum number (next to isospin) was introduced, the strangeness.\\
We can summarize the kaons and the pions in a meson octett depicted in figure \ref{1}. The kaons have masses of $m_Kc^2 \approx 495~\si{MeV}$. Additionally to the four kaons and the three pions there is a $\eta$-meson with the same strangeness and isospin as the $\pi^0$. It is like the $\pi^0$ but has additional strange quarks: $|\eta\rk = \frac{1}{\sqrt{6}}\left( |u\bar{u}\rk + |d\bar{d}\rk - 2 | s \bar{s}\rk\right)$.\\
\begin{figure}[H]
\centering
\includegraphics[scale=0.1]{include/mesonoctett.pdf}
\caption{meson octett for spin-$0$}
\label{1}
\end{figure}

The quarks and leptons are probably the fundamental layer of particles; mesons and baryons are complex bound states described through nuclear physics. The quarks and leptons are described by the dirac equation
\[ \left( i \dslash - \frac{mc}{\hbar} \right)\psi = 0 + \si{interactions}\]
One interaction is for example the electromagnetism: $\p_\mu \df \p_\mu + iq A_\mu$

In table \ref{quarks} and \ref{leptons} all quarks and leptons are summarized with their electric charge and mass. The charge is given in units of elementary charge as $q = Q\cc e$.\\
\begin{minipage}{80mm}
\begin{table}[H]
\centering
\begin{tabular}{r|lr}
quark & $mc^2$ [MeV] & Q \\
\midrule
u & $2.2^{+0.6}_{-0.7}$ & $+\frac{2}{3}$\\
d & $4.7$ & $-\frac{1}{3}$\\
\midrule
c & 1270 & $+\frac{2}{3}$\\
s & 96 & $-\frac{1}{3}$\\
\midrule
t & 173200 & $+\frac{2}{3}$\\
b & 4180 & $-\frac{1}{3}$\\
\bottomrule
\end{tabular}
\caption{quarks}
\label{quarks}
\end{table}
\end{minipage}
\begin{minipage}{80mm}
\begin{table}[H]
\centering
\begin{tabular}{r|lr}
lepton & $mc^2$ [MeV] & Q\\
\midrule
$e^-$ & 0.511 & -1\\
$\mu^-$ & 105.66 & -1 \\
$\tau^-$ & 1777 & -1 \\
\midrule
$\nu_e$ & & 0\\
$\nu_\mu$ & & 0\\
$\nu_\tau$ & &0\\
\bottomrule
\end{tabular}
\caption{leptons}
\label{leptons}
\end{table}
\end{minipage}

It is important, that the down quark is slightly heavier than the up quark; because then the down quark more likely decays into the up quark than vice versa and therefore the proton is much more stable than the neutron. This way also atoms remain to be stable and charged.\\
\\
These leptons and quarks are all known matter fields save the bosons:
\begin{itemize}
\item Higgs boson $H$
\item $\gamma$, $W^\pm$, $Z$ which are carriers of the electromagnetic and weak force
\item gluon $g$ which is the carrier of the strong force
\end{itemize}
Additionaly it is known, from observing the Higgs coupling, that there are no more generations of quarks which behave similarly to the three existing generations. Additional generations might exist but must behave fundamentally different.

\section{Course Contents}
In this course of theoretical particle physics the following topics will be discussed:
\begin{itemize}
\item theoretical description of interactions of quarks and leptons\\
$\df$ gauge theories (\textit{Eichtheorien})
\item pair production of particles and $\gamma$, $W^\pm$, $Z$, $g$ emission\\
$\df$ changing particle number and content\\
$\df$ quantum field theory (QFT) which is relativistic for particle physics
\item development of pertubation theory for QFT
\item calculation of cross sections and decay rates
\item symmetries: Lorentz invariance and internal symmetries like isospin and color
\end{itemize} 

\section{Natural Units}
In particle physics it is not practical to use the usual units. It is much more practicable to factor out constants like $\eps_0$, $\hbar$, $c$ and $k_B$ such that the remaining quantities have dimensions of energy to a power.\\
The unit of energy will be electron volts (eV). In table \ref{units} some important quantities and their dimensions in natural units are shown.
\begin{table}[H]
\centering
\begin{tabular}{l|lll}
quantity & SI units & natural units & dimension\\
\midrule
velocity & $\tilde{v}$ & $v \cc c$ & $[v] = 1$\\
length & $\tilde{L}$ & $L \cc \hbar c$ & $[L] = 1/\si{MeV}$\\
time & $\tilde{t}$ & $t\cc \hbar$ & $[t] = 1/\si{MeV}$\\
electric field & $\tilde{E}$ & $\frac{1}{\sqrt{\eps_0 (\hbar c)^3}} \vv{E}$ & $[\vv{E}] = \si{MeV}^2$\\
magnetic field & $\tilde{B}$ & $\frac{1}{\sqrt{\eps_0 c^2(\hbar c)^3}} \vv{B}$ &  $[\vv{B}] = \si{MeV}^2$\\
\bottomrule
\end{tabular}
\caption{natural units}
\label{units}
\end{table}
An example of the simplification is the Hamiltionan for radiation:
\[ H_{rad} = \frac{\eps_0}{2} \int\dd^3 \tilde{\vx} \left[ \tilde{\vE}^2 + c^2 \tilde{\vB}^2\right] \df \frac{1}{2} \int\dd^3 \vx \left[\vE^2 + \vB^2\right]\]

Another useful thing are translations from the normal system to the natural units and vice versa. For example:
\begin{itemize}
\item $\hbar c = 197~\si{MeV fm}$
\item $\frac{1}{\si{GeV}^2} = \frac{3.89\cc 10^{-4}~\si{b}}{(\hbar c)^2}$ where a barn is $10^{-28}~\si{m^2}$
\item $\tilde{e} = 1.6\cc 10^{-19} ~\si{C} ~~\df~~ e = \frac{\tilde{e}}{\sqrt{\eps_0 \hbar c}} = \sqrt{ 4 \pi \alpha} = 0.3028$
\end{itemize}

\subsection{Klein Gordon and Dirac Equations in Natural Units}
The Klein-Gordon equation in SI units is given as
\[
\left[ \tilde{\square} + \left( \frac{mc}{\hbar}\right)^2 \right] \phi(\tilde{x}) = 0
\]
where $x$ is a four vector $\tilde{x}^\mu = ( x \tilde{t}, \tilde{\vx}) = \hbar c (t , \vx)$. Also the d'Alembert operator in SI units is given as
\[ 
\tilde{\square} = \frac{1}{c^2} \frac{\p^2}{\p \tilde{t}^2} - \tilde{\vn}^2 = \frac{1}{(\hbar c)^2} \square = \frac{1}{(\hbar c)^2} \left( \frac{\p^2}{\p t^2} - \vn^2 \right)
\]
So in natural units the equation simplifies to 
\[ ( \square + m^2 ) \phi(x) = 0\]
\newline
Similarly the Dirac equation simplifies when using natural units
\[
\left( i \gamma^\mu \tilde{\p}_\mu - \frac{m c}{\hbar}\right) \psi(\tilde{x}) = 0 ~~\df~~ (i \gamma^\mu \p_\mu - m ) \psi(x) = 0
\]

\section{Lagrange Density and Equations of Motion}

\subsection{Lagrangian Field Theory}
First we take a look at a classical point particle. Its trajectory is given by $x_i(t)$ for $i = 1,2,3$.\\
For this particle we can define an action
\[ \SSS( [x_i], t_1, t_2 ) = \int_{t_1}^{t_2} \dd t \left(\frac{1}{2} m \left( \sum_i \frac{\dd x_i}{\dd t} \right)^2 - V(x_i(t)) \right)
\]
The action is a functional of the trajectory. Now we can find an extremum of $\SSS$ for the classical path by adding a inifinitesimal variation $\delta x_i$ to the trajectory: $x_i(t) + \delta x_i(t)$. Then, the extremal condition is given by $\Delta S = 0$ where $\Delta S$ is given by
\[ \Delta \SSS = \SSS( [x_i + \delta x_i]) - \SSS([x_i]) = 0\]
where the boundary condition is set, so the variation $\delta x_i$ vanishes at the endpoints
\[ \delta x_i(t_1) = \delta x_i(t_2) = 0\]
Calculating the action for the changed trajectory leads to
\[ \SSS( [x_i + \delta x_i]) = \int_{t_1}^{t_2} \dd t \left( \frac{1}{2} m \left( \frac{\dd x_i}{\dd t} + \frac{\dd (\delta x_i)}{\dd t} \right)^2 - V(x_i + \delta x_i) \right)
\]
with
\[  \left( \frac{\dd x_i}{\dd t} + \frac{ \dd ( \delta x_i)}{\dd t} \right) = \left( \frac{ \dd x_i}{\dd t} \right)^2 + 2 \frac{ \dd x_i }{\dd t} \frac{ \dd (\delta x_i)}{\dd t} = \left( \frac{\dd x_i}{\dd t} \right)^2 + 2 \frac{\dd}{\dd t} \left( \frac{\dd x_i}{\dd t} \delta x_i \right) - 2 \frac{ \dd ^2 x_i}{\dd t^2} \delta x_i\]
where second order terms in $\delta x_i$ were neglected. The first term also appears in the action for the original trajectory and the total derivative in the second term cancels the integral. Therefore
\[ 
\SSS( [x_i + \delta x_i]) = S([x_i]) + \left[ \int_{t_1}^{t_2} \dd t \sum_i \left( -m \frac{\dd ^2 x_i}{\dd t^2} - \frac{\p V}{\p x_i} \right) \delta x_i \right] + \left. m \sum_i \frac{\dd x_i}{\dd t} \delta x_i \right|_{t_1}^{t_2}
\]
Because the last term vanishes due to the boundary conditions and $\delta x_i$ is chosen arbitrarily $\Delta \SSS$ can only vanish if the term inside the integral is zero. Therefore
\[ 
m \frac{\dd x_i }{\dd t} = \frac{\p V}{\p x_i}
\]
This equation of motion is true for all $\delta x_i$.
\newline\newline
\textbf{Symmetries}\\
Lets assume the system has a rotational invariance $V = V(r)$ with $r = \sqrt{ \sum_i x_i^2}$. If a transformation $O_{ij}$ orthogonal to the rotational invariance is applied to the trajectory $x_j$ the action remains the same
\[ \SSS[ \sum_j O_{ij} x_j (t) ] = S[x_j(t)]\]
Also the equation of motions is invariant
\[ m \frac{ \dd^2 ( O_{ij} x_j)}{\dd t^2} = - \frac{ O_{ij} x_j}{r} \frac{\dd V}{\dd r} \]

\subsection{Field Theory Lagrangian}
In quantum field theory the action is given by
\[ \SSS([\phi_r]) = \int \dd ^4 x~\mathcal{L}(\phi_r, \p_\mu \phi_r)\]
with $\phi_r = \phi_r(\vx, t)$. $\LL$ is the Lagrange density. It is not directly dependant on $x$, because it should be invariant in the whole four dimensional space.\\
The integral here is over all four dimensions because time and space are treated equally in field theory.\\
There are some requirements to the Lagrange density:
\begin{enumerate}
\item $\LL$ is local - there are no connections or interactions between two arbitrary space points. Also there can not be any instantaneus interaction of two spacepoints because information travels at finite speeds.
\item $\LL$ is real - this is necessary to conserve probability
\item $\LL$ is Lorentz invariant - $x'^\mu = \Lambda^\mu_{~\nu}x^\nu ~\df~ \dd^4 x' = (\det \Lambda) \dd^4 x = \dd ^4 x$. Therefore also the action is Lorentz invariant.
\item there is no need for derivatives higher than the first, this is implied by causality (?)
\end{enumerate}
In natural units the action is dimensionless (whereas in SI units it has the same unit as $\hbar$). Also $\dd^4 x$ has units of $\si{\frac{1}{GeV^4}}$ in natural units, therefore $\LL$ has to have units of $\si{GeV}^4$
\newline\newline
\textbf{Extremal of Action}\\
Same as before we can calculate the extremal of the action $\SS$ for variations $\delta \phi_r$. Here the boundary condition has to be $\delta \phi_r(x) = 0$ for $x \in \p \Omega$ where $\p \Omega$ is the surface of the integrated space.\\
It follows
\[ 0 = \Delta \SSS = \int_\Omega \dd ^4 x \sum_r \left[ \frac{\p \LL}{\p \phi_r} \delta \phi_r + \frac{ \p \LL}{\p (\p_\mu \phi_r)} \p_\mu(\delta \phi_r) \right]\]

In the second term the equality of $\delta (\p _\mu \phi_r) = \p_\mu(\delta \phi_r)$ was used. Also we can rewrite the partial derivative in the second term to an absolute derivative
\[
\frac{\p \LL}{\p(\p_\mu \phi_r)} = \p_\mu \left( \frac{\p \LL}{\p ( \p_\mu \phi_r)} \delta\phi_r\right) - \delta \phi_r \p_\mu \frac{\p \LL}{\p(\p_\mu \phi_r)}
\]
Therefore
\[
\Delta \SSS = \int_\Omega \dd ^4 x \left[ \sum_r \delta \phi_r \left( \frac{\p \LL}{\p \phi_r} - \p_\mu \frac{\p \LL}{\p(\p_\mu \phi_r)}\right) + \p_\mu \left( \sum_r \delta \phi_r \frac{\p \LL}{\p(\p_\mu \phi_r)} \right)\right]
\]
The last term is rewritable into a surface integral via Gauss' theorem, therefore it vanishes due to the boundary conditions. Similar to the classical approach $\delta \phi_r$ can be chosen arbitrarily and therefore the action only vanishes if the first term is equal to zero
\[ 
\frac{\p \LL}{\p \phi_r} - \p_\mu \frac{\p \LL}{\p(\p_\mu \phi_r)} = 0
\]
These are the Euler-Lagrange equations.
\sub{Example}
Consider a scalar field $\phi(x)$. The Lagrange density is given by
\[ \LL = \frac{1}{2} (\p_\alpha \phi)(\p^\alpha \phi) - V(\phi) = \frac{1}{2} \left[ (\p_0\phi)^2 - \sum_i (\p_i \phi)^2\right]\]
So the equation of motion calculates to
\[ \frac{\p \LL}{\p \phi} = - V'(\phi) = \p_\mu \left( \frac{\p \LL}{\p(\p_\mu \phi)}\right) = \p_0 \p_0 \phi - \vn(\vn \phi) = \square \phi\]
\[\df \square \phi + V'(\phi) = 0\]
Different potentials then lead to different equations of motion
\begin{itemize}
\item $V(\phi) = \frac{m^2}{2}\phi^2 ~\df~ V' = m^2\phi$ leads to $\square\phi + m^2 \phi = 0$
which is the Klein-Gordon equation. Its Lagrange density is
\[ \LL = \frac{1}{2} (\p_\mu \phi)(\p^\mu \phi) - \frac{m^2}{2}\phi^2\]
\item $V(\phi) = \frac{m^2}{2}\phi^2 + \frac{\lambda}{4}\phi^4$ leads to $\square \phi + m^2 \phi + \lambda \phi^3  = 0$
\item $V(\phi) = A \cos \frac{\phi}{M}$ leads to $\square\phi - \frac{A}{M} \sin \frac{\phi}{M}$
which is called the sine-Gordon equation.
\end{itemize}

\subsection{Dirac Lagrangian}
For spin $\frac{1}{2}$ particles the Lagrangian is connected to the Dirac equation. It is given by
\[ \LL = \bar{\psi}(x)(i \dslash - m) \psi(x),~~~\bar{\psi} = \psi^\dagger \gamma^0 = (\psi^*)^T \gamma^0\]
where $\psi$ has four complex and eight real components.\\
The components of $\psi$ and $\bar{\psi}$ are treated as independant fields. This leads to the following equations of motion
\[ \frac{\p \LL}{\p \bar{\psi}} = ( i \dslash - m)\psi = \p_\mu \left( \frac{\p \LL}{\p(\p_\mu \bar{\psi})}:\right) = 0~~~~\df ~~ (i \dslash - m) \psi = 0\]
\[ \frac{\p\LL}{\p\psi} = -m \bar{\psi} = \p_\mu\left( \frac{\p\LL}{\p(\p_\mu \psi)}\right) = i \p_\mu(\bar{\psi}\gamma^\mu)~~~~\df ~~ i \p_\mu(\bar{\psi}\gamma^\mu) + m \bar{\psi} = 0\]


\[
\LL = \bar{\psi} \gamma^\mu \p_\mu \psi + i q \bar{\psi} \gamma^\mu A_\mu \psi + m \bar{\psi}\psi + \frac{1}{16 \pi} F_{\mu\nu}F^{\mu\nu}
\]
%\chapter{Introduction}
\section{Quarks and Leptons}
Particles of matter:
\begin{itemize}
	\item electrons ($e^-$) and other leptons are elementary particles. 
	\item protons and neutrons ($|p\rk = |uud\rk$, $|n\rk = |udd\rk$) are combinations of elementary quarks and gluons. The binding energy of the quarks is very large in comparison to the absolute energy of the proton and neutron ($m_pc^2 = 938~\si{MeV}$) if you compare this to the binding energies of Atoms ($\sim 1~\si{Ry}$) and their absolute energies ($\sim 10^9~\si{Ry}$).\\
		Because the proton and the neutron are similar/symmetric in the strong interaction (not in the electroweak interaction though) we can combine them into a isospin dublett $\vektorz{p}{n}$.
\end{itemize}
There are many more particles/boundstates of quarks and gluons for different combination of quarks. Another example are the $\Delta$ baryons. These are spin $\frac{3}{2}$ particles and have masses of $m_\Delta c^2 \approx 1230 ~\si{MeV}$:
\begin{itemize}
	\item $\Delta^-:~|ddd\rk$
	\item $\Delta^0:~|ddu\rk$
	\item $\Delta^+:~|duu\rk$
	\item $\Delta^{++}:~|uuu\rk$
\end{itemize}
Because the $\Delta$ baryons are spin $\frac{3}{2}$ particles all of the quarks spins must be aligned, so the spin wavefunction is symmetric. Also the orbital wavefunction is symmetric for the $\Delta^{++}$ baryon because it consists of thrice the same quark. However, the total wavefunction of the baryons must be antisymmetric because it is a fermion.\\
This is the reason a color charge was introduced to interpret the Dirac statistics correctly and characterize the strong interaction with a new quantum number.\\
Alltogether one can describe the four $\Delta$ baryons in an isospin quartet with $I = \frac{3}{2}$.\\
\\
Another group of particles are the mesons, they consist of one quark and an anti quark. The lightest examples are the pions:
\begin{itemize}
\item $\pi^+:~|u\bar{d}\rk$
\item $\pi^0:~\frac{1}{\sqrt{2}}\left( |u\bar{u}\rk - |d \bar{d}\rk\right)$
\item $\pi^-:~|d\bar{u}\rk$
\end{itemize}
They have masses of $m_\pi c^2 \approx 140~\si{MeV}$ and are spin $0$ particles. Together they form the isospin triplet $I = 1$.\\
\\
Another group of mesons with spin $0$ are the kaons. These have another type of quark, the strange quark. For this new quark a new quantum number (next to isospin) was introduced, the strangeness.\\
We can summarize the kaons and the pions in a meson octett depicted in figure \ref{1}. The kaons have masses of $m_Kc^2 \approx 495~\si{MeV}$. Additionally to the four kaons and the three pions there is an $\eta$ meson with the same strangeness and isospin as the $\pi^0$. It is like the $\pi^0$ but has additional strange quarks: $|\eta\rk = \frac{1}{\sqrt{6}}\left( |u\bar{u}\rk + |d\bar{d}\rk - 2 | s \bar{s}\rk\right)$.\\
\begin{figure}[H]
\centering
\includegraphics[scale=0.1]{include/mesonoctett.pdf}
\caption{meson octett for spin $0$}
\label{1}
\end{figure}

The quarks and leptons are probably the fundamental layer of particles; mesons and baryons are complex bound states described through nuclear physics. The quarks and leptons are described by the dirac equation
\[ \left( i \dslash - \frac{mc}{\hbar} \right)\psi = 0 + \si{interactions}\]
One interaction is for example the electromagnetism: $\p_\mu \df \p_\mu + iq A_\mu$

In table \ref{quarks} and \ref{leptons} all quarks and leptons are summarized with their electric charge and mass. The charge is given in units of elementary charge as $q = Q\cc e$.\\
\begin{minipage}{80mm}
\begin{table}[H]
\centering
\begin{tabular}{r|lr}
quark & $mc^2$ [MeV] & Q \\
\midrule
u & $2.2^{+0.6}_{-0.7}$ & $+\frac{2}{3}$\\
d & $4.7$ & $-\frac{1}{3}$\\
\midrule
c & 1270 & $+\frac{2}{3}$\\
s & 96 & $-\frac{1}{3}$\\
\midrule
t & 173200 & $+\frac{2}{3}$\\
b & 4180 & $-\frac{1}{3}$\\
\bottomrule
\end{tabular}
\caption{quarks}
\label{quarks}
\end{table}
\end{minipage}
\begin{minipage}{80mm}
\begin{table}[H]
\centering
\begin{tabular}{r|lr}
lepton & $mc^2$ [MeV] & Q\\
\midrule
$e^-$ & 0.511 & -1\\
$\mu^-$ & 105.66 & -1 \\
$\tau^-$ & 1777 & -1 \\
\midrule
$\nu_e$ & & 0\\
$\nu_\mu$ & & 0\\
$\nu_\tau$ & &0\\
\bottomrule
\end{tabular}
\caption{leptons}
\label{leptons}
\end{table}
\end{minipage}

It is important, that the down quark is slightly heavier than the up quark; because then the down quark more likely decays into the up quark than vice versa and therefore the proton is much more stable than the neutron. This way also atoms remain stable and charged.\\
\\
These leptons and quarks are all known matter fields save the bosons:
\begin{itemize}
\item Higgs boson $H$
\item $\gamma$, $W^\pm$, $Z$ which are carriers of the electromagnetic and weak force
\item gluon $g$ which is the carrier of the strong force
\end{itemize}
Additionaly it is known, from observing the Higgs coupling, that there are no more generations of quarks which behave similarly to the three existing generations. Additional generations might exist but must behave fundamentally different.

\section{Course Contents}
In this course of theoretical particle physics the following topics will be discussed:
\begin{itemize}
\item theoretical description of interactions of quarks and leptons\\
$\df$ gauge theories (\textit{Eichtheorien})
\item pair production of particles and $\gamma$, $W^\pm$, $Z$, $g$ emission\\
$\df$ changing particle number and content\\
$\df$ quantum field theory (QFT) which is relativistic for particle physics
\item development of pertubation theory for QFT
\item calculation of cross sections and decay rates
\item symmetries: Lorentz invariance and internal symmetries like isospin and color
\end{itemize} 

\section{Natural Units}
In particle physics it is not practical to use the usual units. It is much more practicable to factor out constants like $\eps_0$, $\hbar$, $c$ and $k_B$ such that the remaining quantities have dimensions of energy to a power.\\
The unit of energy will be electron volts (eV). In table \ref{units} some important quantities and their dimensions in natural units are shown.
\begin{table}[H]
\centering
\begin{tabular}{l|lll}
quantity & SI units & natural units & dimension\\
\midrule
velocity & $\tilde{v}$ & $v \cc c$ & $[v] = 1$\\
length & $\tilde{L}$ & $L \cc \hbar c$ & $[L] = 1/\si{MeV}$\\
time & $\tilde{t}$ & $t\cc \hbar$ & $[t] = 1/\si{MeV}$\\
electric field & $\tilde{E}$ & $\frac{1}{\sqrt{\eps_0 (\hbar c)^3}} \vv{E}$ & $[\vv{E}] = \si{MeV}^2$\\
magnetic field & $\tilde{B}$ & $\frac{1}{\sqrt{\eps_0 c^2(\hbar c)^3}} \vv{B}$ &  $[\vv{B}] = \si{MeV}^2$\\
\bottomrule
\end{tabular}
\caption{natural units}
\label{units}
\end{table}
An example of the simplification is the Hamiltionan for radiation:
\[ H_{rad} = \frac{\eps_0}{2} \int\dd^3 \tilde{\vx} \left[ \tilde{\vE}^2 + c^2 \tilde{\vB}^2\right] \df \frac{1}{2} \int\dd^3 \vx \left[\vE^2 + \vB^2\right]\]

Another useful thing are translations from the normal system to the natural units and vice versa. For example:
\begin{itemize}
\item $\hbar c = 197~\si{MeV fm}$
\item $\frac{1}{\si{GeV}^2} = \frac{3.89\cc 10^{-4}~\si{b}}{(\hbar c)^2}$ where a barn is $10^{-28}~\si{m^2}$
\item $\tilde{e} = 1.6\cc 10^{-19} ~\si{C} ~~\df~~ e = \frac{\tilde{e}}{\sqrt{\eps_0 \hbar c}} = \sqrt{ 4 \pi \alpha} = 0.3028$
\end{itemize}

\subsection{Klein Gordon and Dirac Equations in Natural Units}
The Klein-Gordon equation in SI units is given as
\[
\left[ \tilde{\square} + \left( \frac{mc}{\hbar}\right)^2 \right] \phi(\tilde{x}) = 0
\]
where $x$ is a four vector $\tilde{x}^\mu = ( c \tilde{t}, \tilde{\vx}) = \hbar c (t , \vx)$. Also the d'Alembert operator in SI units is given as
\[ 
\tilde{\square} = \frac{1}{c^2} \frac{\p^2}{\p \tilde{t}^2} - \tilde{\vn}^2 = \frac{1}{(\hbar c)^2} \square = \frac{1}{(\hbar c)^2} \left( \frac{\p^2}{\p t^2} - \vn^2 \right)
\]
So in natural units the equation simplifies to 
\[ ( \square + m^2 ) \phi(x) = 0\]
\newline
Similarly the Dirac equation simplifies when using natural units
\[
\left( i \gamma^\mu \tilde{\p}_\mu - \frac{m c}{\hbar}\right) \psi(\tilde{x}) = 0 ~~\df~~ (i \gamma^\mu \p_\mu - m ) \psi(x) = 0
\]

\section{Lagrange Density and Equations of Motion}

\subsection{Lagrangian Field Theory}
First we take a look at a classical point particle. Its trajectory is given by $x_i(t)$ for $i = 1,2,3$.\\
For this particle we can define an action
\[ \SSS( [x_i], t_1, t_2 ) = \int_{t_1}^{t_2} \dd t \left(\frac{1}{2} m \left( \sum_i \frac{\dd x_i}{\dd t} \right)^2 - V(x_i(t)) \right)
\]
The action is a functional of the trajectory. Now we can find an extremum of $\SSS$ for the classical path by adding a inifinitesimal variation $\delta x_i$ to the trajectory: $x_i(t) + \delta x_i(t)$. Then, the extremal condition is given by $\Delta S = 0$ where $\Delta S$ is given by
\[ \Delta \SSS = \SSS( [x_i + \delta x_i]) - \SSS([x_i]) = 0\]
where the boundary condition is set, so the variation $\delta x_i$ vanishes at the endpoints
\[ \delta x_i(t_1) = \delta x_i(t_2) = 0\]
Calculating the action for the changed trajectory leads to
\[ \SSS( [x_i + \delta x_i]) = \int_{t_1}^{t_2} \dd t \left( \frac{1}{2} m \left( \frac{\dd x_i}{\dd t} + \frac{\dd (\delta x_i)}{\dd t} \right)^2 - V(x_i + \delta x_i) \right)
\]
with
\[  \left( \frac{\dd x_i}{\dd t} + \frac{ \dd ( \delta x_i)}{\dd t} \right) = \left( \frac{ \dd x_i}{\dd t} \right)^2 + 2 \frac{ \dd x_i }{\dd t} \frac{ \dd (\delta x_i)}{\dd t} = \left( \frac{\dd x_i}{\dd t} \right)^2 + 2 \frac{\dd}{\dd t} \left( \frac{\dd x_i}{\dd t} \delta x_i \right) - 2 \frac{ \dd ^2 x_i}{\dd t^2} \delta x_i\]
where second order terms in $\delta x_i$ were neglected. The first term also appears in the action for the original trajectory and the total derivative in the second term cancels the integral. Therefore
\[ 
\SSS( [x_i + \delta x_i]) = S([x_i]) + \left[ \int_{t_1}^{t_2} \dd t \sum_i \left( -m \frac{\dd ^2 x_i}{\dd t^2} - \frac{\p V}{\p x_i} \right) \delta x_i \right] + \left. m \sum_i \frac{\dd x_i}{\dd t} \delta x_i \right|_{t_1}^{t_2}
\]
Because the last term vanishes due to the boundary conditions and $\delta x_i$ is chosen arbitrarily $\Delta \SSS$ can only vanish if the term inside the integral is zero. Therefore
\[ 
m \frac{\dd x_i }{\dd t} = \frac{\p V}{\p x_i}
\]
This equation of motion is true for all $\delta x_i$.
\newline\newline
\textbf{Symmetries}\\
Lets assume the system has a rotational invariance $V = V(r)$ with $r = \sqrt{ \sum_i x_i^2}$. If a transformation $O_{ij}$ orthogonal to the rotational invariance is applied to the trajectory $x_j$ the action remains the same
\[ \SSS[ \sum_j O_{ij} x_j (t) ] = S[x_j(t)]\]
Also the equation of motions is invariant
\[ m \frac{ \dd^2 ( O_{ij} x_j)}{\dd t^2} = - \frac{ O_{ij} x_j}{r} \frac{\dd V}{\dd r} \]

\subsection{Field Theory Lagrangian}
In quantum field theory the action is given by
\[ \SSS([\phi_r]) = \int \dd ^4 x~\mathcal{L}(\phi_r, \p_\mu \phi_r)\]
with $\phi_r = \phi_r(\vx, t)$. $\LL$ is the Lagrange density. It is not directly dependant on $x$, because it should be invariant in the whole four dimensional space. The integral here is over all four dimensions because time and space are treated equally in field theory.\\
There are some requirements to the Lagrange density:
\begin{enumerate}
\item $\LL$ is local - there are no connections or interactions between two arbitrary space points. Also there can not be any instantaneus interaction of two spacepoints because information travels at finite speeds.
\item $\LL$ is real - this is necessary to conserve probability
\item $\LL$ is Lorentz invariant - $x'^\mu = \Lambda^\mu_{~\nu}x^\nu ~\df~ \dd^4 x' = (\det \Lambda) \dd^4 x = \dd ^4 x$. Therefore also the action is Lorentz invariant.
\item there is no need for derivatives higher than the first, this is implied by causality (?)
\end{enumerate}
In natural units the action is dimensionless (whereas in SI units it has the same unit as $\hbar$). Also $\dd^4 x$ has units of $\si{\frac{1}{GeV^4}}$ in natural units, therefore $\LL$ has to have units of $\si{GeV}^4$
\newline\newline
\textbf{Extremal of Action}\\
Same as before we can calculate the extremal of the action $\SSS$ for variations $\delta \phi_r$. Here the boundary condition has to be $\delta \phi_r(x) = 0$ for $x \in \p \Omega$ where $\p \Omega$ is the surface of the integrated space.\\
It follows
\[ 0 = \Delta \SSS = \int_\Omega \dd ^4 x \sum_r \left[ \frac{\p \LL}{\p \phi_r} \delta \phi_r + \frac{ \p \LL}{\p (\p_\mu \phi_r)} \p_\mu(\delta \phi_r) \right]\]

In the second term the equality of $\delta (\p _\mu \phi_r) = \p_\mu(\delta \phi_r)$ was used. Also we can rewrite the partial derivative in the second term to an absolute derivative
\[
\frac{\p \LL}{\p(\p_\mu \phi_r)} = \p_\mu \left( \frac{\p \LL}{\p ( \p_\mu \phi_r)} \delta\phi_r\right) - \delta \phi_r \p_\mu \frac{\p \LL}{\p(\p_\mu \phi_r)}
\]
Therefore
\[
\Delta \SSS = \int_\Omega \dd ^4 x \left[ \sum_r \delta \phi_r \left( \frac{\p \LL}{\p \phi_r} - \p_\mu \frac{\p \LL}{\p(\p_\mu \phi_r)}\right) + \p_\mu \left( \sum_r \delta \phi_r \frac{\p \LL}{\p(\p_\mu \phi_r)} \right)\right]
\]
The last term is rewritable into a surface integral via Gauss' theorem, therefore it vanishes due to the boundary conditions. Similar to the classical approach $\delta \phi_r$ can be chosen arbitrarily and therefore the action only vanishes if the first term is equal to zero
\[ 
\frac{\p \LL}{\p \phi_r} - \p_\mu \frac{\p \LL}{\p(\p_\mu \phi_r)} = 0
\]
These are the Euler-Lagrange equations.
\sub{Example}
Consider a scalar field $\phi(x)$. The Lagrange density is given by
\[ \LL = \frac{1}{2} (\p_\alpha \phi)(\p^\alpha \phi) - V(\phi) = \frac{1}{2} \left[ (\p_0\phi)^2 - \sum_i (\p_i \phi)^2\right]-V(\phi)\]
So the equation of motion calculates to
\[ \frac{\p \LL}{\p \phi} = - V'(\phi) = \p_\mu \left( \frac{\p \LL}{\p(\p_\mu \phi)}\right) = \p_0 \p_0 \phi - \vn(\vn \phi) = \square \phi\]
\[\df \square \phi + V'(\phi) = 0\]
Different potentials then lead to different equations of motion
\begin{itemize}
\item $V(\phi) = \frac{m^2}{2}\phi^2 ~\df~ V' = m^2\phi$ leads to $\square\phi + m^2 \phi = 0$
which is the Klein-Gordon equation. Its Lagrange density is
\[ \LL = \frac{1}{2} (\p_\mu \phi)(\p^\mu \phi) - \frac{m^2}{2}\phi^2\]
\item $V(\phi) = \frac{m^2}{2}\phi^2 + \frac{\lambda}{4}\phi^4$ leads to $\square \phi + m^2 \phi + \lambda \phi^3  = 0$
\item $V(\phi) = A \cos \frac{\phi}{M}$ leads to $\square\phi - \frac{A}{M} \sin \frac{\phi}{M}$
which is called the sine-Gordon equation.
\end{itemize}

\subsection{Dirac Lagrangian}
For spin $\frac{1}{2}$ particles the Lagrangian is connected to the Dirac equation. It is given by
\[ \LL = \bar{\psi}(x)(i \dslash - m) \psi(x),~~~\bar{\psi} = \psi^\dagger \gamma^0 = (\psi^*)^T \gamma^0\]
where $\psi$ has four complex and eight real components.\\
The components of $\psi$ and $\bar{\psi}$ are treated as independant fields. This leads to the following equations of motion
\[ \frac{\p \LL}{\p \bar{\psi}} = ( i \dslash - m)\psi = \p_\mu \left( \frac{\p \LL}{\p(\p_\mu \bar{\psi})}\right) = 0~~~~\df ~~ (i \dslash - m) \psi = 0\]
\[ \frac{\p\LL}{\p\psi} = -m \bar{\psi} = \p_\mu\left( \frac{\p\LL}{\p(\p_\mu \psi)}\right) = i \p_\mu(\bar{\psi}\gamma^\mu)~~~~\df ~~ i \p_\mu(\bar{\psi}\gamma^\mu) + m \bar{\psi} = 0\]


\[
\LL = \bar{\psi} \gamma^\mu \p_\mu \psi + i q \bar{\psi} \gamma^\mu A_\mu \psi + m \bar{\psi}\psi + \frac{1}{16 \pi} F_{\mu\nu}F^{\mu\nu}
\] 


\chapter{Groups and Symmetries}
\section{Representation of Groups}
A group is an object of the form $G = \{ g_i | i = 1,\ldots\}$ where $g_i$ are the elements of the group. In the group a multiplication exists so, that $g_1 g_2 = g_3 \in G$. This also has the following traits:
\begin{itemize}
\item it has to be associatiove: $g_1(g_2g_3) = (g_1g_2)g_3$
\item there is an identity element $e \equiv 1$ so that $g\cc e = e \cc g = g ~~\forall g \in G$
\item $\forall g \in G$ there is an inverse $g_{-1} \in G$: $gg^{-1} = g^{-1}g = e$
\end{itemize}
The representation of $G$ is a mapping $r: G \df \mathbbm{C}^{(n,n)}$ where $r(g_i) = M_i$ is a $n\times n$-matrix.\\
With this representation the multiplication rules are being preserved: $r(g_1g_2) = r(g_1)r(g_2)$ or $M_3 = M_1M_2$. Also $r(e) = \ehm_n$ and $r(g^{-1}g) = MM^{-1} = \ehm$.\\
A reducable representation is a representation such that a single unitary $n\times n$ matrix $U$ exists such that
\[ U M_i U^{-1} = \Matz{M_i'}{0}{0}{M_i''}\]
So instrad of working with $M_i$ we could have worked with the block diagonal matrices $M_i'$ and $M_i''$.\\
An irreducable representation then is a representation where no unitary matrix exists that splits the matrix $M_i$ into block diagonal matrices.\\
For finite groups $G = \{ g_i | i = 1, \ldots, n \}$ the dimensions $d_r$ of all the irreducible representations are bounded by 
\[n = \sum_r d_r^2\]
Therefore an infinite group has infinte number of different finite dimensional irreducible representations.
\section{Lie groups and Lie algebra}
Lie groups are a special case of groups. They are parametrizised by $G = \{ U(\theta)| \theta = ( \theta_1, \ldots, \theta_n) \in \mathbbm{R}^n\}$ with $U(0) = e$. \\
$U(\theta)$ is analytic in all its components; it is infinitely differentiable.\\
The simplest example if a Lie group is the three dimensional rotation group $SO(3)$ with the rotation matrices $R(\phi,\psi,\theta)$. The three rotations are\\
the rotation around the $z$-axis:
\[ R_z(\theta) = \Mat{\cos\theta}{\sin\theta}{0}{- \sin\theta}{\cos\theta}{0}{0}{0}{1}\]
the rotation around the $y$-axis:
\[ R_y(\psi) = \Mat{\cos\psi}{0}{-\sin\psi}{0}{1}{0}{\sin\psi}{0}{\cos\psi}\]
the rotation around the $x$-axis:
\[ R_x(\phi) = \Mat{1}{0}{0}{0}{\cos\phi}{\sin\psi}{0}{-\sin\phi}{\cos\phi}\]


\sub{Lie algebra}
As the elements of the Lie group are infinitely differentiable we can apply the definition of the Taylor expansion onto an element of the group. This leads to the generators of the group
\[L_a \equiv \frac{1}{i} \left.\frac{\p U}{\p \phi_a} \right|_{\theta = 0}\]
These generators completely describe the groups properties.\\
The Taylor expansion also leads to infinitesimal transformations:
\[ U(\theta) = 1 + i \sum_a \theta_a L_a + \ldots\]
In summation convention this also can be written as $\sum_a \theta_a L_a = \theta_a L_a = \theta\cc L$.\\
As one general group property is the formation of the $1$-element: $U(\theta)U^{-1}(\theta) = 1$ this also has to be applicable for inifitiesimal transformations. This leads to
\[ G \ni U(\theta)U(\psi)U^{-1}(\theta)U^{-1}(\psi) \neq 1\]
which is not the $1$-element for non commuting groups. The Taylor expanison of this expression leads to
\[ 1 + i^2 \theta_a\psi_b ( L_aL_b + L_aL_b - L_aL_b - L_bL_a) + \ldots = 1 - \theta_a \psi_b [L_a, L_b]+\ldots = 1 + i \sum_c \theta_a \psi_b ( - L_c f^{abc})\]
In the first step the only terms of first order are either linear in $\psi$ or $\theta$ so they cancel. In the second order all terms quadratic in $\theta$ or $\psi$ also cancel and the only things left are mixed terms. In the second step the definition of the commutator was applied and in the last step we used, that the resulting element had to be in the group $G$ again, so it must be able to be written as a taylor expansion again. This therefore leads to the identity
\[ [ L_a, L_b] = -i L_c f^{abc}\]
Where $f^{abc}$ are group specific structure constants.\\
For the $SO(3)$ group this is for example the known $\eps$-tensor. The generators are the components of angular momentum:
\[L_x = \Mat{}{}{}{}{}{-i}{}{i}{}, ~~ L_y = \Mat{}{}{i}{}{}{}{-i}{}{}, ~~L_z = \Mat{}{-i}{}{i}{}{}{}{}{}\]
The identity here is $[L_x,L_y] = iL_z$.\\
These generators form the basis of the Lie algebra.\\
\\
Another trait of the group can be shown for a finite group with $\theta = (\theta_1, \ldots, \theta_n)$. If we use $\eps = \frac{1}{N}\theta$ then for high $N$, $\eps$ becomes small. So a Taylor expansion can be applied:
\[ U(\eps) = 1 + i \eps_a L_a = 1 + i \frac{\theta_a L_a}{N}\]
So the original $\theta$ can be written as the following
\[ U(\theta) = U(\eps)^N \df \lim_{N \df \infty} U(\eps) ^N = \lim_{N\df\infty} \left( 1 + i \frac{\theta_aL_a}{N}\right)^N = e^{i\theta_aL_a} = \sum_{k = 0}^{\infty}\frac{1}{k!}(i\theta_aL_a)^k\]
The $i$ was introduced in the definition of the generators, so the generators would be hermitian operators. If we look at $U(\theta)$, which is unitary it follows
\[ U(\theta) = e^{i\theta L} ~~\df~~ U^{-1}(\theta) = e^{-i\theta L}\]
and thus
\[ e^{-i\theta_aL_a^\dagger} = U^\dagger(\theta) = U^{-1}(\theta) = e^{-i\theta_aL_a}\]
From this we can see, that $L_a$ and $L_a^\dagger$ must be the same.


\sub{Special unitary groups}
Another group of Lie groups are the special unitary groups $SU(N) = \{ U \in\mathbbm{C}^{(N\times X)} | U^{-1} = U^\dagger, \det(U) = 1\}$. Its group elements are unitary and have a determinante of one. The number of generators $L_a$ can be expressed formally for every $N$:\\
$L_a$ has to be a hermitian $N\times N$ matrix with trace $\tr(L_a) = 0$. This follows from
\[ \det(U) = \det \left( e^{i \theta_a L_a}\right) = e^{i \tr(\theta_a L_a)} \soll 1 ~~\df ~~ \tr(L_a) = 0\]
Now because there are $N^2$ matrices but one of them neccesarily is $\ehm$ which is not traceless, there are always $N^2-1$ generators in $SU(N)$.\\
For $N = 2$ the generators are $L_a = \frac{\sigma_a}{2}$ where $\sigma_a$ are the Pauli matrices.\\
For $N = 3$ the generators are the eight Gell-Mann matrices. They have the following forms
\begin{align*} \lambda_1 &= \Mat{}{1}{}{1}{}{}{}{}{},~~~\lambda_2 = \Mat{}{-i}{}{i}{}{}{}{}{},~~~\lambda_3 = \Mat{1}{}{}{}{-1}{}{}{}{}\\
\lambda_4 &= \Mat{}{}{1}{}{}{}{1}{}{}, ~~~\lambda_5 = \Mat{}{}{-i}{}{}{}{i}{}{}\\
\lambda_6 &= \Mat{}{}{}{}{}{1}{}{1}{}, ~~~\lambda_7 = \Mat{}{}{}{}{}{-i}{}{i}{}, ~~~\lambda_8 = \frac{1}{\sqrt{3}} \Mat{1}{}{}{}{1}{}{}{}{-2}
\end{align*}
These matrices have normalization conditions which also apply for abitrary $N$:
\[ \tr(\lambda^a \lambda^b) = 2\delta^{ab}, ~~~ L_a = \frac{\lambda_a}{2}\]


~\newline\newline\sub{Rank of the Lie algebra}
The Lie algebra also has a rank. The rank is defined as the maximum number of commuting generators in the algebra. $SU(N)$ has $N-1$ diagonal generators, which naturally all commute. So the rank of an arbitraty $SU(N)$ algebra is always $N-1$.\\
These $N-1$ eigenvalues of these $L_a$ then specify the basis states in a $SU(N)$ multiplet.\\
For example for $N = 2$ the rank is one, so it has only one invariant, which is $\vv{J}^2$ so that $[\vv{J}^2, J_i] = 0$.\\
Generally, if we have any polynomial such that $C = \eta_{ab} L_a L_b + \eta_{abc} L_aL_bL_c + \ldots$ that commutes with all generators, $[C,L_a] = 0 ~\forall a$ then it is called a Casimir invariant.\\
For $SU(2)$ the only Casimir invariant is $\vv{J}^2$, for $SU(3)$ there are two different Casimir invariants.\\
For any $SU(N)$ there is a Casimir invariant $C_2 = L_aL_a$ which is also referred to as the quadratic Casimir of the $SU(N)$ group.\\
Furhtermore, the eigenvalues of all independant Casimir invariants do specify an irreducible representation.


\sub{Adjoint irreducible representation}
If we take a look at the Jacobi identity
\[ [A,[B,C]] + [B,[C,A]] + [C,[A,B]] = 0\]
and take $A = L_a$, $B= L_b$, $C=L_j$ we get
\[ [A,B] = i f_{abm} L_m ~~~\df~~~ [C,[A,B]] = i f_{abm}[L_j,L_m] = if_{abm}if_{jmc} L_c\]
from this follows the relationship
\[ (-if_{aim}(-if_{bmj}) - (-if_{bim})(-if_{amj}) = if_{abc}(-if_{aij})\]
If we now write $if_{abc} = (F^a)_{bc}$ as a matrix element of a matrix $F^a$ we get
\[ (F^a)_{im} (F^b)_{mj} - (F^b)_{im}(F^a)_{mj} = i f_{abc} (F^c)_{ij}\]
due to the summation convention this is equivalent with
\[ (F^aF^b)_{ij} - (F^bF^a)_{ij} = if_{abc}(F^c)_{ij}\]
So this in it self satisfies the commutation relation of the Lie algebra:
\[ [F^a, F^b] = i f^{abc} F^c\]
So these objects for an irreducible representation of the Lie algebra called the adjoint irreducible representation.\\
\\
In summary, for any $SU(N)$ group there are three irreducible representations:
\begin{itemize}
\item the trivial representation: $U(\theta) = 1$
\item the fundamental representation: $U(\theta) = \exp\left( i \frac{\lambda^a}{2} \theta^a\right)$
\item the adjoint representation: $U(\theta) = \exp \left( i F^a \theta ^a\right)$
\end{itemize}


\section{The Lorentz Group and Relativistic Invariance}
\sub{Lorentz transformations}
COnsider two inertial frames with common origin at $t = 0$ and moving with relative velocity $v$ along the $x$-axis. Lets assume that in the first frame an event is taking place at $x^\mu = (t, \vx)^T$ and is seen at $x'^\mu = (t', \vx')^T$ in the primed frame. The transformation is as follows
\[ x'^\mu = \vektorz{t'}{\vx'} = \begin{pmatrix} \gamma t - \gamma v t \\  \gamma v t + \gamma x\\ y \\z\end{pmatrix} = \begin{pmatrix} \gamma & - \gamma v & 0 & 0 \\ -\gamma v & \gamma & 0 & 0 \\ 0 & 0 & 1 & 0 \\ 0 & 0 & 0 & 1\end{pmatrix}\cc x^\mu\]
The transformation matrix is called $\Lambda$ so that $x'^\mu = \Lambda^\mu_{~\alpha} x^\alpha \equiv L x$.\\
\\A Lorentz transformation now is any linear transformation $\Lambda$ which keeps the relative length invariant:
\[ s^2 = x'^\mu x'^\nu g_{\mu\nu} = \Lambda^\mu_{~\alpha} \Lambda^\nu_{~\beta}x^\alpha x^\beta g_{\mu\nu}\soll x^\alpha x^\beta g_{\alpha\beta}\]
This must hold for all possible $x^\mu$.\\
From this follows
\[ \Lambda^\mu_{~\alpha} \Lambda^\nu_{~\beta} g_{\mu\nu} = g_{\alpha\beta}\]
If we now regard $x^\mu$ as a column vector $x$ and $\Lambda^\mu_{~\nu}$ as the elements of some $4\times 4$ matrix $L$ then follows
\[x' = L \cc x, ~~~~ s^2 = x^Tg x\]
Then the condition on the $\Lambda$'s reads
\[g^\alpha_{~\beta} = \Lambda_\mu^{~\alpha} g^\mu_{~\nu} \Lambda^\nu_{~\beta}~~\df~~ g_{\alpha\beta} = \Lambda^\mu_{~\alpha} g_{\mu\nu}\Lambda^\nu_{~\beta} = (L^TgL)_{\alpha\beta}\]
Now, because $g$ is symmetric ($g_{\alpha\beta} = g_{\beta\alpha}$) also $L^TgL$ must be symmetric. Therefore there are only ten conditions on $L$ (ten independent elements).\\
If we now take the determinante of that expression we find
\[\det(g) = \det(L^T)\det(g) \det(L)~~~\df~~~ \det(L) = \pm 1\]
and with the Jacobi determinant follows
\[ \int \dd ^4 x' = \int \dd ^4 x| \det(L) | = \int \dd ^4 x\]
We call Lorentz transformations with $\det(L) = 1$ proper Lorentz transformations and Lorentz transformations with $\det(L) = -1$ improper.\\
Also, if we take $\alpha = \beta = 0$ in the equation, we find 
\[g_{00} = 1 = \Lambda^{\mu}_{~0} g_{\mu\nu} \Lambda^\nu_{~0} = (\Lambda^0_{~0})^2 - (\Lambda^i_{~0})^2\]
So necessarily $|\Lambda^0_{~0}| \geq 1$.\\
We then call Lorentz transformations with $\Lambda^0_{~0} \geq 1$ orthochronous and Lorentz transformations with $\Lambda^0_{~0} \leq 1$ non orthonormous.

\sub{Lorentz group}
All Lorentz transformations form a group, the Lorentz group. It has the following attributes
\begin{itemize}
\item the product of two Lorentz transformations is again a Lorentz transformation
\item the inverse exists. In the example of the beginning it would replace $v \df -v$.
\item there is a unitary element ($L = \ehm_4$)
\end{itemize}
Because a product of Lorentz transformations is again a Lorentz transformation we can reduce the types of different transformations to four:
\begin{enumerate}
\item time inversions: $x'^0 = - x^0, x'^i = x^i$ (non orthochronous, improper)
\item parity transformation: $x'^0 = x^0, x'^i = - x^i$ (orthochronous, improper)
\item rotations: $x'^0 = x^0, x'^i = a^{ij} x^j$ with a rotation matrix $a^{ij}$, $L = \Matz{1}{0}{0}{L}$ where $a$ is the $3\times 3$ rotation matrix with $\det(a) = 1$. (orthochronous, proper)
\item boosts (here in one direction): $x'^0 = x^0 \cosh \eta + x^1 \sinh \eta$, $x'^1 = x^0 \sinh\eta x^1 \cosh\eta$, $x'^{(2,3)} = x^{(2,3)}$ where $\eta$ is the rapidity. (orthochronous, proper)
\end{enumerate}
The number of possible rotations is three (for example the Euler angles), the number of boosts also is three (for example the three boost directions or two angles plus $v$). Therefore, any proper, orthochronous Lorentz transformation can be described by six real parameters. The time inversion and party transformation do not have infinitesimal representations, because they are discrete transformations.

\sub{Inifitesimal generators of proper and orthochronous Lorentz transformations}
For any adequate description of the Lorentz group we need to study infinitesimal generators. Therefore we consider the infinitesimal Lorentz transformation:
\[ \Lambda^\mu_{~\nu} = \delta^\mu_{~\nu} + \eps^\mu_{~\nu} \equiv g^\mu_{~\nu} + \eps^\mu_{~\nu}\]
Now, the condition $g = L^T g L$ can be rewritten as 
\[ \delta^\alpha_{~\beta} = g^\alpha_{~\beta} = \Lambda_\mu^{~\alpha} g^\mu_{~\nu}\Lambda^\nu_{~\beta} = \Lambda_\mu^{~\alpha}\Lambda^\mu_{~\beta} = ( g_\mu^{~\alpha} + \eps_\mu^{~\alpha})(g^\mu_{~\beta} + \eps^\mu_{~\beta}) = g_\beta^{~\alpha} \eps_\beta^{~\alpha} + \eps^\alpha_{~\beta} + \OO(\eps^2)\]
From there follows $\eps_\beta^{~\alpha} + \eps^\alpha_{~\beta} = 0$, for example $\eps_{\alpha\beta} = - \eps_{\beta\alpha}$. So this is antisymmetric with six independant real elements.\\
\\
We now introduce $L_{\mu\nu} = i ( x_\mu \p_\mu - x_\nu\p_\mu)$ with $\p_\mu = \left( \frac{\p}{\p t}, \vn\right)^T$ as a generalization of the angular momentum operator $J^i -i \eps^{ijk} x^j \p_k$. It comes as as surprise, that these $L_{\mu\nu}$ exactly are the generators of the infinitesimal Lorentz transformation:
\[J^i = i \eps^{ijk} x_j \p _k = \frac{1}{2}\eps^{ijk} L_{jk}\]
With $\delta x^\mu = \Lambda^\mu_{~\nu} x^\nu - x^\mu = \eps^\mu_{~\nu} x^\nu$ it follows
\[ \frac{i}{2} e^{\rho\sigma} L_{\rho\sigma} x^\mu = \frac{i}{2} i \eps^{\rho\sigma} ( x_\rho g_\sigma^{~\mu}- x_\sigma g_\rho^{~\mu}) = - \frac{1}{2} (e^{\rho\mu} x_\rho - \eps^{\mu\sigma}x_\sigma )= \eps^{\mu\sigma}x_\sigma = \delta x^\mu\]
where we used $\eps^{\rho\mu} = - \eps^{\mu\rho}$. Therefore
\[ \frac{i}{2}\eps^{\rho\sigma} L_{\rho\sigma} x^\mu = \eps^\mu_{~\nu} x^\nu\]
So the $L_{\rho\sigma}$ are indeed the generators of rotations in Minkovski space, explicitly, the $SO(3,1)$. The Lie algebra is
\[ [L_{\mu\nu}, L_{\rho\sigma}] = i ( g_{\nu\rho} L_{\mu\sigma} - g_{\mu\rho}L_{\nu\sigma} - g_{\nu\sigma}L_{\mu\rho} + g_{\mu\sigma}L_{\nu\rho})\]
\newline
As a generalization, $L_{\mu\nu}$ is analogous to orbital angular momentum. We can add a spin term, which of course commutes with $L$ and forms some algebra similar to the $L$'s among themselves.\\
The most general representation of $SO(3,1)$ generators is by $M_{\mu\nu} \equiv i(x_\mu \p_\nu - x_\nu\p_\mu) + S_{\mu\nu}$. The $M_{\mu\nu}$ space comppnents, or more familiarly, the $J^i = \frac{1}{2} \eps^{ijk} M_{jk}$ components are the generators of inifinetismal rotations with $[J_i, J_j] = i \eps^{ijk} J^k$ (the commutation relation form $SO(2)$).\\
The $M^{0i} \equiv K^i$ are space time components and generate the boosts.\\
The commutation relation of these $K^i$ and the known $J^i$ are
\[ [K^i, K^j] = - \eps^{ijk} J^k, ~~~~ [J^i, K^j] = i \eps^{ijk} K^k\]
These are much simpler than the commutation relation of the $L_{\mu\nu}$.\\
Also, there usually is a trick to separate two $SO(2)$'s from each other by taking the linear combinations
\[N^i \equiv \frac{1}{2}(J^i + i K^i), ~~~~~ N^{i\dagger} \equiv \frac{1}{2} ( J^i - i K^i)\]
For these we find the commutation relations
\[ [N^i, N^{i\dagger}] = 0, ~~~ [N^i, N^j] = i \eps^{ijk} N^k, ~~~ [N^{i\dagger}, N^{j\dagger} = i \eps^{ijk} N^{k\dagger}\]
Because the first one is zero, $N^\dagger$ and $N$ are decoupled.\\
Finally, we find, that the finite dimensional representations of the restriced Lorentz group is characterized by $(n,m)$ where $n,m = 0,  \frac{1}{2}, 1, \ldots$ and are given by the eigenvalues of the Casimir operators:
\[N^iN^i|n,n_3\rk = n(n+1)|n,n_3\rk, ~~~~ N^{i\dagger}N^{i\dagger}|m,m_3\rk = m(m+1)|m,m_3\rk\]


\section{Feldtransformationen: Darstellungen der Lorentzgruppe}

\chapter{Klassiche Feldtheorie: Lagrangians}
\section{Bewegungsgleichungen}
\section{Symmetrien (Noether's Theorem)}
\section{Eichsymmetrie, Eichfelder}

\chapter{Kanonische (zweite) Quantisierung von Spin 0, 1/2, 1 Feldern}
\section{Erzeugungs- und Vernichtungsoperatoren}
\section{Fockraum}
\section{Propagatoren}
\section{Gupta Bleuler Quantisierung des Photons}

\chapter{S-Matrix, LSZ Reduktionsformel}

\chapter{Störungstheorie}	
\section{Feynman Regeln der QED}
\section{Wirkungsquerschnitte und Zerfallsraten}
\section{radiative Korrekturen}

\end{document}
