\documentclass{include/thesisclass}
	
%% -------------------------
%% |    Thesis Settings    |
%% -------------------------
\usepackage{bbm}
\usepackage{float}
\SelectLanguage{english}
\usepackage{slashed}
% details on this thesis
\newcommand{\thesisauthor}{Jan van der Linden}
\newcommand{\thesistopic}{Lecture notes\\Theoretische Teilchenphysik I}
\newcommand{\thesisreviewerone}{Prof. Dr. D. Zeppenfeld}
\newcommand{\thesistimestart}{SS 17} 
\newcommand{\thesispagehead}{theoretical particle physics} 

\usepackage{esvect}
\hyphenation
{
    über-nom-me-nen an-ge-ge-be-nen
    %Pro-to-koll-in-stan-zen
    %Ma-na-ge-ment  Netz-werk-ele-men-ten
    %Netz-werk Netz-werk-re-ser-vie-rung
    %Netz-werk-adap-ter Fein-ju-stier-ung
    %Da-ten-strom-spe-zi-fi-ka-tion Pa-ket-rumpf
    %Kon-troll-in-stanz
}
\newcommand{\LL}{\mathcal{L}}
\newcommand{\SSS}{\mathcal{S}}
\newcommand{\DD}{\mathcal{D}}
\newcommand{\Dd}{{\rm D}}
\newcommand{\val}{\vv{\alpha}}
\newcommand{\cc}{\cdot}
\newcommand{\rk}{\rangle}
\newcommand{\lk}{\langle}
\newcommand{\vx}{\vv{x}}
\newcommand{\vy}{\vv{y}}
\newcommand{\vp}{\vv{p}}
\newcommand{\vr}{\vv{r}}
\newcommand{\xd}{\hat{x}}
\newcommand{\ad}{\hat{a}}
\newcommand{\pd}{\hat{p}}
\newcommand{\df}{\rightarrow}
\newcommand{\la}{\lambda}
\newcommand{\dd}{{\rm d}}
\newcommand{\hilb}{\mathcal{H}}
\newcommand{\ehm}{\mathbbm{1}}
\newcommand{\vB}{\vv{E}}
\newcommand{\vE}{\vv{B}}
\newcommand{\trans}{\mathcal{T}}
\newcommand{\p}{\partial}
\newcommand{\vK}{\vv{K}}
\newcommand{\OO}{\mathcal{O}}
\newcommand{\soll}{\overset{!}{=}}
\newcommand{\D}{\Delta}
\newcommand{\vV}{\vv{V}}
\newcommand{\vJ}{\vv{J}}
\newcommand{\eps}{\epsilon}
\newcommand{\vn}{\vv{\nabla}}
\newcommand{\vw}{\vv{\omega}}
\newcommand{\vektor}[3]{\begin{pmatrix} #1 \\ #2 \\ #3 \end{pmatrix}}
\newcommand{\vektorz}[2]{\begin{pmatrix} #1 \\ #2 \end{pmatrix}}
\newcommand{\Mat}[9]{\begin{pmatrix}#1&#2&#3\\#4&#5&#6\\#7&#8&#9\end{pmatrix}}
\newcommand{\Matz}[4]{\begin{pmatrix}#1&#2\\#3&#4\end{pmatrix}}
\newcommand{\Ld}[1]{\Lambda_{~#1}}
\newcommand{\Lu}[1]{\Lambda^{~#1}}
\newcommand{\dslash}{\slashed{\partial}}
\newcommand{\sub}[1]{~\newline\newline\textbf{#1}\\}
\newcommand{\tr}{{\rm tr}}
\newcommand{\hc}{\text{h.c.}}

%% -----------------------
%% |    Main Document    |
%% -----------------------
\begin{document}
    % Titlepage and ToC
    \FrontMatter

    % coordinates for background border
\newcommand{\diameter}{20}
\newcommand{\xone}{-15}
\newcommand{\xtwo}{160}
\newcommand{\yone}{15}
\newcommand{\ytwo}{-253}




\begin{titlepage}
    % background border
    \begin{tikzpicture}[overlay]
    \draw[color=gray]
            (\xone mm, \yone mm)
      -- (\xtwo mm, \yone mm)
    arc (90:0:\diameter pt)
      -- (\xtwo mm + \diameter pt , \ytwo mm)
        -- (\xone mm + \diameter pt , \ytwo mm)
    arc (270:180:\diameter pt)
        -- (\xone mm, \yone mm);
    \end{tikzpicture}



    % KIT image and sign for faculty of physics
    \begin{textblock}{10}[0,0](4.5,2.5)
        \includegraphics[width=.25\textwidth]{include/kitlogo.pdf}
    \end{textblock}
    \changefont{phv}{m}{n}    % helvetica




    % horizontal line
    \begin{textblock}{10}[0,0](4.2,3.1)
        \begin{tikzpicture}[overlay]
        \draw[color=gray]
                (\xone mm + 5 mm, -12 mm)
          -- (\xtwo mm + \diameter pt - 5 mm, -12 mm);
        \end{tikzpicture}
    \end{textblock}



    % begin of text part
    \changefont{phv}{m}{n}    % helvetica
    \centering



    % thesis topic (en and ge)
    \vspace*{3cm}
    \Huge\thesistopic\\




    % author name and institute
    \vspace*{2cm}
    \Large von\\
    \vspace*{1cm}
    \huge\thesisauthor\\
    \vspace*{1cm}
    \Large Vorlesung gehalten von



    % examiners (Referenten)
    \vspace*{1.5cm}
    \Large
    \thesisreviewerone\\
    



    % working time
    \vspace{1cm}
    \begin{center}
        \large{\thesistimestart}
    \end{center}



    % lowest text blocks concerning the KIT
    \begin{textblock}{10}[0,0](4,16.8)
        \tiny{KIT -- Universität des Landes Baden-Württemberg und nationales %
              Forschungszentrum in der Helmholtz-Gemeinschaft}
    \end{textblock}
    \begin{textblock}{10}[0,0](14,16.75)
        \large{\textbf{www.kit.edu}}
    \end{textblock}
\end{titlepage}

    
    \begingroup \let\clearpage\relax    % in order to avoid listoffigures and
    \tableofcontents                    % listoftables on new pages
    \endgroup



    % Contents
    \MainMatter

\chapter{Introduction}
\section{Quarks and Leptons}
Particles of matter:
\begin{itemize}
	\item electrons ($e^-$) and other leptons are elementary particles. 
	\item protons and neutrons ($|p\rk = |uud\rk$, $|n\rk = |udd\rk$) are combinations of elementary quarks and gluons. The binding energy of the quarks is very large in comparison to the absolute energy of the proton and neutron ($m_pc^2 = 938~\si{MeV}$) if you compare this to the binding energies of Atoms ($\sim 1~\si{Ry}$) and their absolute energies ($\sim 10^9~\si{Ry}$).\\
		Because the proton and the neutron are similar/symmetric in the strong interaction (not in the electroweak interaction though) we can combine them into a isospin dublett $\vektorz{p}{n}$.
\end{itemize}
There are many more particles/boundstates of quarks and gluons for different combination of quarks. Another example are the $\Delta$ baryons. These are spin $\frac{3}{2}$ particles and have masses of $m_\Delta c^2 \approx 1230 ~\si{MeV}$:
\begin{itemize}
	\item $\Delta^-:~|ddd\rk$
	\item $\Delta^0:~|ddu\rk$
	\item $\Delta^+:~|duu\rk$
	\item $\Delta^{++}:~|uuu\rk$
\end{itemize}
Because the $\Delta$ baryons are spin $\frac{3}{2}$ particles all of the quarks spins must be aligned, so the spin wavefunction is symmetric. Also the orbital wavefunction is symmetric for the $\Delta^{++}$ baryon because it consists of thrice the same quark. However, the total wavefunction of the baryons must be antisymmetric because it is a fermion.\\
This is the reason a color charge was introduced to interpret the Dirac statistics correctly and characterize the strong interaction with a new quantum number.\\
Alltogether one can describe the four $\Delta$ baryons in an isospin quartet with $I = \frac{3}{2}$.\\
\\
Another group of particles are the mesons, they consist of one quark and an anti quark. The lightest examples are the pions:
\begin{itemize}
\item $\pi^+:~|u\bar{d}\rk$
\item $\pi^0:~\frac{1}{\sqrt{2}}\left( |u\bar{u}\rk - |d \bar{d}\rk\right)$
\item $\pi^-:~|d\bar{u}\rk$
\end{itemize}
They have masses of $m_\pi c^2 \approx 140~\si{MeV}$ and are spin $0$ particles. Together they form the isospin triplet $I = 1$.\\
\\
Another group of mesons with spin $0$ are the kaons. These have another type of quark, the strange quark. For this new quark a new quantum number (next to isospin) was introduced, the strangeness.\\
We can summarize the kaons and the pions in a meson octett depicted in figure \ref{1}. The kaons have masses of $m_Kc^2 \approx 495~\si{MeV}$. Additionally to the four kaons and the three pions there is an $\eta$ meson with the same strangeness and isospin as the $\pi^0$. It is like the $\pi^0$ but has additional strange quarks: $|\eta\rk = \frac{1}{\sqrt{6}}\left( |u\bar{u}\rk + |d\bar{d}\rk - 2 | s \bar{s}\rk\right)$.\\
\begin{figure}[H]
\centering
\includegraphics[scale=0.1]{include/mesonoctett.pdf}
\caption{meson octett for spin $0$}
\label{1}
\end{figure}

The quarks and leptons are probably the fundamental layer of particles; mesons and baryons are complex bound states described through nuclear physics. The quarks and leptons are described by the dirac equation
\[ \left( i \dslash - \frac{mc}{\hbar} \right)\psi = 0 + \si{interactions}\]
One interaction is for example the electromagnetism: $\p_\mu \df \p_\mu + iq A_\mu$

In table \ref{quarks} and \ref{leptons} all quarks and leptons are summarized with their electric charge and mass. The charge is given in units of elementary charge as $q = Q\cc e$.\\
\begin{minipage}{80mm}
\begin{table}[H]
\centering
\begin{tabular}{r|lr}
quark & $mc^2$ [MeV] & Q \\
\midrule
u & $2.2^{+0.6}_{-0.7}$ & $+\frac{2}{3}$\\
d & $4.7$ & $-\frac{1}{3}$\\
\midrule
c & 1270 & $+\frac{2}{3}$\\
s & 96 & $-\frac{1}{3}$\\
\midrule
t & 173200 & $+\frac{2}{3}$\\
b & 4180 & $-\frac{1}{3}$\\
\bottomrule
\end{tabular}
\caption{quarks}
\label{quarks}
\end{table}
\end{minipage}
\begin{minipage}{80mm}
\begin{table}[H]
\centering
\begin{tabular}{r|lr}
lepton & $mc^2$ [MeV] & Q\\
\midrule
$e^-$ & 0.511 & -1\\
$\mu^-$ & 105.66 & -1 \\
$\tau^-$ & 1777 & -1 \\
\midrule
$\nu_e$ & & 0\\
$\nu_\mu$ & & 0\\
$\nu_\tau$ & &0\\
\bottomrule
\end{tabular}
\caption{leptons}
\label{leptons}
\end{table}
\end{minipage}

It is important, that the down quark is slightly heavier than the up quark; because then the down quark more likely decays into the up quark than vice versa and therefore the proton is much more stable than the neutron. This way also atoms remain stable and charged.\\
\\
These leptons and quarks are all known matter fields save the bosons:
\begin{itemize}
\item Higgs boson $H$
\item $\gamma$, $W^\pm$, $Z$ which are carriers of the electromagnetic and weak force
\item gluon $g$ which is the carrier of the strong force
\end{itemize}
Additionaly it is known, from observing the Higgs coupling, that there are no more generations of quarks which behave similarly to the three existing generations. Additional generations might exist but must behave fundamentally different.

\section{Course Contents}
In this course of theoretical particle physics the following topics will be discussed:
\begin{itemize}
\item theoretical description of interactions of quarks and leptons\\
$\df$ gauge theories (\textit{Eichtheorien})
\item pair production of particles and $\gamma$, $W^\pm$, $Z$, $g$ emission\\
$\df$ changing particle number and content\\
$\df$ quantum field theory (QFT) which is relativistic for particle physics
\item development of pertubation theory for QFT
\item calculation of cross sections and decay rates
\item symmetries: Lorentz invariance and internal symmetries like isospin and color
\end{itemize} 

\section{Natural Units}
In particle physics it is not practical to use the usual units. It is much more practicable to factor out constants like $\eps_0$, $\hbar$, $c$ and $k_B$ such that the remaining quantities have dimensions of energy to a power.\\
The unit of energy will be electron volts (eV). In table \ref{units} some important quantities and their dimensions in natural units are shown.
\begin{table}[H]
\centering
\begin{tabular}{l|lll}
quantity & SI units & natural units & dimension\\
\midrule
velocity & $\tilde{v}$ & $v \cc c$ & $[v] = 1$\\
length & $\tilde{L}$ & $L \cc \hbar c$ & $[L] = 1/\si{MeV}$\\
time & $\tilde{t}$ & $t\cc \hbar$ & $[t] = 1/\si{MeV}$\\
electric field & $\tilde{E}$ & $\frac{1}{\sqrt{\eps_0 (\hbar c)^3}} \vv{E}$ & $[\vv{E}] = \si{MeV}^2$\\
magnetic field & $\tilde{B}$ & $\frac{1}{\sqrt{\eps_0 c^2(\hbar c)^3}} \vv{B}$ &  $[\vv{B}] = \si{MeV}^2$\\
\bottomrule
\end{tabular}
\caption{natural units}
\label{units}
\end{table}
An example of the simplification is the Hamiltionan for radiation:
\[ H_{rad} = \frac{\eps_0}{2} \int\dd^3 \tilde{\vx} \left[ \tilde{\vE}^2 + c^2 \tilde{\vB}^2\right] \df \frac{1}{2} \int\dd^3 \vx \left[\vE^2 + \vB^2\right]\]

Another useful thing are translations from the normal system to the natural units and vice versa. For example:
\begin{itemize}
\item $\hbar c = 197~\si{MeV fm}$
\item $\frac{1}{\si{GeV}^2} = \frac{3.89\cc 10^{-4}~\si{b}}{(\hbar c)^2}$ where a barn is $10^{-28}~\si{m^2}$
\item $\tilde{e} = 1.6\cc 10^{-19} ~\si{C} ~~\df~~ e = \frac{\tilde{e}}{\sqrt{\eps_0 \hbar c}} = \sqrt{ 4 \pi \alpha} = 0.3028$
\end{itemize}

\subsection{Klein Gordon and Dirac Equations in Natural Units}
The Klein-Gordon equation in SI units is given as
\[
\left[ \tilde{\square} + \left( \frac{mc}{\hbar}\right)^2 \right] \phi(\tilde{x}) = 0
\]
where $x$ is a four vector $\tilde{x}^\mu = ( c \tilde{t}, \tilde{\vx}) = \hbar c (t , \vx)$. Also the d'Alembert operator in SI units is given as
\[ 
\tilde{\square} = \frac{1}{c^2} \frac{\p^2}{\p \tilde{t}^2} - \tilde{\vn}^2 = \frac{1}{(\hbar c)^2} \square = \frac{1}{(\hbar c)^2} \left( \frac{\p^2}{\p t^2} - \vn^2 \right)
\]
So in natural units the equation simplifies to 
\[ ( \square + m^2 ) \phi(x) = 0\]
\newline
Similarly the Dirac equation simplifies when using natural units
\[
\left( i \gamma^\mu \tilde{\p}_\mu - \frac{m c}{\hbar}\right) \psi(\tilde{x}) = 0 ~~\df~~ (i \gamma^\mu \p_\mu - m ) \psi(x) = 0
\]

\section{Lagrange Density and Equations of Motion}

\subsection{Lagrangian Field Theory}
First we take a look at a classical point particle. Its trajectory is given by $x_i(t)$ for $i = 1,2,3$.\\
For this particle we can define an action
\[ \SSS( [x_i], t_1, t_2 ) = \int_{t_1}^{t_2} \dd t \left(\frac{1}{2} m \left( \sum_i \frac{\dd x_i}{\dd t} \right)^2 - V(x_i(t)) \right)
\]
The action is a functional of the trajectory. Now we can find an extremum of $\SSS$ for the classical path by adding a inifinitesimal variation $\delta x_i$ to the trajectory: $x_i(t) + \delta x_i(t)$. Then, the extremal condition is given by $\Delta S = 0$ where $\Delta S$ is given by
\[ \Delta \SSS = \SSS( [x_i + \delta x_i]) - \SSS([x_i]) = 0\]
where the boundary condition is set, so the variation $\delta x_i$ vanishes at the endpoints
\[ \delta x_i(t_1) = \delta x_i(t_2) = 0\]
Calculating the action for the changed trajectory leads to
\[ \SSS( [x_i + \delta x_i]) = \int_{t_1}^{t_2} \dd t \left( \frac{1}{2} m \left( \frac{\dd x_i}{\dd t} + \frac{\dd (\delta x_i)}{\dd t} \right)^2 - V(x_i + \delta x_i) \right)
\]
with
\[  \left( \frac{\dd x_i}{\dd t} + \frac{ \dd ( \delta x_i)}{\dd t} \right) = \left( \frac{ \dd x_i}{\dd t} \right)^2 + 2 \frac{ \dd x_i }{\dd t} \frac{ \dd (\delta x_i)}{\dd t} = \left( \frac{\dd x_i}{\dd t} \right)^2 + 2 \frac{\dd}{\dd t} \left( \frac{\dd x_i}{\dd t} \delta x_i \right) - 2 \frac{ \dd ^2 x_i}{\dd t^2} \delta x_i\]
where second order terms in $\delta x_i$ were neglected. The first term also appears in the action for the original trajectory and the total derivative in the second term cancels the integral. Therefore
\[ 
\SSS( [x_i + \delta x_i]) = S([x_i]) + \left[ \int_{t_1}^{t_2} \dd t \sum_i \left( -m \frac{\dd ^2 x_i}{\dd t^2} - \frac{\p V}{\p x_i} \right) \delta x_i \right] + \left. m \sum_i \frac{\dd x_i}{\dd t} \delta x_i \right|_{t_1}^{t_2}
\]
Because the last term vanishes due to the boundary conditions and $\delta x_i$ is chosen arbitrarily $\Delta \SSS$ can only vanish if the term inside the integral is zero. Therefore
\[ 
m \frac{\dd x_i }{\dd t} = \frac{\p V}{\p x_i}
\]
This equation of motion is true for all $\delta x_i$.
\newline\newline
\textbf{Symmetries}\\
Lets assume the system has a rotational invariance $V = V(r)$ with $r = \sqrt{ \sum_i x_i^2}$. If a transformation $O_{ij}$ orthogonal to the rotational invariance is applied to the trajectory $x_j$ the action remains the same
\[ \SSS[ \sum_j O_{ij} x_j (t) ] = S[x_j(t)]\]
Also the equation of motions is invariant
\[ m \frac{ \dd^2 ( O_{ij} x_j)}{\dd t^2} = - \frac{ O_{ij} x_j}{r} \frac{\dd V}{\dd r} \]

\subsection{Field Theory Lagrangian}
In quantum field theory the action is given by
\[ \SSS([\phi_r]) = \int \dd ^4 x~\mathcal{L}(\phi_r, \p_\mu \phi_r)\]
with $\phi_r = \phi_r(\vx, t)$. $\LL$ is the Lagrange density. It is not directly dependant on $x$, because it should be invariant in the whole four dimensional space. The integral here is over all four dimensions because time and space are treated equally in field theory.\\
There are some requirements to the Lagrange density:
\begin{enumerate}
\item $\LL$ is local - there are no connections or interactions between two arbitrary space points. Also there can not be any instantaneus interaction of two spacepoints because information travels at finite speeds.
\item $\LL$ is real - this is necessary to conserve probability
\item $\LL$ is Lorentz invariant - $x'^\mu = \Lambda^\mu_{~\nu}x^\nu ~\df~ \dd^4 x' = (\det \Lambda) \dd^4 x = \dd ^4 x$. Therefore also the action is Lorentz invariant.
\item there is no need for derivatives higher than the first, this is implied by causality (?)
\end{enumerate}
In natural units the action is dimensionless (whereas in SI units it has the same unit as $\hbar$). Also $\dd^4 x$ has units of $\si{\frac{1}{GeV^4}}$ in natural units, therefore $\LL$ has to have units of $\si{GeV}^4$
\newline\newline
\textbf{Extremal of Action}\\
Same as before we can calculate the extremal of the action $\SSS$ for variations $\delta \phi_r$. Here the boundary condition has to be $\delta \phi_r(x) = 0$ for $x \in \p \Omega$ where $\p \Omega$ is the surface of the integrated space.\\
It follows
\[ 0 = \Delta \SSS = \int_\Omega \dd ^4 x \sum_r \left[ \frac{\p \LL}{\p \phi_r} \delta \phi_r + \frac{ \p \LL}{\p (\p_\mu \phi_r)} \p_\mu(\delta \phi_r) \right]\]

In the second term the equality of $\delta (\p _\mu \phi_r) = \p_\mu(\delta \phi_r)$ was used. Also we can rewrite the partial derivative in the second term to an absolute derivative
\[
\frac{\p \LL}{\p(\p_\mu \phi_r)} = \p_\mu \left( \frac{\p \LL}{\p ( \p_\mu \phi_r)} \delta\phi_r\right) - \delta \phi_r \p_\mu \frac{\p \LL}{\p(\p_\mu \phi_r)}
\]
Therefore
\[
\Delta \SSS = \int_\Omega \dd ^4 x \left[ \sum_r \delta \phi_r \left( \frac{\p \LL}{\p \phi_r} - \p_\mu \frac{\p \LL}{\p(\p_\mu \phi_r)}\right) + \p_\mu \left( \sum_r \delta \phi_r \frac{\p \LL}{\p(\p_\mu \phi_r)} \right)\right]
\]
The last term is rewritable into a surface integral via Gauss' theorem, therefore it vanishes due to the boundary conditions. Similar to the classical approach $\delta \phi_r$ can be chosen arbitrarily and therefore the action only vanishes if the first term is equal to zero
\[ 
\frac{\p \LL}{\p \phi_r} - \p_\mu \frac{\p \LL}{\p(\p_\mu \phi_r)} = 0
\]
These are the Euler-Lagrange equations.
\sub{Example}
Consider a scalar field $\phi(x)$. The Lagrange density is given by
\[ \LL = \frac{1}{2} (\p_\alpha \phi)(\p^\alpha \phi) - V(\phi) = \frac{1}{2} \left[ (\p_0\phi)^2 - \sum_i (\p_i \phi)^2\right]-V(\phi)\]
So the equation of motion calculates to
\[ \frac{\p \LL}{\p \phi} = - V'(\phi) = \p_\mu \left( \frac{\p \LL}{\p(\p_\mu \phi)}\right) = \p_0 \p_0 \phi - \vn(\vn \phi) = \square \phi\]
\[\df \square \phi + V'(\phi) = 0\]
Different potentials then lead to different equations of motion
\begin{itemize}
\item $V(\phi) = \frac{m^2}{2}\phi^2 ~\df~ V' = m^2\phi$ leads to $\square\phi + m^2 \phi = 0$
which is the Klein-Gordon equation. Its Lagrange density is
\[ \LL = \frac{1}{2} (\p_\mu \phi)(\p^\mu \phi) - \frac{m^2}{2}\phi^2\]
\item $V(\phi) = \frac{m^2}{2}\phi^2 + \frac{\lambda}{4}\phi^4$ leads to $\square \phi + m^2 \phi + \lambda \phi^3  = 0$
\item $V(\phi) = A \cos \frac{\phi}{M}$ leads to $\square\phi - \frac{A}{M} \sin \frac{\phi}{M}$
which is called the sine-Gordon equation.
\end{itemize}

\subsection{Dirac Lagrangian}
For spin $\frac{1}{2}$ particles the Lagrangian is connected to the Dirac equation. It is given by
\[ \LL = \bar{\psi}(x)(i \dslash - m) \psi(x),~~~\bar{\psi} = \psi^\dagger \gamma^0 = (\psi^*)^T \gamma^0\]
where $\psi$ has four complex and eight real components.\\
The components of $\psi$ and $\bar{\psi}$ are treated as independant fields. This leads to the following equations of motion
\[ \frac{\p \LL}{\p \bar{\psi}} = ( i \dslash - m)\psi = \p_\mu \left( \frac{\p \LL}{\p(\p_\mu \bar{\psi})}\right) = 0~~~~\df ~~ (i \dslash - m) \psi = 0\]
\[ \frac{\p\LL}{\p\psi} = -m \bar{\psi} = \p_\mu\left( \frac{\p\LL}{\p(\p_\mu \psi)}\right) = i \p_\mu(\bar{\psi}\gamma^\mu)~~~~\df ~~ i \p_\mu(\bar{\psi}\gamma^\mu) + m \bar{\psi} = 0\]


\[
\LL = \bar{\psi} \gamma^\mu \p_\mu \psi + i q \bar{\psi} \gamma^\mu A_\mu \psi + m \bar{\psi}\psi + \frac{1}{16 \pi} F_{\mu\nu}F^{\mu\nu}
\] 

\chapter{Groups and Symmetries}
\section{Representation of Groups}
A group is an object of the form $G = \{ g_i | i = 1,\ldots\}$ where $g_i$ are the elements of the group. In the group a multiplication exists so, that $g_1 g_2 = g_3 \in G$. This also has the following traits:
\begin{itemize}
\item it has to be associative: $g_1(g_2g_3) = (g_1g_2)g_3$
\item there is an identity element $e \equiv 1$ so that $g\cc e = e \cc g = g ~~\forall g \in G$
\item $\forall g \in G$ there is an inverse $g^{-1} \in G$: $gg^{-1} = g^{-1}g = e$
\end{itemize}
The representation of $G$ is a mapping $r: G \df \mathbbm{C}^{(n,n)}$ where $r(g_i) = M_i$ is a $n\times n$-matrix.\\
With this representation the multiplication rules are being preserved: $r(g_1g_2) = r(g_1)r(g_2)$ or $M_3 = M_1M_2$. Also $r(e) = \ehm_n$ and $r(g^{-1}g) = MM^{-1} = \ehm$.\\
A reducable representation is a representation such that a single unitary $n\times n$ matrix $U$ exists such that
\[ U M_i U^{-1} = \Matz{M_i'}{0}{0}{M_i''}\]
So instead of working with $M_i$ we could have worked with the block diagonal matrices $M_i'$ and $M_i''$.\\
An irreducable representation then is a representation where no unitary matrix exists that splits the matrix $M_i$ into block diagonal matrices.\\
For finite groups $G = \{ g_i | i = 1, \ldots, n \}$ the dimensions $d_r$ of all the irreducible representations are bounded by 
\[n = \sum_r d_r^2\]
Therefore an infinite group has infinte number of different finite dimensional irreducible representations.
\section{Lie Groups and Lie Algebra}
Lie groups are a special case of groups. They are parametrizised by $G = \{ U(\theta)| \theta = ( \theta_1, \ldots, \theta_n) \in \mathbbm{R}^n\}$ with $U(0) = e$. \\
$U(\theta)$ is analytic in all its components; it is infinitely differentiable.\\
The simplest example of a Lie group is the three dimensional rotation group $SO(3)$ with the rotation matrices $R(\phi,\psi,\theta)$. The three rotations are\\
the rotation around the $z$-axis:
\[ R_z(\theta) = \Mat{\cos\theta}{\sin\theta}{0}{- \sin\theta}{\cos\theta}{0}{0}{0}{1}\]
the rotation around the $y$-axis:
\[ R_y(\psi) = \Mat{\cos\psi}{0}{-\sin\psi}{0}{1}{0}{\sin\psi}{0}{\cos\psi}\]
the rotation around the $x$-axis:
\[ R_x(\phi) = \Mat{1}{0}{0}{0}{\cos\phi}{\sin\psi}{0}{-\sin\phi}{\cos\phi}\]


\sub{Lie algebra}
As the elements of the Lie group are infinitely differentiable we can apply the definition of the Taylor expansion onto an element of the group. This leads to the generators of the group
\[L_a \equiv \frac{1}{i} \left.\frac{\p U}{\p \phi_a} \right|_{\theta = 0}\]
These generators completely describe the groups properties.\\
The Taylor expansion also leads to infinitesimal transformations:
\[ U(\theta) = 1 + i \sum_a \theta_a L_a + \ldots\]
In summation convention this also can be written as $\sum_a \theta_a L_a = \theta_a L_a = \theta\cc L$.\\
As one general group property is the formation of the $1$-element: $U(\theta)U^{-1}(\theta) = 1$ this also has to be applicable for inifitiesimal transformations. This leads to
\[ G \ni U(\theta)U(\psi)U^{-1}(\theta)U^{-1}(\psi) \neq 1\]
which is not neccesarily the $1$-element for non commuting groups. The Taylor expanison of this expression leads to
\[ 1 + i^2 \theta_a\psi_b ( L_aL_b + L_aL_b - L_aL_b - L_bL_a) + \ldots = 1 - \theta_a \psi_b [L_a, L_b]+\ldots = 1 + i \sum_c \theta_a \psi_b ( - L_c f^{abc})\]
In the first step the only terms of first order are either linear in $\psi$ or $\theta$ so they cancel. In the second order all terms quadratic in $\theta$ or $\psi$ also cancel and the only things left are mixed terms. In the second step the definition of the commutator was applied and in the last step we used, that the resulting element had to be in the group $G$ again, so it must be able to be written as a taylor expansion again. This therefore leads to the identity
\[ [ L_a, L_b] = -i L_c f^{abc}\]
Where $f^{abc}$ are group specific structure constants.\\
For the $SO(3)$ group this is for example the known $\eps$-tensor. The generators are the components of angular momentum:
\[L_x = \Mat{}{}{}{}{}{-i}{}{i}{}, ~~ L_y = \Mat{}{}{i}{}{}{}{-i}{}{}, ~~L_z = \Mat{}{-i}{}{i}{}{}{}{}{}\]
The identity here is $[L_x,L_y] = iL_z$.\\
These generators form the basis of the Lie algebra.\\
\\
Another trait of the group can be shown for a finite group with $\theta = (\theta_1, \ldots, \theta_n)$. If we use $\eps = \frac{1}{N}\theta$ then for high $N$, $\eps$ becomes small. So a Taylor expansion can be applied:
\[ U(\eps) = 1 + i \eps_a L_a = 1 + i \frac{\theta_a L_a}{N}\]
So the original $\theta$ can be written as the following
\[ U(\theta) = U(\eps)^N \df \lim_{N \df \infty} U(\eps) ^N = \lim_{N\df\infty} \left( 1 + i \frac{\theta_aL_a}{N}\right)^N = e^{i\theta_aL_a} = \sum_{k = 0}^{\infty}\frac{1}{k!}(i\theta_aL_a)^k\]
The $i$ was introduced in the definition of the generators, so the generators would be hermitian operators. If we look at $U(\theta)$, which is unitary it follows
\[ U(\theta) = e^{i\theta L} ~~\df~~ U^{-1}(\theta) = e^{-i\theta L}\]
and thus
\[ e^{-i\theta_aL_a^\dagger} = U^\dagger(\theta) = U^{-1}(\theta) = e^{-i\theta_aL_a}\]
From this we can see, that $L_a$ and $L_a^\dagger$ must be the same.


\sub{Special unitary groups}
Another group of Lie groups are the special unitary groups $SU(N) = \{ U \in\mathbbm{C}^{(N\times N)} | U^{-1} = U^\dagger, \det(U) = 1\}$. Its group elements are unitary and have a determinante of one. The number of generators $L_a$ can be expressed formally for every $N$:\\
$L_a$ has to be a hermitian $N\times N$ matrix with trace $\tr(L_a) = 0$. This follows from
\[ \det(U) = \det \left( e^{i \theta_a L_a}\right) = e^{i \tr(\theta_a L_a)} \soll 1 ~~\df ~~ \tr(L_a) = 0\]
Now because there are $N^2$ matrices but one of them neccesarily is $\ehm$ which is not traceless, there are always $N^2-1$ generators in $SU(N)$.\\
For $N = 2$ the generators are $L_a = \frac{\sigma_a}{2}$ where $\sigma_a$ are the Pauli matrices.\\
For $N = 3$ the generators are the eight Gell-Mann matrices. They have the following forms
\begin{align*} \lambda_1 &= \Mat{}{1}{}{1}{}{}{}{}{},~~~\lambda_2 = \Mat{}{-i}{}{i}{}{}{}{}{},~~~\lambda_3 = \Mat{1}{}{}{}{-1}{}{}{}{}\\
\lambda_4 &= \Mat{}{}{1}{}{}{}{1}{}{}, ~~~\lambda_5 = \Mat{}{}{-i}{}{}{}{i}{}{}\\
\lambda_6 &= \Mat{}{}{}{}{}{1}{}{1}{}, ~~~\lambda_7 = \Mat{}{}{}{}{}{-i}{}{i}{}, ~~~\lambda_8 = \frac{1}{\sqrt{3}} \Mat{1}{}{}{}{1}{}{}{}{-2}
\end{align*}
These matrices have normalization conditions which also apply for abitrary $N$:
\[ \tr(\lambda^a \lambda^b) = 2\delta^{ab}, ~~~ L_a = \frac{\lambda_a}{2}\]


~\newline\newline\sub{Rank of the Lie algebra}
The Lie algebra also has a rank. The rank is defined as the maximum number of commuting generators in the algebra. $SU(N)$ has $N-1$ diagonal generators, which naturally all commute. So the rank of an arbitraty $SU(N)$ algebra is always $N-1$.\\
These $N-1$ eigenvalues of these $L_a$ then specify the basis states in a $SU(N)$ multiplet.\\
For example for $N = 2$ the rank is one, so it has only one invariant, which is $\vv{J}^2$ so that $[\vv{J}^2, J_i] = 0$.\\
Generally, if we have any polynomial such that $C = \eta_{ab} L_a L_b + \eta_{abc} L_aL_bL_c + \ldots$ that commutes with all generators, $[C,L_a] = 0 ~\forall a$ then it is called a Casimir invariant.\\
For $SU(2)$ the only Casimir invariant is $\vv{J}^2$, for $SU(3)$ there are two different Casimir invariants.\\
For any $SU(N)$ there is a Casimir invariant $C_2 = L_aL_a$ which is also referred to as the quadratic Casimir of the $SU(N)$ group.\\
Furhtermore, the eigenvalues of all independant Casimir invariants do specify an irreducible representation.


\sub{Adjoint irreducible representation}
If we take a look at the Jacobi identity
\[ [A,[B,C]] + [B,[C,A]] + [C,[A,B]] = 0\]
and take $A = L_a$, $B= L_b$, $C=L_j$ we get
\[ [A,B] = i f_{abm} L_m ~~~\df~~~ [C,[A,B]] = i f_{abm}[L_j,L_m] = if_{abm}if_{jmc} L_c\]
from this follows the relationship
\[ (-if_{aim})(-if_{bmj}) - (-if_{bim})(-if_{amj}) = if_{abc}(-if_{aij})\]
If we now write $if_{abc} = (F^a)_{bc}$ as a matrix element of a matrix $F^a$ we get
\[ (F^a)_{im} (F^b)_{mj} - (F^b)_{im}(F^a)_{mj} = i f_{abc} (F^c)_{ij}\]
due to the summation convention this is equivalent to
\[ (F^aF^b)_{ij} - (F^bF^a)_{ij} = if_{abc}(F^c)_{ij}\]
So this in it self satisfies the commutation relation of the Lie algebra:
\[ [F^a, F^b] = i f^{abc} F^c\]
So these objects for an irreducible representation of the Lie algebra called the adjoint irreducible representation.\\
\\
In summary, for any $SU(N)$ group there are three irreducible representations:
\begin{itemize}
\item the trivial representation: $U(\theta) = 1$
\item the fundamental representation: $U(\theta) = \exp\left( i \frac{\lambda^a}{2} \theta^a\right)$
\item the adjoint representation: $U(\theta) = \exp \left( i F^a \theta ^a\right)$
\end{itemize}


\section{The Lorentz Group and Relativistic Invariance}
\sub{Lorentz transformations}
Consider two inertial frames with common origin at $t = 0$ and moving with relative velocity $v$ along the $x$-axis. Lets assume that in the first frame an event is taking place at $x^\mu = (t, \vx)^T$ and is seen at $x'^\mu = (t', \vx')^T$ in the primed frame. The transformation is as follows
\[ x'^\mu = \vektorz{t'}{\vx'} = \begin{pmatrix} \gamma t - \gamma v t \\  \gamma v t + \gamma x\\ y \\z\end{pmatrix} = \begin{pmatrix} \gamma & - \gamma v & 0 & 0 \\ -\gamma v & \gamma & 0 & 0 \\ 0 & 0 & 1 & 0 \\ 0 & 0 & 0 & 1\end{pmatrix}\cc x^\mu\]
The transformation matrix is called $\Lambda$ so that $x'^\mu = \Lambda^\mu_{~\alpha} x^\alpha \equiv L x$.\\
\\A Lorentz transformation now is any linear transformation $\Lambda$ which keeps the relative length invariant:
\[ s^2 = x'^\mu x'^\nu g_{\mu\nu} = \Lambda^\mu_{~\alpha} \Lambda^\nu_{~\beta}x^\alpha x^\beta g_{\mu\nu}\soll x^\alpha x^\beta g_{\alpha\beta}\]
This must hold for all possible $x^\mu$.\\
From this follows
\[ \Lambda^\mu_{~\alpha} \Lambda^\nu_{~\beta} g_{\mu\nu} = g_{\alpha\beta}\]
If we now regard $x^\mu$ as a column vector $x$ and $\Lambda^\mu_{~\nu}$ as the elements of some $4\times 4$ matrix $L$ then follows
\[x' = L \cc x, ~~~~ s^2 = x^Tg x\]
Then the condition on the $\Lambda$'s reads
\[g^\alpha_{~\beta} = \Lambda_\mu^{~\alpha} g^\mu_{~\nu} \Lambda^\nu_{~\beta}~~\df~~ g_{\alpha\beta} = \Lambda^\mu_{~\alpha} g_{\mu\nu}\Lambda^\nu_{~\beta} = (L^TgL)_{\alpha\beta}\]
Now, because $g$ is symmetric ($g_{\alpha\beta} = g_{\beta\alpha}$) also $L^TgL$ must be symmetric. Therefore there are only ten conditions on $L$ (ten independent elements).\\
If we now take the determinante of that expression we find
\[\det(g) = \det(L^T)\det(g) \det(L)~~~\df~~~ \det(L) = \pm 1\]
and with the Jacobi determinant follows
\[ \int \dd ^4 x' = \int \dd ^4 x| \det(L) | = \int \dd ^4 x\]
We call Lorentz transformations with $\det(L) = 1$ proper Lorentz transformations and Lorentz transformations with $\det(L) = -1$ improper.\\
Also, if we take $\alpha = \beta = 0$ in the equation, we find 
\[g_{00} = 1 = \Lambda^{\mu}_{~0} g_{\mu\nu} \Lambda^\nu_{~0} = (\Lambda^0_{~0})^2 - (\Lambda^i_{~0})^2\]
So necessarily $|\Lambda^0_{~0}| \geq 1$.\\
We then call Lorentz transformations with $\Lambda^0_{~0} \geq 1$ orthochronous and Lorentz transformations with $\Lambda^0_{~0} \leq 1$ non orthochronous.

\sub{Lorentz group}
All Lorentz transformations form a group, the Lorentz group. It has the following attributes
\begin{itemize}
\item the product of two Lorentz transformations is again a Lorentz transformation
\item the inverse exists. In the example of the beginning it would replace $v \df -v$.
\item there is a unitary element ($L = \ehm_4$)
\end{itemize}
Because a product of Lorentz transformations is again a Lorentz transformation we can reduce the types of different transformations to four:
\begin{enumerate}
\item time inversions: $x'^0 = - x^0, x'^i = x^i$ (non orthochronous, improper)
\item parity transformation: $x'^0 = x^0, x'^i = - x^i$ (orthochronous, improper)
\item rotations: $x'^0 = x^0, x'^i = a^{ij} x^j$ with a rotation matrix $a^{ij}$, $L = \Matz{1}{0}{0}{a}$ where $a$ is the $3\times 3$ rotation matrix with $\det(a) = 1$. (orthochronous, proper)
\item boosts (here in one direction): $x'^0 = x^0 \cosh \eta + x^1 \sinh \eta$, $x'^1 = x^0 \sinh\eta x^1 \cosh\eta$, $x'^{(2,3)} = x^{(2,3)}$ where $\eta$ is the rapidity. (orthochronous, proper)
\end{enumerate}
The number of possible rotations is three (for example the Euler angles), the number of boosts also is three (for example the three boost directions or two angles plus $v$). Therefore, any proper, orthochronous Lorentz transformation can be described by six real parameters. The time inversion and parity transformation do not have infinitesimal representations, because they are discrete transformations.

\sub{Inifitesimal generators of proper and orthochronous Lorentz transformations}
For any adequate description of the Lorentz group we need to study infinitesimal generators. Therefore we consider the infinitesimal Lorentz transformation:
\[ \Lambda^\mu_{~\nu} = \delta^\mu_{~\nu} + \eps^\mu_{~\nu} \equiv g^\mu_{~\nu} + \eps^\mu_{~\nu}\]
Now, the condition $g = L^T g L$ can be rewritten as 
\[ \delta^\alpha_{~\beta} = g^\alpha_{~\beta} = \Lambda_\mu^{~\alpha} g^\mu_{~\nu}\Lambda^\nu_{~\beta} = \Lambda_\mu^{~\alpha}\Lambda^\mu_{~\beta} = ( g_\mu^{~\alpha} + \eps_\mu^{~\alpha})(g^\mu_{~\beta} + \eps^\mu_{~\beta}) = g_\beta^{~\alpha} \eps_\beta^{~\alpha} + \eps^\alpha_{~\beta} + \OO(\eps^2)\]
From there follows $\eps_\beta^{~\alpha} + \eps^\alpha_{~\beta} = 0$, for example $\eps_{\alpha\beta} = - \eps_{\beta\alpha}$. So this is antisymmetric with six independant real elements.\\
\\
We now introduce $L_{\mu\nu} = i ( x_\mu \p_\mu - x_\nu\p_\mu)$ with $\p_\mu = \left( \frac{\p}{\p t}, \vn\right)^T$ as a generalization of the angular momentum operator $J^i =i \eps^{ijk} x_j \p_k$. It comes as as surprise, that these $L_{\mu\nu}$ exactly are the generators of the infinitesimal Lorentz transformation:
\[J^i = i \eps^{ijk} x_j \p _k = \frac{1}{2}\eps^{ijk} L_{jk}\]
With $\delta x^\mu = \Lambda^\mu_{~\nu} x^\nu - x^\mu = \eps^\mu_{~\nu} x^\nu$ it follows
\[ \frac{i}{2} e^{\rho\sigma} L_{\rho\sigma} x^\mu = \frac{i}{2} i \eps^{\rho\sigma} ( x_\rho g_\sigma^{~\mu}- x_\sigma g_\rho^{~\mu}) = - \frac{1}{2} (e^{\rho\mu} x_\rho - \eps^{\mu\sigma}x_\sigma )= \eps^{\mu\sigma}x_\sigma = \delta x^\mu\]
where we used $\eps^{\rho\mu} = - \eps^{\mu\rho}$. Therefore
\[ \frac{i}{2}\eps^{\rho\sigma} L_{\rho\sigma} x^\mu = \eps^\mu_{~\nu} x^\nu\]
So the $L_{\rho\sigma}$ are indeed the generators of rotations in Minkovski space, explicitly, the $SO(3,1)$. The Lie algebra is
\[ [L_{\mu\nu}, L_{\rho\sigma}] = i ( g_{\nu\rho} L_{\mu\sigma} - g_{\mu\rho}L_{\nu\sigma} - g_{\nu\sigma}L_{\mu\rho} + g_{\mu\sigma}L_{\nu\rho})\]
\newline
As a generalization, $L_{\mu\nu}$ is analogous to orbital angular momentum. We can add a spin term, which of course commutes with $L$ and forms some algebra similar to the $L$'s among themselves.\\
The most general representation of $SO(3,1)$ generators is by $M_{\mu\nu} \equiv i(x_\mu \p_\nu - x_\nu\p_\mu) + S_{\mu\nu}$. The $M_{\mu\nu}$ space components, or more familiarly, the $J^i = \frac{1}{2} \eps^{ijk} M_{jk}$ components are the generators of inifinetismal rotations with $[J_i, J_j] = i \eps^{ijk} J^k$ (the commutation relation from $SO(2)$).\\
The $M^{0i} \equiv K^i$ are space time components and generate the boosts.\\
The commutation relation of these $K^i$ and the known $J^i$ are
\[ [K^i, K^j] = - \eps^{ijk} J^k, ~~~~ [J^i, K^j] = i \eps^{ijk} K^k\]
These are much simpler than the commutation relation of the $L_{\mu\nu}$.\\
Also, there usually is a trick to separate two $SO(2)$'s from each other by taking the linear combinations
\[N^i \equiv \frac{1}{2}(J^i + i K^i), ~~~~~ N^{i\dagger} \equiv \frac{1}{2} ( J^i - i K^i)\]
For these we find the commutation relations
\[ [N^i, N^{i\dagger}] = 0, ~~~ [N^i, N^j] = i \eps^{ijk} N^k, ~~~ [N^{i\dagger}, N^{j\dagger}] = i \eps^{ijk} N^{k\dagger}\]
Because the first one is zero, $N^\dagger$ and $N$ are decoupled.\\
Finally, we find, that the finite dimensional representations of the restriced Lorentz group is characterized by $(n,m)$ where $n,m = 0,  \frac{1}{2}, 1, \ldots$ and are given by the eigenvalues of the Casimir operators:
\[N^iN^i|n,n_3\rk = n(n+1)|n,n_3\rk, ~~~~ N^{i\dagger}N^{i\dagger}|m,m_3\rk = m(m+1)|m,m_3\rk\]
These $m$ and $n$ are related by parity: $J_i \overset{P}{\df} J_i$, $K_i \overset{P}{\df} - K_i$. This follows, because $J_i$ has two spacial directions and $K_i$ only has one. Now, the parity transformation changes the sign of the spacial components, so for $J_i$ this cancels. The $N$ transform as
\[ N^i \overset{P}{\df} N^{i\dagger}, ~~~ \df ~~~ (n,m) \overset{P}{\df} (m,n)\]
Hence $n$ and $m$ are related. Additionaly we can identify the spin of the representation as $n+m$, this follows, because $J^i = N^i + N^{i\dagger}$ (?).\\
Some special cases are for example:
\begin{itemize}
\item The scalar representation with spin 0 is $(0,0)$.
\item $\left(\frac{1}{2}, 0\right)$ and $\left( 0, \frac{1}{2}\right)$ are representations of spin $\frac{1}{2}$ where the first one represents right handed spinors and the second one left handed spinors. Both are Weyl spinors.
\item Combining $\left(0, \frac{1}{2}\right)\oplus \left(\frac{1}{2}, 0 \right)$ as a direct sum this leads to Dirac spinors. This is for example needed, when a parity invariant theory is wanted.
\end{itemize}
With $J$ and $K$ we can describe finite Lorentz transformations by exponenciating the infinitesimal ones in a non covariant form:
\[ e^{-i(\vv{\omega}\vv{J} + \vv{\nu}\vv{K})}\]
Here $\vv{\omega}$ is the direction of the rotation axis and $|\vv{\omega}|$ is the rotation angle. The direction of the boost is given by $\vv{\nu}$ and $|\vv{\nu}|$ is the rapidity of the boost.\newline
\sub{Poincaré group}
The Poincaré group is obtaioned from the Lorentz group by adding translations $x^\mu \df x'^\mu = x^\mu + a^\mu$ where $a^\mu$ is a constant 4 vector. The most general translation is
\[ x^\mu \df x'^\mu = \Lambda^\mu_{~\nu}x^\nu + a^\mu\]
This has ten real parameters, so we need four additional generators. These will be the momentum operators.\\
Infinitesimally the difference between the translated and untranslated vector can be expressed as $\delta x^\mu = \eps^\mu = - \eps^\rho P_\rho x^\mu$ with $P_\rho = i \p_\rho$.\\
The commutation relations of the momentum operator are
\[ [P_\mu, P_\nu] = 0, ~~~[M_{\mu\nu}, P_\rho] = -i g_{\mu\nu}P_\nu + ig_{\nu\rho}P_\mu\]


\section{Transformation of Fields under Lorentz Transformations}
Lets take a look at a vector field $A_\mu(x)$ (for example the vector potential of the electromagnetic field). An observer in the primed frame describes the same physical situation by another field $A'_\mu(x)$. The two observed fields are of course related, for $x'_\mu = \Lambda_\mu^{~\nu}x_\nu$ it follows
\[A'_\mu = \Lambda_\mu^{~\nu}A_\nu(x)\]
This relation is, what defines $A_\mu$ as a vectorfield. We can generalize this:\\
Let $x_\mu$ and $x'_\mu = \Lambda_\mu^{~\nu} x_\nu$. This should describe some point $P$ in two inertial frames. Let $\Lambda_A^{~B}$ ($A,B = 1, \ldots, d$) be the corresponding $d$-dimensional representation matrixes of an irreducible representation of $SO(3,1)$. A $d$-component field $f_A(x)$ and $f'_A(x')$ in the two transformations is said to transform under this irreducible representation if $f'_A(x') = \Lambda_A^{~B} f_B(x)$. In the infinitesimal version this is $\Lambda_A^{~B} = \delta_A^{~B} - \frac{i}{2} \eps^{\alpha \beta}(S_{\alpha\beta})_A^{~B}$.\\
Here are some examples
\begin{table}[H]
\begin{tabular}{lll}
irr. rep. of $SO(3,1)$ & field & transf. property\\
\midrule
$\left(0, 0\right)$, scalar & $\phi(x)$ & $\phi'(x') = \phi(x)$\\
$\left(\frac{1}{2}, \frac{1}{2}\right)$, vector & $V^\mu(x)$ & $V'^\mu(x') = \Lambda^\mu_{~\nu}V^\nu(x)$\\
$\left(0,\frac{1}{2}\right)$, left handed Weyl spinor & $\psi_L(x)$ & $\psi'_L(x') = \Lambda_L \psi_L(x)$\\
$\left(\frac{1}{2}, 0 \right)$, right handed Weyl spinor & $\psi_R(x)$ & $\psi'_R(x') = \Lambda_R \psi_R(x)$\\
\end{tabular}
\end{table}
The transformation matrices $\Lambda_R$ and $\Lambda_L$ are given by:
\[\Lambda_{L,R} = e^{-i(\vv{\omega}\vv{J} + \vv{\nu}\vv{K})}\]
with $\vv{J} = \vv{N} + \vv{N}^\dagger$ and $\vv{K} = i (\vv{N}^\dagger - \vv{N})$. For $\Lambda_L$ the $N$ are $\vv{N}^\dagger = \frac{\vv{\sigma}}{2}$ and $\vv{N} = 0$. For $\Lambda_R$ they are $\vv{N}^\dagger = 0$ and $\vv{N} = \frac{\vv{\sigma}}{2}$. This follows, because $\vv{N}$ are $SU(2)$ groups and we need a representation of $SU(2)$ which describes spin $\frac{1}{2}$ particles.\\
With this parametrization the $\Lambda$ are given by
\[ \Lambda_L = \exp\left( -i \frac{\vv{\sigma}}{2} \left(\vv{\omega} + i \vv{\nu}\right)\right)\]
\[ \Lambda_R = \exp\left( i \frac{\vv{\sigma}}{2} \left( \vv{\omega} - i \vv{\nu}\right)\right)\]
Some properties of $\Lambda_{R,L}$ are
\begin{itemize}
\item $\Lambda_L^{-1} = \Lambda_R^\dagger$
\item $\sigma^2 \Lambda_L \sigma^2 = \Lambda_R^*$
\item $\sigma^2 \Lambda_L^{-1} \sigma^2 = \Lambda_L^T$
\end{itemize}
Now consider a Lorentz transformation of the field $\sigma^2 \psi_R^*$:
\[ \sigma^2 \psi_R^* ~~~\df~~~ \sigma^2 \Lambda_R^*\psi_R^* = \underbrace{\sigma^2 \Lambda_R^*\sigma^2}_{\Lambda_L} \sigma^2 \psi_R^* = \Lambda_L \sigma^2 \psi_R^*\]
Thus $\sigma^2 \psi_R^*$ is a left handed spinor because it transforms the same way. It is called the charge conjugate of $\psi_R$. Therefore:
\begin{itemize}
\item $\psi_L$ and $\sigma^2\psi_R^*$ transform as $\left(0, \frac{1}{2}\right)$
\item $\psi_R$ and $\sigma^2\psi_L^*$ transform as $\left(\frac{1}{2}, 0\right)$
\end{itemize}
$\psi_L$ and $\psi_R$ are called Weyl spinors.
\sub{Construction of scalars and vectors}
The Weyl spinors can be combined to construct scalars and vectors. First, consider products of two left handed spinors $\chi_L$ and $\psi_L$. Since $\left( 0, \frac{1}{2}\right) \otimes \left( 0, \frac{1}{2}\right) = (0,0) \oplus (0,1)$ we can construct a scalar from them:
\[ \chi_L^T \sigma^2 \psi_L ~~\overset{LT}{\df} \chi_L^T \Lambda_L^T \sigma^2 \Lambda_L \psi_L = \chi_L^T \sigma^2 \psi_L\]
In the second step we used, that $\Lambda_L^T \sigma^2 \Lambda_L$ is equal to $\sigma^2 \Lambda_L^{-1}\sigma^2\sigma^2 \Lambda_L$ due to the properties of $\Lambda_{L}$ and that $(\sigma^2)^2= 1$ and $\Lambda_L^{-1}\Lambda_L = 1$. \\
We therefore know, that $\chi^T_L\sigma^2\psi_L$ is a scalar. With
\[ \chi_L = \vektorz{\chi_1}{\chi_2}, ~~~ \psi_L = \vektorz{\psi_1}{\psi_2}\]
follows
\[ \chi_L^T \sigma^2 \psi_L = \begin{pmatrix} \chi_1 & \chi_2\end{pmatrix} \Matz{0}{-i}{i}{0} \vektorz{\psi_1}{\psi_2} = - i (\chi_1 \psi_2 - \chi_2 \psi_1)\]
This shows, that this scalar is antisymmetrical in the two fields. It is important to realize, that these are spin $\frac{1}{2}$ fields, so they must be represented by anitcommuting Grassmann numbers to conserve the properties demanded by spin $\frac{1}{2}$.\\
Hence even for $\chi_L = \psi_L$ we get
\[ \psi_L^T \sigma^2 \psi_L = - i( \psi_1 \psi_2 - \psi_2 \psi_1) = 2i \psi_2 \psi_1 \neq 0\]
This is non zero because of the anticommuting Grassmann numbers. This would not be the case, if these scalars would commute like normal scalars.\\
This combination is used in Majorana mass terms.\\
We can also take $\chi_L = \sigma^2\psi_R^*$ because it also is a lefthanded spinor. Then we get
\[\chi_L^T \sigma^2 \psi_L = \psi^\dagger_R(-\sigma^2)^2 \psi_L = - \psi_R^\dagger \psi_L\]
Which is a particle of the Dirac mass term.\\
\\
Vectors $\left(\frac{1}{2},\frac{1}{2}\right)$ can be obtained from the product of $\psi_L$ and $\sigma^2\psi_R^*$, $\left(\frac{1}{2}, 0 \right) \otimes \left(0,\frac{1}{2}\right) = \left(\frac{1}{2},\frac{1}{2}\right)$. \\
We start from $\psi_L^\dagger \psi_L$ which is invariant under rotations. Under boosts this transforms as 
\[ \psi_L^\dagger \psi_L ~~\df~~ \psi_L^\dagger \Lambda_L^\dagger \Lambda_L \psi_L = \psi_L^\dagger e^{\vv{\sigma}\vv{\nu}} \psi_L = \psi_L^\dagger \psi_L + \vv{\nu} \psi_L^\dagger \vv{\sigma} \psi_L + \ldots\]
Similarly for $\psi_L^\dagger \sigma^i \psi_L$:
\[ \psi_L^\dagger \sigma^i \psi_L ~~\df~~ \psi_L^\dagger e^{\vv{\sigma}\vv{\nu}/2} \sigma^i e^{\vv{\sigma}\vv{\nu}/2} \psi_L = \psi_L^\dagger \sigma^i \psi_L - \nu^i \psi_L^\dagger \psi_L\]
In the second step we used $e^{\vv{\sigma}\vv{\nu}/2} \sigma^i e^{\vv{\sigma}\vv{\nu}/2} = \sigma^i - \frac{1}{2} \nu^j \{\sigma^j,\sigma^i\} = \sigma^i - \nu^i \delta^{ji}$.\\
Summarizing the difference between transformed and untransformed vectors $\delta$ is given by
\[ \delta(\psi_L^\dagger \psi_L) = - \nu^i \psi_L^\dagger (-\sigma^i) \psi_L\]
and
\[ \delta( \psi_L(-i\sigma^i) \psi_L) = -\nu^i \psi^\dagger_L \psi_L\]
If we compare this with the general transformation of 4 vectors under boosts which is given by
\[ \frac{1}{2} \eps^{\mu\nu} M_{\mu\nu} = \eps^{0i} M_{0i} = - \eps^{0i}K^i = \vv{\nu}\vv{K} \]
where of course only boosts were considered. The difference here is given by
\[ \delta V^\mu = - \eps^\mu_{~\nu} V^\nu\]
For the time component this leads to
\[ \delta V^0 = \eps^{0i} V^i = - \nu^i V^i\]
and for the spacial components
\[ \delta V^i = - \eps^{i0} V^0 = \eps^{0i} V^0 =  -\nu^i V^0\]
Therefore $( \psi_L^\dagger \psi_L, - \psi_L^\dagger \vv{\sigma} \psi_L)$ transforms as a 4 vector. This is often times written as $\psi_L ^\dagger \sigma_-^\mu \psi_L$ with $\sigma_-^\mu \equiv (\ehm, - \vv{\sigma})$.\\
Similarly for the right handed spinors $\psi_R^\dagger \psi_R$ with $\Lambda_R = e^{\vv{\sigma}\vv{\nu}/2}$ we get the vector
\[ \psi_R^\dagger \sigma_+^\mu \psi_R = ( \psi_R^\dagger \psi_R, \psi_R^\dagger \vv{\sigma} \psi_R), ~~~ \sigma_+^\mu \equiv (\ehm, \vv{\sigma})\]
Lorentz scalars now can be formed by contracting with Lorentz vectors in particular with $\p_\mu$: $( \p_\mu \psi_L^\dagger) \sigma_-^\mu \psi_L,~~ \psi_L^\dagger \sigma_-^\mu ( \p_\mu \psi_L) $, etc are thus candidates for kinetic energy terms in Lagrangians. Becaue no point in spacetime is special $x^\mu$ does not appear in the physical theories, only terms with $\p^\mu$. Now, because $\p^\mu$ transforms as $\left(\frac{1}{2}, \frac{1}{2}\right)$ and thus as a vector field, $\psi^\dagger_L \sigma_-^\mu ( \p_\mu\psi_L)$ and $\psi_R^\dagger \sigma_+^\mu \p_\mu \psi_R$ transform as $(0,0)$ and are thus scalars. These expressions are just contractions of two four vectors.\\
\\
If we take a look at $\psi_R^\dagger\psi_L$ and $\psi^\dagger_L\psi_R$ it is obvious, that we need a $\psi_L$ if we have a $\psi_R^\dagger$ to form a scalar. This is the only way to get a scalar. Equally we can only get a vector if $\psi_R^\dagger$ is multiplied with another $\psi_R$, or the same with $\psi_L$.\\
That means, that in $\psi^\dagger_R \sigma_+^\mu \p_\mu \psi_R$ the $\sigma_+^\mu\p_\mu \psi_R$ must transform as a left handed Weyl spinor with $\left( 0,\frac{1}{2}\right)$. The same holds for $\sigma_-^\mu\p_\mu\psi_L$ with must transform as a right handed Weyl spinor with $\left( \frac{1}{2},0\right)$.\\
These terms are just spinors with a derivative multiplied with dirac matrices. So they are candidates for terms in equations of motion.


\sub{Field Equations for Weyl Spinors}
If we take a look at the newly found terms and try to construct an equation of motion for them, we can use $i \sigma_-^\mu\p_\mu \psi_L$ as a starting point. This is a right handed Weyl spinorfield, and thus it must be also equal to another right handed spinorfield.\\
There are three possibilities, to which every other solution can be reduced:
\begin{itemize}
\item $i \sigma_-^\mu \p_\mu \psi_L = 0$. This is the correct equation for massless, left handed particles, for example the massless neutrinos.
\item $i \sigma_-^\mu \p_\mu \psi_L = m \sigma^2 \psi_L^*$. This is the equation for Majorana particels. It would describe Majorana neutrionos. The $m$ was introduced, because the left side has units of energy, coming from $\p^\mu$, thus also the right side has to have units of energy, which led to an introduction of the mass $m$.
\item $i \sigma_-^\mu \p_\mu \psi_L = m \psi_R$. This is the Dirac equation. Now, because there is a $\psi_R$ in the equation for $\psi_L$ we also need an equation of motion for $\psi_R$, which must have at least one derivative of $\psi_R$ in it. Therefore we introduce another Dirac equation $i \sigma_+^\mu \p_\mu \psi_R = m \psi_L$.
\end{itemize}
The last two equations are equivalent with the known Dirac equation. This follows easily if we write the two equations as a combination for the Dirac spinor $\psi = \vektorz{\psi_L}{\psi_R}$:
\[ m \psi = m \vektorz{\psi_L}{\psi_R} = i \Matz{0}{\sigma_+^\mu}{\sigma_-^\mu}{0} \p_\mu\vektorz{\psi_R}{\psi_L}\]
The resulting matrix is called $\gamma^\mu$ and should have the known anticommutation relation for gamma matrices: $\{\gamma^\mu, \gamma^\nu\} = 2g^{\mu\nu}$:
\[\{\gamma^\mu, \gamma^\nu\} = \Matz{\sigma_+^\mu\sigma_-^\nu + \sigma_+^\nu\sigma_-^\mu}{0}{0}{\sigma_-^\mu\sigma_-^\nu+ \sigma_-^\nu\sigma_+^\mu}\]
If both indices are zero, this reduces to $2\ehm_4 = 2 g^{00}$ which satisfies the anticommutation relation. If only one index is zero and the other is $i = 1,2,3$ the matrix is equal to zero, because $\sigma_-^i + \sigma_+^i = \sigma^i - \sigma^i = 0 = 2 g^{0i}$ due to the signs in the definition of the $\sigma_\pm$ matrices. \\
If both indices are $i,j = 1,2,3$ the matrix can be written as 
\[ \{\gamma^i,\gamma^j\} = \Matz{-\{\sigma^i,\sigma^j\}}{0}{0}{-\{\sigma^i,\sigma^j\}}\]
The anticommutator of the Pauli matrices is known to be $\{\sigma^i, \sigma^j\} = 2\delta_{ij}$, thus the anticommutation relation is $\{\gamma^i,\gamma^j\} = - 2\delta_{ij} = 2 g^{ij}$ due to Minkovski metrics. All in all, the wanted anticommutation relation is fullfilled and therefore this is in fact the Dirac equation in another form.

\section{Parity}
Lets take a look at the parity transformation $x^\mu = (t, \vx) ~\overset{P}{\df}~ (t, - \vx) = \tilde{x}^\mu$.\\
In the irreducible representation of $SO(3,1)$ this transformation is $(j_1, j_2)~\overset{P}{\df}~ (j_2, j_1)$. This follows from the first component being $\vv{N} = \vv{J} + i \vv{K}$ and the second one being $\vv{N}^\dagger = \vv{J} - i \vv{K}$. Under parity transformation $\vv{J}$ stays the same and $i\vv{K}$ gets a minus sign, thus $\vv{N}$ transforms to $\vv{N}^\dagger$ and vice versa.\\
This means, that a left handed field becomes a right handed one and vice versa:
\[\psi_L(x) ~\overset{P}{\df}~ \psi_R(\tilde{x})\]
To construct a parity invariant theory, the right handed and left handed fields have to have the same masses and signs, etc. to remain conserved under parity transformation.\\
For the Dirac spinor the parity transformations yields
\[ \psi(x) = \vektorz{\psi_L}{\psi_R} ~\overset{P}{\df}~ \vektorz{\psi_R(\tilde{x})}{\psi_L(\tilde{x})} = \Matz{0}{\ehm_2}{\ehm_2}{0} \vektorz{\psi_L}{\psi_R} = \gamma^0 \psi(\tilde{x})\]
A projection onto $\psi_L$ or $\psi_R$ can be done with the projection operators:
\[ P_L \equiv P_- = \frac{1}{2} ( 1 - \gamma^5) = \Matz{\ehm_2}{0}{0}{0}\]
\[ P_R \equiv P_+ = \frac{1}{2} ( 1 + \gamma^5) = \Matz{0}{0}{0}{\ehm_2}\]
These operators project $\psi_R$ or $\psi_L$ out of $\psi$. Here the $\gamma^5$ matrix is $\gamma^5 = i \gamma^0 \gamma^1\gamma^2\gamma^3 = \Matz{-\ehm}{0}{0}{\ehm}$. The projection can be seen by applying $P_\pm$:
\[P_L\psi = \vektorz{\psi_L}{0} \equiv \psi_L, ~~~ P_R \psi = \vektorz{0}{\psi_R} \equiv \psi_R\]
This four component vector is sometimes written as $\psi_L$ or $\psi_R$ respectively, which can be easily confused with the two component Weyl spinor $\psi_L$ or $\psi_R$ which it consists of. Only from context it is visible, which of the two is meant.\\
This treatment of parity transformation is needed, because it will show, that quantum electro dynamics (QED) - the electromagnetic interaction, and quantum chromo dynamics (QCD) - the strong interaction, are in fact parity invariant, but the weak interaction is not. In fact it violates the parity conservation maximally, only left handed particles interact with the $W^\pm$ bosons.

\section{Plane Wave Solutions of the Dirac Equation}

Lets take a look at $\psi(x) = e^{-ikx}\psi(k)$ and find its plane wave solutions with the Dirac equation $i \p^\mu \gamma_\mu \psi = m \psi$. In $k$ space the Dirac equation yields $i (-i k^\mu) \gamma_\mu \psi = \slashed{k} \psi$ thus the equation to be solved is $(\slashed{k} - m) \psi(k) = 0$. Multiplying this equation with another $\slashed{k}$ leads to
\[ (k^2 - m \slashed{k}) \psi(k) = (k^2 - m ^2) \psi(k)= 0 \]
because $\slashed{k}^2 = k^2$ and $m\slashed{k}\psi(k) = m^2 \psi(k)$. (?)\\
Thus, $k^2 = m^2$ or $k^0 = \pm \sqrt{ \vv{k}^2 + m ^2}$ are solutions for $k$. The $\pm$ solutions are positive and negative energy solutions. For positive energy this is
\[\psi^+(x) = e^{-ipx}u(p)\]
and for negative energies
\[ \psi^-(x) = e^{+ipx}v(p)\]
Where $p^0 = + \sqrt{\vv{p} + m^2} = \pm k^0$ was used to express the exponents the same way.\\
All in all we have four solutions here, two $v$ solutions which correspond to anti partices and two $u$ solutions which correspond to partices, both with spin $\frac{1}{2}$. For both, $\left( 0, \frac{1}{2}\right)$ and $\left(\frac{1}{2}, 0\right)$ holds $\vv{J} \equiv \vv{S} = \frac{\vv{\sigma}}{2}$.\\
Thus for the Dirac spinor, the spin operator is given by
\[ \vv{S} = \Matz{\frac{\vv{\sigma}}{2}}{0}{0}{\frac{\vv{\sigma}}{2}}\]
The problem here arises from $(\slashed{k} - m)\psi(k) = 0$. If we look at $\slashed{p}$ we get
\[ \slashed{p} = \Matz{0}{E - \vv{\sigma}\vv{p}}{E + \vv{\sigma}\vv{p}}{0}\]
where $\vv{\sigma}\vp$ is the projection of the Pauli matrices onto the direction of the momentum, so $[\slashed{p}, S_i]\neq 0$ for directions not in the directions of the spin. So no simultaneous eigenvalues can be found for both.\\
The exceptions are twofold, the first is in the rest frame of the particle with $\vp = 0$. There we can choose a spin direction and boost to an arbitrary frame. This does not work for massless particles in ultrarelativistics.\\
The second, more general approach is, to choose the direction of $\vp$ as the quantization axis, then the helicity eigenvalues are the simultaneous eigenvalues for momentum and spin of the particle.\\
The components of $\vv{S}$ in the direction of $\vp$ are called the helicity:
\[\frac{\vv{S}\vp}{|\vp|} u(\vp, \lambda) = \frac{1}{2|\vp|}\Matz{\vv{\sigma}\vp}{0}{0}{\vv{\sigma}\vp}u(p,\lambda) = \frac{\lambda}{2}u(p,\lambda)\]
This matrix now commutes with $\slashed{p}$, and in any specific direction the eigenvalues are $\pm \frac{1}{2} = \lambda$. This means, that $\lambda = + \frac{1}{2}$ corresponds with positive heilcity and this spin in the direction of the momentum $\vp$.\\
The same holds for anti particles
\[ \frac{ \vv{S}\vp }{ |\vp| } v(p,\lambda) = - \frac{\lambda}{2}v(\vp, \lambda)\]
The minus sign is due to the interpretation of the anti particles as holes in the Dirac sea.\\
Expicitly, $u$ and $v$ have the following forms
\[ u(p, \lambda = \pm 1) = \vektorz{ \sqrt{ E\mp |\vp|} \chi_\pm(\vp)}{\sqrt{E\pm|\vp|} \chi_\pm(\vp)}\]
\[ v(p, \lambda = \pm 1) = \pm\vektorz{-\sqrt{E\pm|\vp|} \chi_\mp(\vp)}{\sqrt{E\mp |\vp|}\chi_\mp(\vp)}\]
These normalize to
\[ \bar{u}(p, \lambda) u(p, \lambda') = 2m \delta_{\lambda\lambda'}, ~~\bar{v}(p, \lambda)v(p, \lambda') = - 2m \delta_{\lambda\lambda'}, ~~ \bar{u}(p, \lambda)v(p,\lambda') = 0\]
and form the completeness relations
\[ \sum_\lambda u(p,\lambda) \bar{u}(p,\lambda) = \slashed{p} + m\]
\[ \sum_\lambda v(p, \lambda) \bar{v}(p,\lambda) = \slashed{p} - m\]
The important thing to notice is, that if we have a highly relativistic particle, the energy becomes $E = \sqrt{\vp^2 + m^2} \approx \vp$, so whenever there is a minus sign in the square roots of the expressions for $u$ and $v$, these are zero. So the two component spinors are reduced:
\[ u(p, \lambda = +1) ~~\df~~ \vektorz{0}{\sqrt{2E} \chi_+} = \vektorz{0}{\psi_R}\]
which has positive helicity and positive chirality,
\[ u(p, \lambda = -1) ~~\df~~ \vektorz{\sqrt{2E} \chi_-}{0} = \vektorz{\psi_L}{0}\]
which has negative helicity and negative chirality. And for the anti particles
\[v(p, \lambda = +1) ~~\df~~\vektorz{-\sqrt{2E}\chi_-}{0} = \vektorz{\psi_L}{0}\]
which has positive helicity but negative chirality and
\[v(p, \lambda = -1) ~~\df~~\vektorz{0}{- \sqrt{2E} \chi_+} = \vektorz{0}{\psi_R}\]
which has negative helicity but positive chirality.\\
Thus a particle with positive helicity in the ultra relativistic limit always consists of only the right handed component, whereas an anti particle with positive helicity is always reduced to a left handed component.

\section{Charge Conjugation}
Lets take a look at the Dirac equation $(i \slashed{\p} - q \slashed{A} -m ) \psi = 0$. This describes particles and anti particles at the same time. However, what is defined to be a particle and what an anti particles is random. The equation also describes particles with charge $-q$ instead of $q$ in the same electro magnetic field. The equation of a charge conjugated particle thus yields
\[ ( i \slashed{\p} + q \slashed{A} - m) \psi^C = 0\]
where $\psi^C(x) = C \bar{\psi}^T = - i \gamma^0 \gamma^2 \bar{\psi}^T$.\\
$\psi^C$ therefore is the chage conjugate field which makes anti particles a basic degree of freedom.\\
Explicitly we can take a look at a Dirac field
\[ \psi = \vektorz{\psi_L}{\psi_R} = P_L \psi + P_R\psi = \vektorz{\psi_L}{0} + \vektorz{0}{\psi_R} \equiv \psi_L + \psi_R\]
the charge conjugate of this field is
\[ \psi^C = i \gamma^2 \gamma^0 ( \psi^\dagger \gamma^0 )^T = i \gamma^2 (\gamma^0)^2 \psi^* = i \gamma^2 \psi^* = i \Matz{0}{\sigma_2}{-\sigma_2}{0} \vektorz{\psi_L^*}{\psi_R^*} = \vektorz{i \sigma_2 \psi_R^*}{-i\sigma_2\psi_L}\]
As we showed earlier, $i \sigma_2 \psi_R^*$ transforms as a left handed field and $i \sigma_2 \psi_L^*$ transforms as a right handed field.\\
One special case of these Dirac fields are the Majorana fermions. For them $\psi = \psi^C$ holds. Therfore, the charge conjugated field is
\[ \psi^C = \vektorz{i \sigma_2 \psi_R}{-i \sigma_2 \psi_L} = \vektorz{\psi_L}{\psi_R}\]
which follows because $\psi_L = i \sigma_2 \psi_R^* ~\df~ -i \sigma_2 \psi_L^* = -i \sigma_2( -i(-\sigma_2)) \psi_R = \psi_R$. Thus we can write a Majorana field as
\[ \psi_M = \vektorz{\psi_L}{-i \sigma_2 \psi_L^*}\]
This defines a fermion spinor which is automatically its own charge conjugate. Furthermore it can be shown, that every particles for which this relation holds, must have no charge.\\
\\
If we now take a look at the charge conjugation of $\psi_L$ we get
\[ (\psi_L)^C = \vektorz{\psi_L}{0}^C = \vektorz{ 0}{-i \sigma_2 \psi_L^*} = (\psi^C)_R\]
which is a right handed field. On the other hand, if we take the charge conjugate of the Dirac field and take the left handed component of it we get
\[ (\psi^C)_L = \vektorz{ i \sigma_2 \psi_R^*}{-i \sigma_2 \psi_L^*} = \frac{1}{2} ( 1 - \gamma^5) \vektorz{ i \sigma_2 \psi_R^* }{-i \sigma_2 \psi_L^*} = \vektorz{ i \sigma_2 \psi_R^*}{0} \]
which is a left handed field. Therfore the charge conjugation of a left handed field and the left handed component of a charge conjugated Dirac field are not equal.


 



\chapter{Classical Field Theory: Lagrangians}

The action $\SSS$ was defined as
\[ \SSS = \int_\Omega \dd^4 x \cc \LL( \phi_r(x), \p_\mu \phi_r(x))\]
where $\Omega$ was some region in Minkovski space. An infinitesimal variation around the original field was given as $\delta \phi_r(x) + \phi_r(x) = \phi_r'(x)$. This induced a variation of the action
\[ \delta \SSS = \int_\Omega \dd^4 x \cc \sum_r \left[ \p_\mu \left( \frac{\p \LL}{\p (\p_\mu \phi_r)} \delta \phi_r \right) + \left( \frac{ \p \LL}{\p \phi_r} - \p_\mu \left( \frac{\p \LL}{\p (\p_\mu \phi_r)}\right) \right)  \delta \phi_r\right]\soll 0 \]
For the equation of motion we looked at variations $\delta \phi_r$ that vanish on the surface $\p \Omega$ of $\Omega$ but are arbitrary elsewhere. Therefore the first term in the sum vanishes, because it can be reduced to a surface integral vial Gauss' law. Now, because $\delta \phi_r$ is arbitrary, the second term has to be equl to zero to satisfy $\delta \SSS = 0$. This then led to the Euler Lagrange equation
\[ \left( \frac{\p \LL}{\p \phi_r} - \p_\mu \frac{\p \LL}{\p ( \p_\mu \phi_r)} \right) = 0\] 

Now, if we only look at fields that do satisfy the Euler Lagrange equation the variation of the action can be reduced to
\[ \delta \SSS = \int_\Omega \dd ^4 x \cc \sum_r \left[ \p_\mu \left( \frac{\p \LL}{\p( \p_\mu \phi_r)} \delta \phi_r\right)\right] = 0\]
If we now take $\delta \phi_r$ to be a correlated variation instead of an arbitrary one, which then describes a symmetry, so that $\delta \SSS = 0$ still holds, then $\frac{\p \LL}{\p ( \p_\mu \phi_r)} \delta \phi_r$ has to be a conserved current. If the correlation of $\delta \phi_r$ is then given by $\delta_r = \frac{\delta \phi_r}{\delta \omega}\delta \omega$ then the conserved current is given by
\[ j^\mu \equiv \sum_r \frac{\p \LL}{\p ( \p_\mu \phi_r)} \frac{\delta \phi_r}{\delta \omega}\]
Also the variation of the action can be written as $\delta \SSS = \int_\Omega \dd^4 x \cc \p_\mu j^\mu \delta \omega = 0$ which implies the conservation of the current as $\p_\mu j^\mu = 0 = \p_0 j^0 + \vn \vv{j}$.
\sub{Example}
Lets take a look at a complex scalar field as an example:
\[ \phi = \frac{1}{\sqrt{2}} ( \phi_1(x) + i \phi_2(x))\]
The Lagrangian of this is given as
\[ \LL = ( \p_\mu \phi)^*(\p^\mu \phi) - V( \phi^*\phi) = \frac{1}{2} \p_\mu \phi_1^2 + \frac{1}{2} \p_\mu \phi_2^2 - V( \phi_1^2 + \phi_2^2)\]
Is invariant under $\phi(x) \df e^{-i\alpha} \phi(x)$ and $\phi^*(x) \df e^{i\alpha}\phi^*(x)$ or $\delta \phi = - i \alpha \phi$ and $\delta \phi^* = i \alpha \phi*$ in the infinitesimal representation. So the infinitesimal parameter $\omega$ here is called $\alpha$. The equation of motion of this is 
\[ \frac{\p \LL}{\p \phi} = - V'( | \phi|^2) \phi^* \soll \p_\mu \left( \frac{\p \LL}{\p ( \p_\mu \phi)}\right) = \p_\mu ( \p^\mu \phi^*) = \square \phi^*\]
Therefore the equations of motions are
\[ ( \square + V'(| \phi|^2)) \phi^* = 0, ~~~ ( \square + V'(|\phi|^2)) \phi = 0\]
These are generalizations of the Klein Gordon equations with a generalized $V$.\\
For fields which satisfy the equation of motion the conserved current $j^\mu$ can be constructed:
\[ j ^\mu = \frac{\p \LL}{\p (\p_\mu \phi)} \frac{\delta \phi}{\delta \alpha} + \frac{\p \LL}{\p(\p_\mu \phi^*)} \frac{\delta \phi^*}{\delta \alpha} = ( \p^\mu \phi^*)(-i \phi) + ( \p^\mu \phi) (+i \phi^*) = i ( \phi^* \p_\mu \phi - \phi \p_\mu \phi^*)\]
\section{Symmetries: Noethers Theorem}
Connected to the conserved current, there also is a conserved charge
\[ Q(x^0) \equiv \int \dd^3 \vx \cc j^0(x^0, \vx)\]
The fact that it is conserved just means, that 
\[ \frac{ \dd Q}{\dd t} = \int_{\Omega = \mathbbm{R}^3} \dd^3 \vx \cc \p_0 j^0 = - \int_\Omega \dd^3 \vx\cc \vn \vv{j} = \int_{\p \Omega} \dd \vv{S} \cc \vv{j} = 0\]
where the integrated volume is all of $\mathbbm{R}^3$. In the second step, the integral was transformed via Gauss' law and the last integral has to be equal to zero, because the current has to dissapear in infinity.\\
The connection between conserved current and conserved charge is described by Noethers theorem:\\\
The invariance of the Lagrangian $\LL$ under a continuous one-parameter set of transformations implies the existence of a conserved current $j^\mu$ with $\p_\mu j^\mu = 0$ and hence a conserved charge $Q = \int \dd ^3 x \cc j^0(x)$.\\
\\
In the general case this transformation can be written as $x^\mu ~\df ~ x'^\mu = x^\mu + \delta x^\mu$ and simultaneous $\phi_r(x) ~\df~ \phi_r'(x') = \phi_r(x) + \delta \phi_r(x)$.\\
These are correlated transformations which leave the action invariant: $\delta \SSS = 0$. The change in the action then is given by
\[ \delta \SSS = \int_\Omega \dd^4 x \cc \LL\left(\phi_r'(x'), (\p_\mu \phi_r)'(x')\right) - \int_\Omega \dd^4 x \cc \LL(\phi_r, \p_\mu \phi_r) = 0~~ \forall \Omega\]
This means, that $\delta(\dd ^4 x \LL) = 0$ must be fulfilled aswell, because $\Omega$ is arbitrary. This is given by
\[ \delta( \dd^4 x \LL) = \delta( \dd^4x) \LL + \dd^4x \delta\LL\]
We first take a look at $\delta( \dd^4 x)$, this is the difference between the measures of $\dd^4x$ and $\dd^4 x'$ which is given my the Jacobi determinant
\[ \delta( \dd^4 x) = \dd^4 x \cc \left( \det\left| \frac{\dd^4 x'}{\dd^4 x} \right| - 1\right)\]
The determinante yields
\[ \det( \p_\mu x^\nu + \p_\mu \delta x^\nu) = \det( g_\mu^{~\nu} + \p_\mu \delta x^\nu) ~\grqq = \grqq \det\begin{pmatrix}1+\eps & \eps & \eps & \eps\\ \eps &  1+ \eps & \eps & \eps \\ \eps & \eps & 1+ \eps & \eps \\ \eps & \eps & \eps & 1+ \eps \end{pmatrix}\]
The matrix can be expressed by infinitesimal $\epsilon$. On the diagonal entries the one from $g_\mu^{~\mu}$ is present. Because the $\eps$ are inifintesimally small the determinante approximally yields $1 + \eps + \OO(\eps^2)$. Thus follows
\[ \delta(\dd^4 x) = \dd^4x( 1+ \p_\mu \delta x^\nu -1) = \dd^4x( \p_\mu \delta x^\nu)\]
By separating the changes of $x$ and $\phi_r$ in $\delta \phi_r$ we get
\[ \delta \phi_r = \phi_r'(x') - \phi_r(x) = \underbrace{\phi_r'(x') - \phi_r'(x)}_{=\p_\rho \phi'_r(x) \delta x^\rho} + \underbrace{\phi_r'(x) - \phi_r(x)}_{\equiv \delta_0 \phi_r(x)} \]
where the second and third term were introduced as a $\pm$. This is being separated into a term which only changes in $\phi$ and one which only changes in $x$. By only regarding first oder terms we get
\[ \delta \phi_r = \delta x^\rho \p_\rho \phi_r(x) + \delta_0 \phi_r = \phi_r\]
The same can be done with $\delta \LL$:
\[ \delta \LL = \underbrace{ \LL( \phi'_r(x'), \ldots) - \LL(\phi_r'(x)}_{= \delta x^\mu \p_\mu \LL}+ \LL(\phi'_r(x)) - \LL(\phi_r) \]
The third and fourth term can be combined:
\[ \LL(\phi_r') - \LL(\phi_r) = \sum_r \frac{ \p \LL}{\p \phi_r} \delta_0 \phi_r + \frac{\p \LL}{\p (\p_\mu \phi_r)} \phi_\mu ( \delta_0 \phi_r) = \p_\mu \sum_r \frac{\p \LL}{\p ( \p_\mu \phi_r)} \delta_0 \phi_r\]
Here, it was used, that the first term in the second step can be rewritten through the equation of motion to $\frac{\p \LL}{\p \phi_r} = \p_\mu \frac{\p \LL}{\p ( \p_\mu \phi_r)}$. With that the whole thing could be rewritten as a total derivative in step three. Therefore follows
\[ \delta \LL = \delta x^\mu ( \p_\mu \LL) + \p_\mu \sum_r \left( \frac{\p \LL}{\p( \p_\mu \phi_r) } \delta_0 \phi_r\right)\]
All in all the change $\delta( \dd^4 x \LL)$ thus yields
\[ \delta( \dd^4 x\LL) = \dd^4 x \left[ \LL \p_\mu \delta x^\mu + \delta x^\mu ( \p_\mu \LL) + \p_\mu \left( \sum_r \frac{\p \LL}{\p( \p_\mu \phi_r)} \delta_0 \phi_r \right)\right]\]
the $\delta_0 \phi_r$ can be rewritten from the relation calculated before:
\[ \delta( \dd^4 x \LL) = \dd^4 x \p_\mu\left[ \LL \delta x^\mu + \sum_r \frac{\p \LL}{\p (\p_\mu \phi_r)} \delta\phi_r - \delta x^\rho \left( \p_\mu \frac{\p \LL}{\p ( \p_\mu \phi_r)} \right) \p_\rho \phi_r\right]\]
In the first term in the brackets, the $\delta x^\mu$ can be expressed as $g^\mu_{~\rho} \delta x^\rho$ so it can be written together with the third term which also is proportional to $\delta x^\rho$. The second term is proportional to $\delta \phi$ and everything can thus be written as
\[ \delta( \dd^4 x \LL) = \dd^4x \p_\mu \left[ \left( \LL g^\mu _{~\rho} - \sum_r \frac{\p \LL}{\p ( \p_\mu \phi_r)} \p_\rho \phi_r \right) \delta x^\rho + \sum_r \frac{\p \LL}{\p ( \p_\mu \phi_r)} \delta \phi_r \right]  = 0\]
The first term proportional to $\delta x^\rho$ will be called $- \mathcal{T}^\mu_{~\rho}$ and the whole thing in the brackets is proportional to a conserved current.
\[\mathcal{T}^\mu_{~\rho} = \sum_r \frac{\p \LL}{\p ( \p_\mu \phi_r)} \p_\rho \phi_r - g^\mu_{~\rho} \LL\]
With $\delta x^\rho = \frac{\delta x^\rho}{\delta \omega}\delta\omega$ and $\delta \phi_r = \frac{\delta \phi_r}{\delta \omega}{\delta \omega}$ we have the conserved current
\[j ^\mu = - \mathcal{T}^\mu_{~\rho} \frac{\delta x^\rho}{\delta \omega} + \sum_r \frac{\p \LL}{\p ( \p_\mu \phi_r)} \frac{\delta \phi_r}{\delta \omega}\]
So in summary, this is the energy momentum tensor times the variation of the coordiantes plus the variation of the field. Lets apply this to actual transformations:
\sub{Translations}
$x' = x + \eps$ and $\delta x^\rho = \eps ^\rho$. Here, moving the coordinate frame by $- \eps$ or alternatively $x$ by $\eps$ is equivalent. Thus $ \phi'(a) = \phi(a - \eps)$, which is $\phi'_r(x) = \phi_r(x - \eps)$ and thus $\phi_r'(x') = \phi_r(x)$. Therefore $\delta \phi_r = 0$ for a translation.\\
Therefore, in the current only the first term is non zero and for conserved quantities this means $\p_\mu \mathcal{T}^{\mu\rho} = 0$ where $\mathcal{T}^{\mu\rho}$ is the energy momentum tensor, which is therefore conserved. If we take a look at the momentum
\[ p ^\rho = \int \dd^3 x \mathcal{T}^{0\rho}(t, \vx)\]
we see, that they are time independant, so they are conserved in time. Furthermore, the zeroth component ($\rho = 0$) of the momentum corresponds to the energy of the field which is therefore also time independant
\[p^0 = \int \dd^3 x \left[ \sum_r \frac{ \p \LL}{\p ( \p_0 \phi_r)} \p_0 \phi_r- \LL\right] = \int \dd ^3 x \left[ \sum_r \frac{\p \LL}{\p \dot{\phi_r}} \dot{\phi_r}-\LL\right]\]
If we compare this to the Hamiltionian in classical quantum mechanics of a one dimensional system we get $H = p \dot{q} - L$ with the canonical momentum $p = \frac{\p L}{\p \dot{q}}$. This is somewhat equivalent to what we got for fields, just with less degrees of freedom. For fields the conjugate momentum is
\[ \Pi_r(x) \equiv \frac{\p \LL(x)}{\p \dot{\phi_r}(x)}\]
In summary, Translation invariance implies conserved $H$ and $\vp$.
\\Lets take a look at an example for this, the real Klein Gordon field. The Lagrangian of this is 
\[\LL = \frac{1}{2} ( \p_\mu \phi)^2 - \frac{m^2}{2}\phi^2\]
Thus, the energy momentum tensor is
\[ \mathcal{T}^{\mu\nu} = \frac{\p\LL}{\p (\p_\mu \phi)} \p^\nu \phi - g^{\mu\nu} \LL = (\p ^\mu \phi)(\p^\nu \phi) - g^{\mu\nu} \LL\]
The Hamiltionian in particular then is
\[ H = p ^0 = \int \dd^3 x \mathcal{T}^{00} = \int \dd^3 x ( (\p^0 \phi)^2 - \LL ) = \int \dd^3 x ( \dot{\phi}^2 - \LL)\]
By rewriting the Lagrangian as
\[ \LL = \frac{1}{2} \left( \dot{\phi}^2 - ( \vn \phi)^2 \right) - \frac{m^2}{2} \phi^2\]
we get the Hamiltionian
\[ H = \frac{1}{2} \int \dd ^3 x \left[ \dot{\phi}^2(x) + ( \vn \phi(x))^2 + m^2 \phi^2(x) \right]\]
So every term of the Hamiltonian is real and squared. Therefore the integral must be positive, $H \geq 0$. This is going to be the solution for the negative energy problem of the Klein Gordon equation later on.
\sub{Lorentz Invariance of the Action}
Lets take a look at the Lorentz invariance of the action $\SSS$, that is to say, the invariance under $ \delta x^\rho = \eps^{\rho \sigma}x_\sigma$ with antisymmetric $\eps^{\rho\sigma}$. The fields themselves are invariant under $\delta \phi_r = - \frac{i}{2} \eps^{\rho \sigma} \sum_s (S_{\rho\sigma})_r^{~s}\phi_s$ where $S_{\rho\sigma}$ are the generators.\\
With $\delta(\dd^4 x \LL) = 0 = \dd^4 x \p_\mu j^\mu$ we get
\[ 0 = \dd^4 x \p_\mu \left[ - \mathcal{T}^\mu_{~\rho} x_\sigma + \frac{1}{2} \sum_{r,s} \frac{\p \LL}{\p( \p_\mu \phi_r)} (-i S_{\rho \sigma})_r^{~s}\phi_s\right] e^{\rho \sigma}\]
where we pulled out the $\eps^{\rho\sigma}$ factor. Now the first term is symmetric under exchange of $\rho$ and $\sigma$ whereas the second one is anti symmetric. To have uncorrelated equations for each combination of $\sigma$ and $\rho$ also the first term has to be anti symmetric. Thus we add the corresponding term $\mathcal{T}^{\mu}_{~\sigma}x_\rho$. Now, the equations $\p_\mu \mathcal{M}^\mu_{~\rho\sigma} =0$ hold at an arbitrary choice of $\eps^{\rho\sigma}$.
\[\mathcal{M}^\mu = \mathcal{T}^\mu_{~\sigma} x_\rho - \mathcal{T}^\mu_{~\rho}x_\sigma - i \frac{\p \LL}{\p ( \p _\mu \phi_r)} ( S_{\rho\sigma})_r^{~s}\phi_s\]
Now, the conserved charges are
\[ M_{\rho \sigma} \equiv \int \dd^3 x \mathcal{M}^0_{\rho\sigma}\]
with $M_{ij}$ the generating rotations ($\sim \vv{J}$) and $M_{0i}$ the generating boosts ($\sim K_i$).
\sub{Free Dirac Field}
Another example is the free Dirac field with $\LL = \bar{\psi}( i \gamma^\mu \p_ \mu - m ) \psi$. Here the energy momentum tensor is
\[ \mathcal{T}^{\mu \nu} = (\p^\nu \bar{\psi}) \frac{\p \LL}{\p ( \p_\mu \bar{\psi})} + \frac{\p \LL}{\p ( \p_\mu \psi)} ( \p^\nu \psi) - g^{\mu\nu} \LL = i \bar{\psi} \gamma^\mu \p ^\nu \psi - g^{\mu\nu} \LL\]
And the Hamiltionian
\[ H = \int \dd^3 x T^{00} = \int \dd^3 x \left[ i \bar{\psi} \gamma^0 \p_0 \psi - \LL\right] = \int \dd^3 x \left[ i \bar{\psi}\vv{\gamma} \vn \psi + m \bar{\psi} \psi \right]\]
Alternatively we can use, that here, $\LL = 0$ if the equations of motion are satisfied, such that 
\[ H = \int \dd^3 x \left[ \psi^\dagger i \frac{\p}{\p t} \psi \right]\]
\section{Inner Symmetries: Dirac Equation}
The Dirac equation is $(i \gamma^\mu \p_\mu - m ) \psi = 0$ and the corresponding Lagrange density is $\LL = \bar{\psi}(i \gamma^\mu \p_\mu - m ) \psi$. The Lagrange density is invariant under $U(1)$ transformations
\[ \psi(x) ~\df~ \psi'(x) = e^{-i\alpha}\psi(x), ~~~ \bar{\psi}(x) ~\df~ \bar{\psi}'(x) = e^{i\alpha} \bar{\psi}(x)\]
This implies a conserved current
\[ j^\mu = \frac{\p \LL}{\p ( \p_\mu \psi)} \frac{\p \psi}{\p \alpha} + \frac{\p \bar{\psi}}{\p \alpha} \underbrace{ \frac{\p \LL}{\p( \p_\mu \bar{\psi})}}_{ = 0} \]
There is no term proportional to the energy momentum tensor, because we are looking at an internal symmetry here.\\
We do not only have one single field here, we do have many four component Dirac spinors. Therefore $\psi \df \psi_i$ with $i = 1, \ldots, N$. However, the masses should stay the same to conserve the symmetriy of the $\psi_i$. All of the $\psi_i$ can be written together as $\psi$. Then we replace $\psi ~\df~ U\psi$ and $\bar{\psi} ~\df~ \bar{\psi}U^\dagger$ which means $\psi_i ~\df~ U_{ij}\psi_i$. Under this transformation the Lagrange density should stay invariant:
\[ \bar{\psi} U^\dagger ( i \slashed{\p} - m ) U \psi \soll \bar{\psi}(i \slashed{\p} - m ) \psi\]
To fulfill this requirement for symmetry, $U$ must be unitary, so that $U^{-1}U = \ehm$ and $U$ must not be dependant on $x$, because it must commutate with $\slashed{\p}$. Therefore $U$ must be a constant unitary $N \times N$ matrix.\\
This matix can generally be split into a phasefactor times a special unitary matrix.\\
If we now consider the special case, that $U$ itself already is part of $SU(N)$: $U = e^{i \theta^a T^a}$ then a symmetry of the Lagrangian is
\[ \psi' = e^{i \theta^a T^a} \psi ~~~\df ~~~ \delta \psi = i \theta^a T^a\]
The continuous parameter of this transformation $\delta \omega$ here corresponds to $\delta \theta^a$ for $a = 1, \ldots, N$. For each $a$ there is a conserved current:
\[ J^{\mu a} = - \sum_j \frac{\delta \LL}{\delta( \p_\mu \psi_j)} \frac{\delta \psi_j}{\delta \theta^a} = - \sum _j ( \psi_j i \gamma^\mu) i T^a \psi_j) = \bar{\psi} \gamma^\mu T^a \psi\]
The minus sign was only introduced to cancel the $i^2$. The $T^a$ here is a $N \times N$ matrix and $\bar{\psi}$ as well as $\psi$ are $N$ component spinors.\\
\sub{scalar fields}
For scalar fields (specifically the Higgs field) we have two complex scalar fields $\phi_1$ and $\phi_2$. The Lagrangian here is
\[ \LL = \sum_{i = 1}^2 ( \p _ \mu \phi_i)^*( \p^\mu \phi_i) - V( | \phi_1|^2+ | \phi_2|^2)\]
We can combine $\phi_1$ and $\phi_2$ into a doublet $\phi = \vektorz{\phi_1}{\phi_2}$ which transforms under $SU(2)$ transformation to $U \vektorz{\phi_1}{\phi_2}$. Because $\phi^\dagger \phi$ is inviariant under $SU(2)$ transformation:
\[ \phi^\dagger \phi ~\df ~ \phi^\dagger U ^\dagger U \phi = \phi^\dagger \phi\]
it is useful to write the Lagrangian as
\[ \LL = ( \p_\mu \phi)^\dagger ( \p^\mu \phi) - V( \phi^\dagger \phi)\]
with $ \phi^\dagger = \begin{pmatrix} \phi_1^* & \phi_2^*\end{pmatrix}$.\\
We can always write $U$ as $U = e^{i \theta^a \sigma^a/2}$ for $SU(2)$, such that 
\[ \phi' = e^{i \theta^a \sigma^a/2} \phi = \left( 1 + i \theta ^a \frac{\sigma^a}{2}\right) \phi = \phi + \delta \phi, ~~~\delta \phi = i \theta^a \frac{\sigma^a}{2} \phi\]
The same holds for $\phi^\dagger$:
\[ \delta \phi^\dagger = - i \theta^a \phi^\dagger \frac{\sigma^a}{2}\]
Then the conserved currents are
\[ J^{\mu a} = \frac{ \delta \LL}{\p ( \p_\mu \phi)} \frac{\delta \phi}{\delta \theta^a} + \frac{\delta \phi^*}{\delta \theta^a} \frac{\delta \LL}{\p ( \p_\mu \phi^*)} = i\left( ( \p^\mu \phi^\dagger) \frac{\sigma^a}{2} \phi - \phi^\dagger \frac{\sigma^a}{2} ( \p^\mu \phi) \right)  \]
\section{Gauge Symmetries}
Imagin a free Dirac field $\psi$ with some Lagrangian
\[ \LL_0 = \bar{\psi}( i \gamma^\mu \p_\mu - m) \psi ~\df~(i \slashed{\p} - m )\psi = 0\]
With electromagnetic interaction we get an addition to this by minimal substitution
\[ \LL_\psi = \bar{\psi}\gamma^\mu ( i \p_\mu - q A_\mu) \psi - m \bar{\psi}\psi ~\df~ (i \slashed{\p} - q \slashed{A}- m) \psi = 0\]
As the Maxwell equations are gauge invariant we can gauge transform $A_\mu ~\df~A'_\mu = A_\mu + \p_\mu \Lambda$. With this gauge transformation the Lagrangian does not look gauge invariant:
\[\LL_\psi ~\df~ \LL'_\psi = \LL_\psi - q \bar{\psi}\gamma^\mu \psi \p_\mu \Lambda\]
We can however recover the gauge invariance by simultaneously transforming the field $\psi(x)$ with a global phase in $U(1)$ transformation:
\[ \psi(x) ~\df~ e^{-i \alpha} \psi(x)\]
If we take that phase to be local, that is to say, $\alpha = q \Lambda(x)$, this exactly cancels the extra term im the Lagrangian introduced by the transformation of $A_\mu$:
\[ \psi(x) ~\df~ e^{-iq \Lambda(x)} \psi(x) = U(x) \psi(x),~~~ \bar{\psi}(x) ~\df~ e^{iq\Lambda(x)} \bar{\psi}(x) = \bar{\psi}(x) U^{-1}(x)\]
Because of the $x$-dependance of $\alpha$ the derivative of $\psi$ yields
\[ i \p_\mu \psi ~\df~ i \p_\mu ( e^{-iq\Lambda} \psi) = e^{-iq\Lambda}(i \p_\mu \psi) - i ( \p_\mu\Lambda) \psi\]
this extra term exactly cancels the one in $\LL_\psi$.\\
Thus the Lagrangian $\LL_\psi$ is invariant under the simultaneous transformations $\psi ~\df~U\psi$, $\bar{\psi}~\df~\bar{\psi}U^\dagger$, $A_\mu ~\df~A_\mu + \p_\mu \Lambda$.\\
That we used the minimal subsitution to derive this gauge invariance was only motivated by the gauge invariance of the Maxwell equations. In another way we can say, that the momentum transforms as $p ~\df p - q A$ and therefore also the derivative $\p_\mu ~\df~ \p_\mu + i p A_\mu \equiv \Dd_\mu$ transforms into the covariant derivative. If we look at the local transformation $\psi(x) ~\df~ U(x)\psi(x)$, we also want the covariant derivative of $\psi$ to transform like $\psi$:
\[ \Dd_\mu \psi = ( \p _\mu + i q A_\mu) \psi ~\df~ U(x)( \Dd_\mu \psi)(x) = (\Dd_\mu\psi)' = \Dd'_\mu \psi' = \Dd'_\mu U \psi \]
This transformation implies $U\Dd_\mu \psi = \Dd'_\mu U \psi$ and thus $\Dd'_\mu = U \Dd_\mu U^{-1}$. Therefore the transformation of $\Dd_\mu$ yields:
\[ \Dd'_\mu = e^{-iq\Lambda}( \p_\mu + i q A_\mu) e^{i q \Lambda} = e^{-iq\Lambda}\left[ e^{iq\Lambda}(iq (\p_\mu \Lambda) + \p_\mu) + i q A_\mu e^{iq\Lambda}\right] = \p_\mu + iq(A_\mu + \p_\mu \Lambda) = \p_\mu + iq A'_\mu\]
So we rederived the transformation of the vector potential from the Maxwell equations which we already used earlier. Therefore, the transformation $\Dd'_\mu = U \Dd_\mu U^{-1}$ is equivalent to $A'_\mu = A_\mu + \p_\mu \Lambda$ for the $U(1)$-symmetry.\\
This all gives
\[ \LL_\psi = \bar{\psi} i \gamma^\mu \Dd_\mu \psi - m \bar{\psi}\psi\]
and
\[ \LL'_\psi = \bar{\psi} U^\dagger i \gamma^\mu U \Dd_\mu \psi - m \bar{\psi}U^\dagger U \psi = \bar{\psi}i \gamma^\mu \Dd_\mu \psi - m \bar{\psi}\psi\]
which is invariant.\\
If we now add the Lagrangian for the free electromagnetic field
\[ \LL_A = - \frac{1}{4} F_{\mu \nu} F^{\mu \nu}, ~~~ F_{\mu \nu} = \p_\mu A_\nu - \p_\nu A_\mu\]
The covariant derivative allows us to construct the field strenght tensor in the first place
\[ [\Dd_\mu, \Dd_\nu] = [ \p_\mu + i q A_\mu, \p_\nu + i q A_\nu] = i q ( \p_\mu A_\nu - iq ( \p_\nu A_\mu) = iq F_{\mu\nu}\]
The full form of the field strenght tensor is
\[ F_{\mu\nu} = \begin{pmatrix} 0 & -E_x & - E_y & -E_z \\
E_x & 0 & - B_z & B_y \\
E_y & B_z & 0 & - B_x \\
E_z & -B_y & B_x & 0 \end{pmatrix}\]
Also, $F_{\mu\nu} ~\df~F'_{\mu\nu} = F_{\mu\nu}$ doesnt change under gauge transformation.\\
The complete Lagrangian for quantum electro dynamics then is
\[\LL = -\frac{1}{4} F_{\mu\nu}F^{\mu\nu} + \bar{\psi} ( i \gamma^\mu \Dd_\mu - m )\psi\]
\subsection{Generalization: Non Abelian Gauge Theory}
In the non abelian case, we start with $N$ Dirac fields $\psi_i$, $i = 1, \ldots, N$, all with the same mass $m$. Then
\[ \LL_0 = \sum_{i = 1}^N \bar{\psi}_i ( i \slashed{\p} - m ) \psi_i = \bar{\psi}( i \slashed{\p}- m) \psi, ~~~ \psi = \vektor{\psi_1}{\vdots}{\psi_N}, ~~~ \bar{\psi} = \begin{pmatrix}\bar{\psi}_1 & \ldots & \bar{\psi}_N\end{pmatrix}\]
We already know, that this is invariant under global $U(1)\times SU(N)$ transformation:
\[ \psi_i(x) ~\df~ U_{ij} \psi_j(x), ~~~~ \psi ~\df~ U \psi, ~~~~ \bar{\psi}~\df~\bar{\psi}U^\dagger, ~~~~ U = e^{i \theta^a \frac{\lambda^a}{2}}
\]
If we make the $SU(N)$ symmetry a local one, such that $\psi(x) ~\df~ U(x) \psi(x)$ and $\bar{\psi}(x) ~\df~\bar{\psi}(x) U^{-1}(x)$ and also replace the derivative $\p_\mu \psi ~\df~ \Dd_\mu \psi$ we can again try to make the covariant derivative to have the same transformation properties as the fields $\psi$: $(\Dd_\mu \psi)' = U \Dd_\mu \psi$. This means, $\Dd'_\mu$ must transform as $\Dd'_\mu = U \Dd_\mu U^{-1}$.\\
If such a $\Dd_\mu$ exists, then $\LL_\psi = \bar{\psi}(i \gamma^\mu \Dd_\mu - m ) \psi$ will be manifestally invariant under the local transformation. Lets find the structure of this covariant derivative by trying the same as in the abelian case:
$\Dd_\mu = \p_\mu + i g A_\mu$ yields
\[\Dd'_\mu = \p_\mu + i g A'_\mu = U(\p_\mu + i gA_\mu) U^{-1} = U( \p_\mu U^{-1}) + UU^{-1} \p_\mu + i g U A_\mu U^{-1}\]
Here, $U A_\mu U^{-1}$ does not neccesarily commute. Thus this only reduces to
\[ \Dd'_\mu = U(\p_\mu U^{-1}) + \p_\mu + i g U A_\mu U^{-1}\]
From which the transformation of $A_\mu$ follows as
\[ A'_\mu = \frac{1}{ig} U( \p_\mu U^{-1} )+ U A_\mu U^{-1}\]
In the abelian case, $U A_\mu U^{-1} = A_\mu$ commutes and we get the same thing we derived earlier.\\
If we now take a look at infinitesimal transformations $\theta^a$ this yields
\[ \frac{1}{ig} U \left( \p_\mu e^{-i \theta^a \frac{\lambda^a}{2}}\right) = \frac{1}{ig} ( -i \p_\mu \theta^a) \frac{\lambda^a}{2} + \ldots\]
Here $\p_\mu \theta^a$ plays the exact same role as the $\p_\mu \Lambda$ played in the abelian case: $\p_\mu \theta^a \leftrightarrow \p_\mu \Lambda$. The only real difference being, that there is only one $\Lambda$ but $N^2 -1$ $\theta^a$.\\
We now want to absorb the $N^2-1$ $\p_\mu \theta^a$ terms into the gauge fields:
\[ A_\mu = A_\mu ^a \frac{\lambda}{2} \equiv \sum_{a = 1}^{N^2 -1} A_\mu^a T^a\]
Here, $A_\mu^a$ must be a Lorentz four vector, because $\p_\mu \theta^a$ implies, that $\theta^a$ transforms like a Lorentz four vector. To absorb $\p_\mu \theta^a$ into $A_\mu^a$ it has to transform the same way.\\
The infinitesimal gauge transformation yields
\[ A_\mu^{a\prime} T^A = - \frac{1}{g} ( \p_\mu \theta^a) T^a + ( 1 + i \theta^a T^a) A_\mu^b T^b ( 1-i \theta^c T^c)\]
Expanding this up to linear terms yields
\[ A_\mu^{a\prime} = - \frac{1}{g} ( \p_\mu \theta^a) T^a - A_\mu^a T^a + i \theta^a\underbrace{( T^a T^b - T^b T^a)}_{= i f^{abc}T^c} A_\mu^b = -\frac{1}{g} (\p_\mu \theta^a)T^a + T^a( A_\mu^a - f^{cba} A_\mu^b \theta^c)\]
where in the last step the indices were relabelled.\\
This equation is equivalent to
\[ A_\mu^{a\prime} = A_\mu^a + f^{abc} A_\mu^b \theta^c - \frac{1}{g}(\p_\mu \theta^a)\]
Which follows from pulling out the $T^a$. Therefore $f^{abc}A_\mu^b \theta^c$ is an additional term only appearing in the non abelian case.\\
In summary, $SU(N)$ gauge theory is constructed with covariant derivatives $\Dd_\mu + igA_\mu^a T^a$ with $N^2-1$ gauge fields $A_\mu^a$ where $N^2-1$ is the number of generators. These $A_\mu^a$ transform inhomogenously under the adjoint representation. From this it is apparent, that people claim, that there are eight gluons, as this is the number of gauge fields for $SU(3)$ which describes quantum chromo dynamics.\\
\sub{Field stength tensor in non abelian gauge theory}
Lets try to define a field strength tensor for non abelian gauge theory. We choose the same commutation relation as in the abelian case
\[ [\Dd_\mu, \Dd_\nu] ~\df [ U \Dd_\mu U^{-1}, U\Dd_\nu U^{-1}] = U [ \Dd_\mu, \Dd_\nu]U^{-1}\]
this yields
\[ U [ \Dd_\mu, \Dd_\nu]U^{-1} = [ \p_\mu + i g A_\mu^a T^a, \p_\mu + i gA_\nu^aT^a] = ig ( \p_\mu A_\nu^b)T^b - igT^a(\p_\nu A_\mu^a) + (ig)^2 A_\mu ^a A_\mu^b \underbrace{[T^a, T^b]}_{= i f^{abc} T^c}\]
If we factor out one factor of the generators by relabelling the indices we get
\[ U [ \Dd_\mu, \Dd_\nu]U^{-1} = i gT^a \left( \p_\mu A_\nu^a - \p_\nu A_\mu^a - g \underbrace{f^{bca}}_{= f^{abc}}A_\mu^b A_\mu^c\right) \equiv i g T^a F^a_{\mu\nu}\]
with the non abelian field stenght tensor
\[ F^a_{\mu\nu} = \p_\mu A^a_\nu - \p_\nu A_\mu^a - gf^{abc} A_\mu^b A_\nu^c\]
The non abelian Lagrangian can thus be postulated (as done by Yang and Mills) as
\[ \LL_{YM} = - \frac{1}{4} F_{\mu\nu}^a F^{a\mu\nu} + \bar{\psi}(i \slashed{\Dd} - m ) \psi\]
The first term $F^a_{\mu\nu}F^{a\mu\nu}$ is gauge invariant:
\[ \frac{1}{2} F_{\mu\nu}^a F^{a\mu\nu} = \tr\left( \frac{\lambda^a}{2}\frac{\lambda^b}{2}\right) F_{\mu\nu}^aF^{b\mu\nu} = \tr( F_{\mu\nu}F^{\mu\nu})\]
(?)The gauge transformed term then is
\[\tr(F_{\mu\nu}'F^{\prime\mu\nu}) = tr( U F_{\mu\nu}U^{-1}UF^{\mu\nu}U^{-1}) = \tr(F_{\mu\nu}F^{\mu\nu})\]
So the whole Lagrangian $\LL_{YM}$ is gauge invariant. 




\chapter{Quantization of Fields}

In classical mechanics we described a point particle with the action $S = \int \dd t L(q_i, \dot{q}_i)$ and the conjugate momentum $p_i = \frac{\p L}{\p \dot{q}_i}$. So we get pairs of coordinates $q$ and conjugate momenta, which are constructed by this. The quantization then followed from the condition $[q_i(t), p_j(t)] = i \hbar \delta_{ij}$. From this Lagrangian and the definition of the conjugate momentum we can impose the quantization condition on the Hamiltonian $H = \sum_i p_i \dot{q}_i - L$.\\
Now, looking at field theories, we can use the same basic structures. Here the $q_i(t)$ correspond to te vaules of the field at space time points $\phi_r(\vx,t)$ where $r$ and $\vx$ are just indices of the field, just like the $i$. The conjugate momenta here are defined as 
\[ \Pi_r(\vx, t) = \frac{\p \LL}{\p \dot{\phi}_r(\vx,t)}\]
as derived in a previous chapter.\\
We then postulate equal time commutators
\[ [\phi_r(\vx, t) , \Pi_s(\vy, t) ] = i \delta_{rs} \delta^{(3)}(\vx - \vy)\]
\[ [ \phi_r(\vx, t) , \phi_s( \vy, t) ] = 0 = [ \Pi_r(\vx, t), \Pi_s(\vy,t)]\]
where the times are equal, but the indices not neccesarily are. From this follows, that two fields at different space points are independant. These commutations only hold for fields with integer spin though. For fermion fields with spin $\frac{1}{2}$ the commutation in these relations will be replaced by the anticommutator.
\sub{Application: free, real Klein Gordon field}
For a free, real Klein Gordon field, the Lagrangian is $\LL = \frac{1}{2}( \p_\mu \phi)^2 - \frac{m^2}{2} \phi^2$, so the conjugate momentum is $\Pi(x) = \frac{\p \LL}{\p \dot{\phi}} = \dot{\phi}(x)$. From the commutation relation follows $[\phi(x), \Pi(y)]_{x^0=y^0} = i \delta( \vx-\vy)$, all other commutators at equal times vanish. Also, this Lagrangian yields an equation of motion known as the Klein Gordon equation, $\square \phi + m^2 \phi = 0$ where the solutions are superpositions of plane waves
\[ \phi(x) = \int \frac{\dd ^3 \vv{k}}{(2\pi)^3 2 \omega_k} \left( e^{-ikx}a(k) + e^{ikx}a^\dagger(k)\right)\]
with $\omega_k \equiv \sqrt{ \vv{k}^2 + m^2}$. These solutions are fundamentally hermitian.\\
To get a hand on $a(k)$ and $a^\dagger(k)$ we have to invert the fouriertransformation of $\phi(x)$. For this it is useful, to rewrite $e^{-ikx} = e^{-i\omega_kt}e^{i \vv{k}\vx}$ and the h.c. accordingly. Now, by integrating the field $\phi(x)$ we get
\[ \int \dd^3 \vx e^{-i\vp\vx}\phi(t, \vx) = \frac{1}{2\omega_p}\left( a(\vp) e^{-i\omega_p t} + a^\dagger(-\vp)e^{i \omega_p t}\right)\]
It is important to notice, that in this notation $\vp$ and $\vv{k}$ are two instances of the same thing. However, we do notice, that the integration of $\phi(x)$ on its on does not suffice to separate $a$ and $a^\dagger$, we also need the integration of the conjugate momentum
\[ \int \dd^3 \vx e^{-i \vp \vx} \underbrace{\Pi(t, \vx)}_{\dot{\phi}(t,\vx)} = \frac{i}{2} \left( - a (\vp) e^{-i \omega_p t} + a^\dagger(- \vp) e^{i \omega_p t}\right)\]
In the calculation of both these integrals we used
\[ \int \dd ^3 \vx e^{i(\vv{k} - \vp)\vx} = (2\pi)^3 \delta^{(3)}(\vp - \vv{k})\]
which in the $a^\dagger$ terms has a plus sign in the $\delta^{(3)}$ function which is why there are minus signs in the $\vp$ dependencies of $a^\dagger(-\vp)$. Now, looking at the sum of both integrations we get
\[ a(p,t) \equiv a(p) e^{-i\omega_p t} = \int \dd^3 \vx e^{-i \vp \vx} \left( \omega_p \phi(t, \vx) + i \Pi(t, \vx)\right)\]
and similarly for the h.c. case
\[ a^\dagger(p,t) \equiv e^{i \omega_p t} a^\dagger(\vp) = \int \dd^3 \vx e^{i \vp \vx} ( \omega_p \phi(t, \vx) - i \Pi(t, \vx) )\]
What we need to find out now, is, what the commutation relations for $a$ and $a^\dagger$ are:
\[ [a(\vp), a^\dagger( \vv{k}) ] = e^{i(\omega_p - \omega _k) t} \int \dd^3 x \dd^3 y e^{-i \vp \vx} e^{ i \vv{k} \vy} \cc [ \omega_p \phi(t, \vx) + i \Pi(t, \vx), \omega_k\phi(t, \vy) - i \Pi(t, \vy)]\]
In the commutator, only cross terms between $\phi$ and $\Pi$ have an impact, which yields
\[ [ \omega_p \phi(t, \vx) + i \Pi(t, \vx), \omega_k\phi(t, \vy) - i \Pi(t, \vy)] = \omega_p(-i)i\delta^{(3)}(\vx - \vy) + i \omega_k ( -i \delta^{(3)}(\vx - \vy)) = \delta^{(3)} ( \omega_p + \omega_k)\]
where the $\delta$-distribution in the second term got a minus sign, because the argument was switched around. With this, the Integration can be reduced to one variable, yielding 
\[ \int \dd^3 \vx e^{-i ( \vp - \vv{k} ) \vx} = (2\pi)^3 \delta^{(3)}(\vp - \vv{k})\]
Alltogether this gives
\[ [a(\vp) , a^\dagger( \vv{k})] = e^{i(\omega_p - \omega_p)t} (2\pi)^3 \delta^{(3)}(\vp - \vv{k}) ( \omega_p + \omega_p) = (2\pi)^3 2 \omega _p \delta^{(3)}(\vp - \vv{k})\]
In a similar calculation, the other commutator $[a(\vp), a(\vv{k})] = [a^\dagger(\vp), a^\dagger(\vv{k})] = 0$ follow.\\
Now, if we take a look at the field $\phi(x)$ in a primed frame
\[ \phi'(x') = \int \frac{\dd^3 \vv{k}'}{(2\pi)^3 w \omega_k'} \left( a'(k') e^{-i k'x'} + \hc\right) \soll \phi(x)\]
must be equal to $\phi(x)$, because the Klein Gordon equation is Lorentz invariant, and thus also the field must be equivalent to its boosted $\phi'(x')$. That means, $\phi(x)$ is a scalar field.\\
We now start from a Lorentz invariant measure $\dd ^4 k \delta( k^2 - m^2) \theta(k^0)$ which is invariant under an orthochronous Lorentz transformation, that is to say a transformation that doesnt change the sign of the $k^0$ component. Here, also $x'^\mu = \lambda^\mu_{~\nu} x^\nu$ and $k'^\mu = \Lambda^\mu_{~\nu}k^\nu$ also are Lorentz invariant. Now, any integral of a test function is $\int \dd^4 k\cc \delta(k^2 - m^2) \theta(k^0) f(k)$ where the $\delta$-function can be rewritten as
\[ \delta\left( (k^0)^2 - \vv{k}^2 - m^2\right) = \delta\left( (k^0)^2 - \omega_k^2\right) = \frac{1}{2 |k^0|} \left( \delta( k_0 - \omega_j) + \delta( k_0 + \omega_k)\right) \]
Now, because $k^0 > 0$ because of $\theta(k^0)$, the second term is equal to zero and therefore the integral yields
\[ \int \frac{\dd^3 \vv{k}}{2 \omega_k} f( k^0 = \omega_k, \vv{k})\]
This means, that this measure also is Lorentz invariant. So, $\frac{\dd^3 \vv{k}'}{2\omega_k'} = \frac{\dd^3 \vv{k}}{2 \omega_k}$ is the same, which then in turn leads to
\[ \phi'(x') = \int \frac{ \dd^3 \vv{k}}{2 \omega_k (2\pi)^3} \left( a'(k') e^{-i k x} + \hc\right) \soll \phi(x)\]
Thus, also $a(k)$ must be Lorentz invariant, so that $a(k) = a'(k')$. Which means, that $a(k)$ must be a Lorentz scalar.\\
As we will need the calculated measure for many more problems in the future, we will abbreviate it as $\dd \tilde{k} \equiv \frac{\dd^3 \vv{k}}{(2\pi)^3 2 \omega_k}$.\\
Now, the Hamiltonian and $\vp$ we get from the energy momentum tensor as derived in an eariler chapter are
\[ \mathcal{T}^{\mu\nu} = ( \p^\mu \phi)(\p_\mu \phi) - \LL g^{\mu\nu}, ~~~p^\nu = \int \dd^3 \vx \mathcal{T}^{0\nu}\]
and
\[ \vp = \int \dd^3 \vx \Pi(t,x)\left(-\vn \phi(t,\vx)\right)\]
which gives the Hamiltonian
\[ H = \frac{1}{2} \int \dd^3 \vx \left( \dot{\phi}^2(t,\vx) + ( \vn \phi)^2 + m^2 \phi^2\right)\]
With the field
\[ \phi(\vx, t) = \int \dd \tilde{k} \left[ a (\vv{k}) e^{-i \omega_k t} + a^\dagger(-\vv{k}) e^{i \omega_k t} \right] e^{i \vv{k}\vx}\]
and its derivative
\[ \dot{\phi}(\vx, t) = \int \dd \tilde{k}\cc i \omega_k \left[ -a(\vv{k}) e^{-i \omega_k t} + a^\dagger (-\vv{k}) e^{i \omega_k t} \right] e^{i \vv{k}\vx}\]
the Hamiltionian yields

\begin{align*}
H =& \frac{1}{2} \int \dd \tilde{k} \dd \tilde{p} \left[ - \omega_k \omega_p \left( - a(k) + a^\dagger(-k) \right)\left( -a(p) + a^\dagger(-p)\right) \right. \\
&+ \left. ( - \vv{k} \vp + m^2) \left( a(k) + a^\dagger(-k) \right) \left( a(p) + a^\dagger(-p)\right)\right] \int \dd^3 \vx e^{i \vv{k} \vx}e^{i \vp \vx}
\end{align*}
where the notation $a(k) = a(\vv{k})e^{-i\omega_kt}$ and $a^\dagger(-k) = a^\dagger(-\vv{k})e^{i\omega_kt}$ was used. The last integral over $\dd^3 \vx$ is equal to $(2\pi)^3 \delta^{(3)}(\vv{k} + \vp)$ which then implies $\vv{k} = - \vp$ and $\omega_k = \omega_p = \omega$.\\
Thus $-\vv{k}\vp + m^2 =  \vp^2 + m^2 = \omega_p^2$ and $-\omega_k \omega_p = -\omega_p^2$ yield twice the same factor for both terms in the first integral. Also the double integral at the front reduces to $\int \dd\tilde{p}\dd\tilde{k} = \int \frac{\dd\tilde{p}}{2 \omega_p (2\pi)^3}$. The $(2\pi)^3$ then cancels with the one from the last integral and the $\frac{1}{\omega_p}$ cancels with one of the now global $\omega_p^2$. All in all this reduces the Hamiltonian to
\[H = \frac{1}{2} \int \dd\tilde{p} \frac{\omega_p}{2} \left[ a(-\vp)a^\dagger(-\vp) + a^\dagger(\vp)a(\vp)\right] = \int \dd \tilde{p} \frac{\omega_p}{2} \left[a(\vp) a^\dagger(\vp) + a^\dagger(\vp) a(\vp) \right]\]
Here, all the time dependant phases $e^{i\omega t}$ vanish. This can further be rewritten by turning around the order of $a$ and $a^\dagger$ in the first term, such that
\[H = \int \dd \tilde{p} \cc\omega_p \left( a^\dagger(\vp) a(\vp) + \frac{1}{2}[ a(\vp), a^\dagger(\vp) ] \right)\]
The commutator here is equal to $[a(\vp), a^\dagger(\vp)] = (2\pi)^3 \omega_p \delta^{(3)}(0)$.
\section{Annihilation and Creationoperators}
If we take a look at the four momentum operator
\[P^\mu = \int \dd \tilde{k} k^\mu a^\dagger(k) a(k) + \ehm Q_0^\mu\]
with $k^\mu = (\omega_k, \vv{k})$ and $Q_0^\mu$ (?) we can show, that the $a$ and $a^\dagger$ are infact annihilation and creation operators.\\
Therefore, lets consider the eigenstate $|k\rk$ of $P^\mu$, $P^\mu|k\rk = k^\mu|k\rk$ and now calculate
\[ P^\mu\left( a^\dagger(p) |k\rk\right) = [ P^mu, a^\dagger(p)]|k\rk + a^\dagger(p) k^\mu|k\rk\]
The commutator is
\[ [P^\mu, a^\dagger(p)] = \int \dd \tilde{k}\cc k^\mu [ a^\dagger(k)a(k), a^\dagger(p)] = \int \dd \tilde{k}\cc k^\mu \left([ a^\dagger(k), a^\dagger(p)]a(k) + a^\dagger(k) [a(k), a^\dagger(p)]\right)\]
Now, the first commutator is equal to zero and the second one is known from before, $[a(k), a^\dagger(p)] = (2\pi)^3 2 \omega_k \delta^{(3)} (\vv{k} - \vp)$, thus
\[ [P^\mu, a^\dagger(p)] = \int \dd \tilde{k}\cc a^\dagger(k)(2\pi)^3 2 \omega_k \delta^{(3)}(\vv{k}-\vp) = p^\mu a^\dagger(p)\]
From which follows
\[ P^\mu \left( a^\dagger(p) |k\rk\right) = ( p + k )^\mu a^\dagger(p)\]
So we found, that applying $a^\dagger(p)$ to a state $|n\rk$ raises the energy by $\omega_p$ and the momentum by $\vp$, which is euqivalent to adding a new particle of four momentum $p^\mu$ to the state.\\
Similarly we get the commutator $[P^\mu, a(p)] = - p^\mu a(p)$ from is equivalent to taking away a particle of four momentum $p^\mu$ from the state.\\
Also, there is a ground state, or vacuum state $|0\rk$ with $a(k) |0\rk = 0 ~\forall k = (\omega_k, \vv{k})$.


\section{Fockspace, Hilbertspace}
The basis of the Hilbertspace (Fockspace) is given by the zero particle state, or vacuum state $|0\rk$, which is unique with the property $a(k)|0\rk = 0~\forall k$. All the other states are build by applying the creation operator onto the vacuum state.\\
For example, the one particle states are constructed as $a^\dagger(k_1)|0\rk = |k_1 \rk$, and the two particle states as $a^\dagger(k_1)a^\dagger(k_2)|0\rk = |k_1, k_2\rk$. This is the same as $a^\dagger(k_2)a^\dagger(k_1)|0\rk = + |k_2,k_1\rk$ because the $a^\dagger$ commute. Therefore, this is automatically Bose symmetric.\\
In the Hilbertspace, the momentum eigenstates are normailzed as distributions, the matrix element of two one particle states is
\[ \lk k_1 | k_2 \rk = \lk 0 | a(k_1)a^\dagger(k_2) | 0 \rk = \lk 0 | [ a(k_1), a^\dagger(k_2)] + a^\dagger(k_2) a(k_1)|0\rk\]
In the last step, the expression between the bra-ket was rewritten with the commutator. Now the extra term vanishes, because $a(k_1)|0\rk$ is zero, and the commutator can be calculated, which leads to
\[ \lk 0 | 0 \rk  (2\pi)^3 2 \omega_{k_1} \delta^{(3)}(\vv{k}_1 - \vv{k}_2)\]
Note, that the ground state is normalized to $\lk 0 | 0 \rk = 1$.\\
To nomalize a state, we have to look at a wave package, as a one particle state with a wafe function $f$:
\[ | 1 , f\rk = \int \dd \tilde{k} f(k) |k\rk\]
where $f(k)$ is the wave function in momentum space. The norm then is given by
\[ \| | 1, f\rk \|^2 = \lk 1, f| 1, f\rk = \int \dd\tilde{k}_1 f^*(k_1) \dd \tilde{k}_2 f(k_2) \lk k_1 |k_2\rk = \int \dd \tilde{k}_2 | f(k_2) |^2\]
In the last step, the just derived expression for the matrix element was used, and the $\delta$-function was evaluated, which reduced the double integral to just one. Thus, the norm is given by the quared wave function.\\
A normalizable $n$-particle state then is
\[ |n, F\rk = \frac{1}{\sqrt{ n! \int \dd \tilde{k}_1 \ldots \dd\tilde{k}_n | F(k_1, \ldots k_n) |^2}} \int \dd \tilde{k}_1 \ldots \dd \tilde{k}_n F(k_1, \ldots k_n) a^\dagger(k_1) \ldots a^\dagger(k_n) |0\rk\]
The $n!$ appears, because there are $n!$ permutations of the $a^\dagger$ which all yield the same result. $F$ is without loss of generality a bose function; even if it wasnt, only the Bose symmetric part of it would contribute. $F$ is therefore symmetric under exchange of two indices.\\
A state with $n$ particles in mode $\psi(k)$ then can be expressed as
\[ \frac{1}{\sqrt{n!}} \left( \int \dd \tilde{k} \psi(k) a^\dagger(k) \right)^n | 0 \rk\]
with $\int \dd \tilde{k} | \psi(k) |^2 = 1$. The expressin in the brackets is essentialy a wave package.
\sub{Complex Klein Gordon field}
The Lagrangian of a complex Klein Gordon field is
\[ \LL = ( \p_\mu \phi) ^* ( \p ^\mu \phi) - m^2 \phi^* \phi \]
from this also follows the Klein Gordon equation ($\square + m^2) \phi(x) = 0$.\\
We can now express the two complex fields as two separate real fields:
\[ \phi(x) = \frac{1}{\sqrt{2}} \left( \phi_1(x) + i \phi_2(x) \right)\]
\[ \phi^*(x) = \frac{1}{\sqrt{2}} \left( \phi_1(x) - i \phi_2(x) \right)\]
With these real fields $\phi_1$ and $\phi_2$ the Lagrangian is
\[ \LL = \frac{1}{2} ( \p_\mu \phi_1)^2 - \frac{m^2}{2} \phi_1^2 + \frac{1}{2} (\p_\mu\phi_2)^2 - \frac{m^2}{2} \phi_2^2\]
So we have two indepeandat, equivalent, real Klein Gordon fields of the same mass $m$. If we take a look at the canonical quantization
\[ [ \phi_i (\vx, t), \p_0 \phi_j(\vy, t) ] = i \delta_{ij} \delta^{(3)}( \vx - \vy) \]
the fields do commute for different indices $i$ and $j$. Thus they really are independant.\\
In terms of creation and annihilation operators we have
\[ \phi_i(x) = \int \dd \tilde{k} \left[ a_i(k) e^{-ikx} + a_i^\dagger(k) e^{ikx} \right]\]
This implies, that the complex filed is
\[ \phi(x) = \int \dd \tilde{k} \left[ \frac{1}{\sqrt{2}} \left(a_1(k) + i a_2(k)\right) e^{-ikx} + \frac{1}{\sqrt{2}} \left( a_1^\dagger(k) + i a_2^\dagger(k)\right)e^{ikx} \right]\]
We now call the expressions in the brackets $a(k)$ and $b^\dagger(k)$:
\[ a(k) \equiv a_1(k) + i a_2(k), ~~~ b(k) \equiv a_1(k) - i a_2(k)\]
These operators have a more physical interpretation than the $a_i(k)$.\\
We can now calculate the commutator
\[ [a(k), a^\dagger(p)] = (2\pi)^3 2 \omega_k \delta^{(3)}(\vp - \vv{k})\]
and similarly
\[ [b(k), b^\dagger(p)] = (2\pi)^3 2 \omega_k \delta^{(3)}(\vp - \vv{k})\]
also the mixed commutator vanishes: $[a(k), b^\dagger(k)] = 0$.\\
The four momentum operator $P^\mu$ which has been introduced earlier
\[ P^\mu = \int \dd \tilde{k} \cc k^\mu \left( a_1^\dagger(k) a_1(k) + a_2^\dagger(k) a_2(k)\right)\]
can now be written with the newly defined operators
\[ P^\mu = \int \dd \tilde{k} \cc k^\mu \left( a^\dagger(k) a(k) + b^\dagger(k) b(k) \right)\]
There is however no real advantage of using the new operators here, it basically is the same. A real difference emerges when looking at the conserved charge. Its current is given as
\[ j^\mu = -i:(\p^\mu \phi)\phi - \phi^\dagger(\p^\mu \phi)|\]
where the $:$-notation just means, ordering the terms in a way, that the vacuum state is zero. The conserved charge then is
\[ Q = \int \dd ^3 x j^0(x) = \int \dd \tilde{k} i \left( a_1^\dagger(k) a_2(k) - a_2^\dagger(k) a_1(k) \right)\]
for the old operators. This constellation of operators describes the taking away of particle 2 with $k$ and replacing it with particle 1 with momentum $k$ and vice versa for the second term. This is not a very clear and physical interpretation. However, if we write the charge with the new operators we find
\[  Q = \int \dd \tilde{k} \left( a^\dagger(k) a(k) - b^\dagger(k)b(k)\right)\]
Here are no mixed terms, so this can be interpreted as $N_a- N_b$, the difference in number of particles, where $N_a = a^\dagger a$ and similarly for $N_b$. This interpretation then yields, that $a^\dagger(k)$ creates particles of charge $Q = +1$ and $b^\dagger(k)$ creates anti particles with charge $Q = -1$. The total charge $Q = N_a - N_b$ is conserved, this even holds for interacting fields, as long as we have a $U(1)$ symmetry. So, the only posibility to change $N_a$ is a pair creation and annihilation.\\
An example of this formalism is the electric charge of $\pi^+$ and $\pi^-$ or the hyper charge or strangeness of $K^0$ and $\bar{K}^0$.
\sub{Covariant commutation relations}
Earlier we looked at equal time commutators, now lets take a more general approach. We first start the discussion just for real Klein Gordon fields
\[ \phi(x) = \int \dd \tilde{k} a(k) e^{-ikx} + \int \dd \tilde{k} a^\dagger(k) e^{ikx}\]
We now call the first term, the term for positive energy $\phi^+(x)$ and the second one, for negative energy $\phi^-(x)$. Now, the commutator for arbitrary, four dimensional $x$ and $y$ is given as
\[ [ \phi(x), \phi(y)] = [\phi^+(x) + \phi^-(x), \phi^+(y), \phi^-(y)]=[\phi^+(x), \phi^-(y)] + [ \phi^-(x), \phi^+(y)]\]
In the last step we used, that the remaining two commutators would vanish, because they are proportional to $[a(p),a(k)]$ or $[a^\dagger(p), a^\dagger(k)]$ which are equal to zero.  Now, as the first commutator is kinda the same as the second one, just with interchanged $x$ and $y$ we only calculate the first one
\[ [\phi^+(x), \phi^-(y)] = \int \dd \tilde{k}_1 \dd \tilde{k}_2 e^{-ik_1x} e^{ik_2x} [a(k_1), a^\dagger(k_2)] = \int \dd \tilde{k} e^{-ik(x-y)} \equiv i \Delta^+(x-y)\]
In the last step we again used the commutator relation $[ a(k_1), a^\dagger(k_2)] = (2\pi)^3 2 \omega_{k_1} \delta^{(3)} (\vv{k}_1 - \vv{k}_2)$. The resulting expression is called $i \Delta^+$, where $\Delta^+$ only depends on the difference $x-y$.\\
From this we get the whole commutator
\[ i \Delta(x-y) \equiv [ \phi(x), \phi(y)] = i \Delta^+(x-y) + i \Delta^-(x-y) = \int \dd\tilde{k} \left( e^{-ik(x-y)} - e^{ik(x-y)}\right)\]
This can be writtes as a sine, so that
\[ i\Delta(x) = -2 i\int \dd\tilde{k} \sin\left( k x\right)\]
Some properties of these functions $\Delta(x)$ are
\begin{itemize}
\item $\Delta(x)$ is Lorentz invariant, that is to say it is only dependant on $x^2$ and the sign of $x^0$ because only they are Lorentz invariant (?).
\item $\Delta(x-y)|_{x^0 = y^0}$ is the equal time commutator, which is zero. This implies that $\Delta(x-y) = 0$ for $(x-y)^2 <0$ because in some Lorentz frame $x'_0 = y'_0$ for the same physical point. In other words, the commutator will vanish outside of the forward and backward light cones. This is sometimes also called micro causality:\\
Measurements of fields $\phi$ at $x$ and $y$ are independat, that is to say, they do not influence each other, for space like separations. Since the signal between $x$ and $y$ would need to exceed the speed of light.
\end{itemize}
The explicit form of $\Delta(x)$ is
\[ i \Delta(x) = \frac{i}{4 \pi | \vx|} \frac{\p}{\p | \vx|} \begin{cases}
J_0 (m \sqrt{x^2}), ~~~~~~ x^0 > |\vx|\\
0 ,~~~~~~~~~~~~~~~~~~ - | \vx| < x^0 | |\vx| \\
-J_0(m\sqrt{x^2}), ~~~~ x^0 < |\vx|
\end{cases}
\]
With the commutator, also the vacuum expectation value of two Klein Gordon fileds is calculatable
\[ \lk 0 | \phi(x) \phi(y) | 0 \rk = \lk 0 | \phi^+(x) \phi^-(y) | 0 \rk = \lk 0 | [\phi^+, \phi^-(y)] | 0\rk = \lk 0 | i \Delta^+(x-y) | 0  \rk \]
In the first step we used, that every other term would vanish, because it would be equal with $a(k)$ acting on $|0\rk$. In the second step we rewrote $\phi^+(x) \phi^-(y) = [\phi^+(x), \phi^-(y)] + \phi^-(y)\phi^+(x)$. The extra term vanishes though for the same reason. Now, because $\lk 0 | 0 \rk = 1$ we get
\[ \lk 0 | \phi(x) \phi(y) | 0 \rk = i \Delta^+ (x-y) = \int \frac{ \dd^3 \vv{k}}{(2\pi)^3 2 \omega_k} e^{-ik(x-y)}\]

\section{Propagatoren}
\section{Gupta Bleuler Quantisierung des Photons}

\chapter{S-Matrix, LSZ Reduktionsformel}

\chapter{Störungstheorie}	
\section{Feynman Regeln der QED}
\section{Wirkungsquerschnitte und Zerfallsraten}
\section{radiative Korrekturen}

\end{document}
